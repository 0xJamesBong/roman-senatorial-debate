\documentclass[9pt]{article}

\usepackage{amsmath,amssymb}
\usepackage[utf8]{inputenc}
\usepackage{ctex}
\usepackage{epigraph}
\usepackage{amsthm}
\usepackage{multicol}
\usepackage{array}
\usepackage{booktabs}
\usepackage{threeparttable}

% Remove paragraph indentation for cleaner look
\setlength{\parindent}{0pt}
\setlength{\parskip}{1em}

% Reduce line spacing for more compact text
\linespread{0.2}

% Define theorem environments
\newtheorem{theorem}{Theorem}
\newtheorem{lemma}[theorem]{Lemma}
\newtheorem{corollary}[theorem]{Corollary}
\newtheorem{proposition}[theorem]{Proposition}
\newtheorem{definition}[theorem]{Definition}
\newtheorem{example}[theorem]{Example}
\newtheorem{remark}[theorem]{Remark}
\newtheorem{axiom}[theorem]{Axiom}

\renewcommand{\proofname}{Proof}


% Set CJK main font (for Chinese/Japanese/Korean characters)


\setCJKmainfont{BabelStone Han}




% You can also use \newfontfamily for custom non-CJK fonts if needed
% \setCJKmainfont{JyutcitziWithPMingLiURegular}[Path = ./, Extension = .ttf]
% \setCJKmainfont{JyutcitziWithSourceHanSerifTCRegular}[Path = ./, Extension = .ttf]



% \newfontfamily{\jczPMingLiU}{JyutcitziWithPMingLiURegular}[Path = ./fonts/, Extension = .ttf]
% % This has the best rendition for latin characters 
\newfontfamily{\jcz}{JyutcitziWithSourceHanSerifTCRegular}[Path = ./fonts/, Extension = .ttf]
\newcommand{\jcztext}[1]{{\jcz #1}}

\newfontfamily{\batang}{batang}[Path = ./fonts/, Extension = .ttf]
\newCJKfontfamily\koreanfont{Batang}[Path = ./fonts/, Extension = .ttf]
\newfontfamily{\taigi}{GentiumBookPlus-Regular}[Path = ./fonts/, Extension = .ttf]


% Load ruby package for furigana (Ruby text)

\usepackage{ruby}







% IDC - ideographic description characters
% https://en.wikipedia.org/wiki/Chinese_character_description_languages#Ideographic_Description_Sequences

\newcommand{\superimpose}[2]{{%
  \ooalign{%
    \hfil$\m@th\text{#1}\@firstoftwo\text{#2}$\hfil\cr
    \hfil$\m@th\text{#1}\@secondoftwo\text{#2}$\hfil\cr
  }%
}}


% Define the \tb command

\newcommand{\tb}[2]{%
\scalebox{2}[1]{
\ooalign{%
    \hfil\raisebox{0.25em}{\text{\scalebox{0.33}{#1}}}\hfil\cr % Top text, squished and raised
    \hfil\raisebox{-0.25em}{\text{\scalebox{0.33}{#2}}}\hfil\cr % Bottom text, squished and lowered
  }%
  }
}

% The \lr command - can be combined with \tb
\newcommand{\lr}[2]{
  \scalebox{0.5}[1.0]{#1}\scalebox{0.5}[1.0]{#2}\!\!
}

% Define the \ul command for upper left positioning - for characters like 疒
\newcommand{\ul}[2]{%
  \ooalign{%
    \hfil#1\hfil\cr  % Top text (unscaled)
    \hfil\hspace{0.3em}\scalebox{0.8}{#2}\cr % Bottom text (scaled and raised)
    % \hfil\raisebox{0.2em}{\scalebox{0.5}{#2}}\hfil\cr % Bottom text (scaled and raised)
  }%
}

% Define the \tone command for upper right positioning of a diacritic
\newcommand{\tone}[2]{%
  \ooalign{%
    \hfil#1\hfil\cr  % Main text (unscaled)
    \hfil\hspace{0.9em}\raisebox{0.3em}{\scalebox{0.8}{#2}}\hfil\cr % Tone mark (scaled and raised)
  }%
}

% for vertical Chinese boxes
\usepackage{graphicx} % for \rotatebox

\newfontlanguage{Chinese}{CHN}

\setCJKfamilyfont{BabelStoneVert}[RawFeature={vertical;+vert},Script=CJK,Language=Chinese,Vertical=RotatedGlyphs]{BabelStone Han}

\newcommand*\CJKmovesymbol[1]{\raise.35em\hbox{#1}}
\newcommand*\CJKmove{\punctstyle{plain}% do not modify the spacing between punctuations
  \let\CJKsymbol\CJKmovesymbol
  \let\CJKpunctsymbol\CJKsymbol}

% Define a new environment for vertical text
\newcommand{\VertCell}[1]{\rotatebox{-90}{\CJKfamily{BabelStoneVert}\CJKmove #1}}


% *-----------------------------------------------------------------------*
% | Packages and formatting                                               |
% *-----------------------------------------------------------------------*
% *-----------------------------------------------------------------------*
% | Math & Equations     |
% *-----------------------------------------------------------------------*
\usepackage{amsmath} % For advanced math formatting
\usepackage{amssymb} % For mathematical symbols
% \usepackage{tikz} % For drawing logic decision trees


% *-----------------------------------------------------------------------*
% | Table Management                                                      |
% *-----------------------------------------------------------------------*


\usepackage{graphicx}
\usepackage{array}
\usepackage{tabularx}
\usepackage{tabularray}

\usepackage{float}      % Add the float package
\usepackage{longtable}


\usepackage[table,xcdraw]{xcolor}
 \definecolor{stone}{rgb}{0.7,0.9,0.7} % light green
\definecolor{lightblue}{rgb}{0.8,0.9,1} % light blue
\definecolor{liquid}{rgb}{1,0.8,0.9} % light pink
\definecolor{gold}{rgb}{1,0.9,0.7} % light yellow
\definecolor{gas}{rgb}{0.7,0.8,1} % light blue

\newcommand{\elcol}[3]{\if\relax\detokenize{#1}\relax\begin{tabular}{@{}c@{}}#2\\[0.3em]#3\end{tabular}\else\cellcolor{#1}\begin{tabular}{@{}c@{}}#2\\[0.3em]#3\end{tabular}\fi}




% super long table

\usepackage{pdflscape} % For rotated pages

\setlength{\arrayrulewidth}{0.5mm} % Optional: to thicken table lines
\renewcommand{\arraystretch}{1.5} % Optional: to increase row height


\usepackage{makecell} 


\usepackage{threeparttable}


% *-----------------------------------------------------------------------*
% | Chinese and Soochow Numerals                                          |
% *-----------------------------------------------------------------------*
% numerals.tex
% Define Chinese and Soochow numerals for chapter management

\newcommand{\soochowNumeral}[1]{%
  \ifnum#1<10
    \ifcase#1 〇\or 〡\or 〢\or 〣\or 〤\or 〥\or 〦\or 〧\or 〨\or 〩\fi%
  \else
    \ifnum#1<20
      〸\soochowUnits{\numexpr#1-10\relax}%
    \else
      \ifnum#1<30
        〹\soochowUnits{\numexpr#1-20\relax}%
      \else
        \ifnum#1<40
          〺\soochowUnits{\numexpr#1-30\relax}%
        \else
          \ifnum#1<50
            卅\soochowUnits{\numexpr#1-40\relax}%
          \else
            \ifnum#1<60
              〥十\soochowUnits{\numexpr#1-50\relax}%
            \else
              \ifnum#1<70
                〦十\soochowUnits{\numexpr#1-60\relax}%
              \else
                \ifnum#1<80
                  〧十\soochowUnits{\numexpr#1-70\relax}%
                \else
                  \ifnum#1<90
                    〨十\soochowUnits{\numexpr#1-80\relax}%
                  \else
                    \ifnum#1<100
                      〩十\soochowUnits{\numexpr#1-90\relax}%
                    \fi
                  \fi
                \fi
              \fi
            \fi
          \fi
        \fi
      \fi
    \fi
  \fi
}

\newcommand{\soochowUnits}[1]{%
  \ifnum#1=0
  \else
    \ifnum#1<4
      \ifcase#1 \or 一\or 二\or 三\fi%
    \else
      \soochowNumeral{#1}
    \fi
  \fi
}

\newcommand{\chinesenumeral}[1]{%
  \ifnum#1<10
    \ifcase#1 〇\or 一\or 二\or 三\or 四\or 五\or 六\or 七\or 八\or 九\fi%
  \else
    \ifnum#1<20
      十\chinesenumeral{\numexpr#1-10\relax}%
    \else
      \ifnum#1<30
        二十\chinesenumeral{\numexpr#1-20\relax}%
      \else
        \ifnum#1<40
          三十\chinesenumeral{\numexpr#1-30\relax}%
        \else
          \ifnum#1<50
            四十\chinesenumeral{\numexpr#1-40\relax}%
          \else
            \ifnum#1<60
              五十\chinesenumeral{\numexpr#1-50\relax}%
            \else
              \ifnum#1<70
                六十\chinesenumeral{\numexpr#1-60\relax}%
              \else
                \ifnum#1<80
                  七十\chinesenumeral{\numexpr#1-70\relax}%
                \else
                  \ifnum#1<90
                    八十\chinesenumeral{\numexpr#1-80\relax}%
                  \else
                    \ifnum#1<100
                      九十\chinesenumeral{\numexpr#1-90\relax}%
                    \fi
                  \fi
                \fi
              \fi
            \fi
          \fi
        \fi
      \fi
    \fi
  \fi
}


% *-----------------------------------------------------------------------*
% | Table of contents & Chapter management                                |
% *-----------------------------------------------------------------------*

\renewcommand{\figurename}{圗} % So figures would be labeled with 圗 instead of "figure"

% * * * Now for the table of contents
\renewcommand{\contentsname}{目錄} % Traditional Chinese characters for "Contents"
\setcounter{secnumdepth}{0} % no numbering for sections 
% Increase chapter title size in TOC
\usepackage{tocloft} % For customizing table of contents

% Redefine how chapter numbers are displayed in the TOC using Soochow numerals

% comment this out if you don't want to use the soochow numerlas
\renewcommand{\numberline}[1]{\soochowNumeral{#1}\hspace{1em}} % Use Soochow numerals for TOC chapter numbers
\makeatother


% create index - run \makeindex in the document
\usepackage{makeidx}
\makeindex



% % Custom chapter title formatting with Chinese numeral chapter numbers
\usepackage{titlesec}

% Custom chapter title formatting with Chinese numeral chapter numbers
\titleformat{\chapter}[block] % 'block' means the title appears on a new line
  {\Huge\bfseries} % Font size and bold formatting for the title
  {\soochowNumeral{\thechapter}} % Chinese character for chapter number
  {1em} % Space between the number and the title
  {\Huge} % Custom style for the chapter title itself (can modify)



% *-----------------------------------------------------------------------*
% | Document begins                                                       |
% *-----------------------------------------------------------------------*

\title{Introducing the Roman Senatorial Debate}
\author{James Bong}
\date{9th July 2025}

\begin{document}
% Avoid overfull hbox
\sloppy

% Initialize jyutcitzi processing
% \jcz{}

\maketitle



% \section{Epigraphs}

\begin{table}[H]
\centering
\small
\begin{tabular}{p{0.45\textwidth}p{0.45\textwidth}}
\epigraph{Speech was bestowed upon man to disguise his thoughts.}{---Tiberius (as quoted by Tacitus, Annals VI.6)} &
\epigraph{In the Senate, men do not speak the truth. They speak to win.}{---(Paraphrase of sentiments found throughout Sallust and Cicero)} \\
\epigraph{The Senate is a council of kings.}{---Ennius (quoted by Cicero)} &
\epigraph{Let arms yield to the toga, the laurel of the triumph to the tongue of the orator.}{---Cicero, Pro Milone} \\
\epigraph{Senators are not ashamed to praise whom they must, nor to attack whom they dare.}{---Tacitus, Annals I.74} &
\epigraph{The more corrupt the state, the more numerous the laws.}{---Tacitus (Ab excessu divi Augusti)} \\
\end{tabular}
\end{table}

\begin{table}[H]
\centering
\small
\begin{tabular}{p{0.45\textwidth}p{0.45\textwidth}}
\epigraph{The parliament passes some acts or decree which may have the most devastating consequences, yet nobody bears the responsibility for it. Nobody can be called to account. For surely one cannot say that a Cabinet discharges its responsibility when it retires after having brought about a catastrophe. Or can we say that the responsibility is fully discharged when a new coalition is formed or parliament dissolved? Can the principle of responsibility mean anything else than the responsibility of a definite person?}{---Adolf Hitler, Mein Kampf} &
\epigraph{Parliament can do anything but make a man a woman and a woman a man.}{---Sir Edward Coke (attributed)} \\
\epigraph{The British Constitution has always been puzzling and paradoxical. It is an unwritten constitution, and yet it is written all over the place.}{---E.M. Forster} &
\epigraph{The British Parliament is like a curious old clock: its hands move, but no one quite knows why.}{---Aneurin Bevan (paraphrased)} \\
\end{tabular}
\end{table}

\begin{table}[H]
\centering
\small
\begin{tabular}{p{0.45\textwidth}p{0.45\textwidth}}
\epigraph{The House of Commons is the longest running farce in the West End.}{---Clement Attlee (allegedly said in frustration during debates)} &
\epigraph{Parliament is a talking shop.}{---Common saying, derisive or admiring depending on tone} \\
\epigraph{It is the duty of the opposition to oppose.}{---Stanley Baldwin} &
\epigraph{In the Parliament of a free nation, all men are entitled to speak, and all others are entitled not to listen.}{---Attributed to Lord Melbourne} \\
\end{tabular}
\end{table}

\begin{table}[H]
\centering
\small
\begin{tabular}{p{0.45\textwidth}p{0.45\textwidth}}
\epigraph{Congress is so strange. A man gets up to speak and says nothing. Nobody listens—and then everybody disagrees.}{---Boris Marshalov (often misattributed to Will Rogers)} &
\epigraph{In America, anybody can be president. That's one of the risks you take.}{---Adlai Stevenson} \\
\end{tabular}
\end{table}


\section{Introduction}

I grew up and was trained in the British Parliamentary style in my intellectual youth. It might perhaps be rather unbecoming, embarassing, to still talk about one's "debating days". The person smells as if he has plateaued in the debating arena, and he's hankering to his glorious past at the expense of his interlocuter's patience. But the British Parliamentary style was incredibly influential to me, on my mode of internal dialectic, my rhetoric style, and substantive philosophical commitments. It was particularly impactful because prior to the British Parliamentary style, none of the various formats of debate I was acquainted with offered a comparable degree of intellectual challenge, game complexity, or even just speech time for one to really make a pie \footnote{oeuvre} out of the motion. I was also convinced that the British Parliamentary style itself could serve as truthseeking, soothsaying machine. I was convinced that was a successful gameified implementation of the socratic dialectic method. The fact it was culturally contiguous with its namesake and inspiration the British Parliament, and that it shares nontrivial similarities in both operation and evolutionary tendencies (as I observed them when I was a competitive debater), made this position very attractive. The historical and cultural weight of the style, and the respect accorded by the venues in which debates of this style were often held, automatically lends legitimacy to one's speech, and therefore, a sense of responsibility - the British Parliamentary style was the style in which Oxford debated the motion "This House Would fight for King and Country", a debate that Hitler listened to on the radio and a debate that he frequently cited in his military decisions concerning Britain. The demands and standards of dynamism, quickwittedness, and argumentative flare, sanctioned by the rambunctious institution known as the point of information, was entirely alien to the repetoire of Sinitic intellectual discourse methodologies. I was even convinced that a backward, intellectually inferior, vocabularily and logically impoverished language, through the British Parliamentary style, could be beaten like glass - and acquire the latent and subterrenean instruments of easy good thinking that the English speaker has taken for granted - what I call "English Rationality". In my eyes, all other debate styles such as All Parliamentary, and World Schools', are mere sloppy imitations of the superior British Parliamentary style, watered down for those with weak intellectual palates.

I no longer agree with this. Or rather, I think although the British Parliamentary style can still deliver some of these promises to young and impressionable minds - particularly the intellectually honest, even if they are immature - these benefits do not scale or carry over into the more advanced stages of the style. Once a person or a society has progressed beyond the novical stages of the British Parliamentary experience, the style rapidly degrades into a chamber that breeds and amplifies intellectual pathologies.


A person, and by extension its debating society, quickly approaches the limits of the British Parliamentary style if they are not intellectually honest or responsible - if they don't check the stuff coming out of their mouths against God - and most people today don't do this, ever in their lives. They never ask "what if I am wrong?" - and they never ever imagine a voice or a being or a mind - God - being there being there to give the ultimate answer. They never test their propositions twice in their head. This laziness is of course common amongst all humans. Intellectual fastidiousness is a rare and precious mutation. And the discpline to commit to that fastidiousness without accidental descension to insanity is even rarer. But the point is, this intellectual laziness, along with other mechanisms and forces, form the prime cause of group mind drift. Since all the minds that make up a tournament have minimal will or intellectual steadfastness of their own, a tournament of minds becomes a brownian motion with drift, as the implicit ideological vectors embedded in the choice of motions, equity policies, adjudication programming, among other things, supervene their latent ideological baggage on the intellectual body of debating individuals. \footnote{「風行而草偃,披之無令而民從也。」
"When the wind moves, the grass bends; the people follow him without being commanded." ——《荀子‧君道》 (Xunzi – On Kingship).
}

The British Parliamentary style does not punish intellectual lazy individuals. In fact, no debating style does. No debating style has a mechanism to impose a duty to be intellectually honest and fastidious to oneself. Nor is there encouragement to do so. And so the downstream problems of that intellectual dishonest multiply. Why is there no such imposition? I think the primary reason is that no debating format has ever come into terms with these problems. The British Parliametnary style is guilty of this failure, only because it is the most advanced style, and has the most developed debating community, and is therefore the first to come into grips with this problem.

There are many other conditions that can beget degenerate intellectual group behaviour, and therefore push against the limits of the British Parliamentary style - but regardless of what they are, once a debating community reaches the limits of the British Parliamentary style, it starts to emit harmful intellectual radiation, and it gradually irradiates its debaters, its debating community, and even the politics of the state. In its most mature form, and in the most celebrated events where idols are made, the British Parliamentary style has become a farce.

It has become necessary for one to enumerate and declare the problems and ills that the British Parliamentary style manifests before us, the mechanisms and design flaws responsible for them, and the higher order downstream impacts on the intellectual developments of participants and wider society.

\begin{itemize}
    \item \textbf{The Dictatorship of the Adjudicator Class.} The adjudicator 
    
    - which is a bug that makes the entire game, and its outgrowth ecosystem (debate teams, clubs, debate tutorial schools, national training squads and teams) extremely prone to intellectual capture.

    The British Parliamentary debate organizing authorities and the culture that they themselves and their participants have upheld voluntarily imposes expectations, norms, and customs, on the adjudicator. These norms are toothless in force, as violating them solicits no punishment - noncompliance solicits at best only embarassment, humiliation, or bin-room adjudication duties. 

    But the problem is not at all noncompliance with these norms. The vast and overwhelming number adjudicators willingly and happily adhere to the programming instilled upon them from senior and esteemed adjudicators, and noncompliance is almost unfailingly the result of intellectual misfiring, incompetence, or inexperience - not malice or intellectual unyieldiness.

    This is to say, that the internal policing is done very well. The adjudication programming and its enforcement mechanism via culture imposition and prestige distribution is very effective. There are extremeley rare cases, none actually in my experience, where the adjudicator has ruled entirely based on his own personal belief system entirely untempered by the expectations of the adjudicator class or the debating class. The debater can be reasonably confident what metric the adjudicator will be using to judge the debate, what the conditions of victory and defeat are, and how the adjudicator will apply the metric of adjudication. The outcome is reasonably predictable, and therefore, the game is reasonably fair. There is a game. 

    The fact that such a convergence of expectations could be forged is a miracle. It is almost exactly the same as in Common Law courts where Judge, Prosecution, and Defence all operate on the same standards and expectations - and it's fair game. Even with the introduction of a jury can one still operate with a stable set of expectations. 
    \footnote{Does Sharia courts have the same convergence of expectations?}
    
    So what's the problem? The problem is that it is very effective in imposing the wrong things. The problem is that the programming imposed is bad.


    There is no real judgement. There's only computation. 
    This is why British Parliamentary debate, just like Common Law courts in Britain, Hong Kong, Singapore, and elsewhere, are all 做大戲 - they're just shows. 

    The only exception is America, where court battles are still genuine battles. 


    \item \textbf{Equity} - which basically prevents hate speech. To those who see value in Cicero's deployment of as hominem attacks as effective highlights of praxological inconsistencies, this custom is highly regrettable. The entire custom abstracts the debate away from reality, breaks the connection between the speaker and his speech, notwithstanding the fact that reality is a place populated by instances of fierce burning hatred and irreconcilable differences.
    \item \textbf{Stakelessness - the nonexistent bumps against the decrepit} - The game does not have a proxy for a real actual stake that often pushes people into politics, like money or power. People debate because there's politics. There's politics because as a solution to conflict, it's cheaper than violence. There is conflict because people have opposing stakes and interests. Debates abstracts all of that away into just intellectualism, without stakes or interests. At best only the speaker's reputation is at stake.
    \item \textbf{Persuasion} - There is no real room or need for persuasion. The target of persuasion is the judge not the interlocutor. This encourages intellectual antagonism without restraint beyond equity nicities. Why bother agreeing anything the other side has said? You're not in the business of persuading the other side.
    \item \textbf{Intellectual dishonesty, and mindless intellectual outgrowth.} The British Parliamentary style, like all debate formats, impose on the debater a side. The debater does not choose his or her own side. The reason for this is quite straightforwardly obvious - it's pedagogical. "It's about seeing an issue from different sides". That's fair enough, and very probably very important for young and impressionable minds. The confiscation of the debater's choice of a side, forces the debater to confront and come into terms with the arguments of a side he may not necessarily agree with, and in doing so, compels his neuroplasticity to do its magic. Indeed, very probably very important for young individuals. If they are to ever be good at carrying out the dialectic, with others, and more importantly, within themselves, then they must be able to argue from this view, that view, and possibly the view from nowhere. The debater is trained to be the advocate, the devil's advocate, His Majesty's Government, His Majesty's Most Loyal Opposition, all at the same time.
    
    What this means fundamentally, is that the British Parliamentary style does not care what you really think. Nobody cares what you think really. And indeed, as a debater does more and more rounds, he himself cares less and less what he really thinks as well. Because the debate tournament is an reiterated game, and a competitive game, with prizes of prestige, recognition, and sometimes possibly money attached. Just search for the argument that can most plausibly win. If that happens to be an argument you like, great - but if not, who cares? Just win the damn debate.
    
    Why is this a problem? This is an problem, because it induces the disfigurement of the soul, and of the body polity, given enough time.

(1) The disinvestment of one's soul from the argument. 

In the British Parliamentary style, there is little reason to offer arguments grounded in one’s own genuine beliefs. The format mildly discourages — and at times punishes — self-forwardance. Why risk your mind on the floor, when you can borrow a prefabricated line of reasoning from some clever intelligence, orga mecha or otherwise, who came before you? Why put a piece of your soul on display, only for it to be ruled against and humiliated by a room full of strangers? Just recycle what works.

When one does make an argument out of true belief, one necessarily invests the soul. That is what makes it costly. That is what makes it real. The argument becomes a kind of vessel, a horcrux of the self. And in turn, when that argument succeeds or fails, something deep within the person is transformed — clarified, wounded, or reborn. And when that happens, the chamber, the tournament, society, transforms along with the individual person.

The debate that disallows the debater from choosing his own position is one that compels the debater to be an actor. The debate compels him to fabricate conviction in a speech he does not necessarily believe. Any doubt must be castrated by the debater himself. \footnote{Although it should be theoretically possible to argue on the Opposition to any motion by merely asking questions, by evoking doubt, and by skepticism, I have never seen such a thing done in practice. If the motion before the house "This House Believes that God does not exist", it is entirely plausible to argue on the Opposition just by appealing to the intuition of humanity, so as to "see" the implausiblity of the motion vis-a-vis an invokation of a feeling of "there's something funny going on here".}

In the absence of that stake - when the soul is withheld from the argument — absurdity finds license. Disconnected from belief, argument becomes play. Not in the noble sense of ludic experimentation, but in the shallow, nihilistic sense of intellectual theatre. The debater learns to perform, not to think. To win, not to discover.

The drive to find new truths, make new mistakes, and therefore new persons — is never activated. The mind never undergoes self-operation. And so the debater is never transformed or clarified as a person.

The intellectual battle, therefore, is always fake. It is never gloves off. No one is bruised and no one bleeds. And this produces a class of thinkers who are apathetic, dishonest, lazy, — all equally disqualified from truth-seeking. In a jury of such epistemic cripples, Condorcet’s Theorem guarantees convergence on the wrong answer. 

In the worst case scenario, the debater adopts knowingly or unknowingly whatever family of arguments that maximises his probability of winning as his own genuinely privately held belief - so as to reconcile the discomfort, or to make comfortable, that arises from cognitive and intellectual dissonance. This is why university debate tournaments breed so many leftists, because leftists arguments win. 

This is disfigurement of the soul, the furthest thing from ennoblement. The strategy favoured by the game theoretic pressure on the debater is one of submission and self-annulment, not domination, self-assertion or self-discovery. Denial and not creation. 

Naturally, never is one's belief ever put to the test. One can conjure up arguments, but one has no stake or loyalty to those arguments - when those arguments fall, one's ego is not bruised, if one has learnt to disinvest one's ego from the argument made. Arguments made are not arguments one believes in, and arguments made are made for the sake of victory, not for the sake of announcing sincere belief. You are more less likely to find genuine epistemic discovery in a debate than you are to find the moon made of cheese. Even if you do not treat the debate as an instrument of epistemic discovery, this is still damaging to the individual. Very damaging. It is damaging to the individual via the same mechanisms in which porn damages the expectations and therefore the dyanmics of life life of the individual, and how video games infects people with hoodish behaviour. 

Watch your thoughts, for they become words; watch your words, for they become actions; watch your actions, for they become habits; watch your habits, for they become character; watch your character, for it becomes your destiny. The absolution of the mind from intellectual liability leads to the watchlessness of thoughts. Who cares what you think? No one - not even yourself. 
    
The ultimate cumulative effective of this kind of pressure over many, many rounds of debate, over many, many tournaments, inflicted on class after class of young adults, is mass intellectual drift with no real input or pushback. A mass of mindless evolutionary outgrowth mushrooms into existence. 

What is the substantive content of this mindless evolutionary outgrowth? What is it intellectually or dogmatically committed to? Well, whatever it is, there is aname for this kind of phenomenon. It is called "intellectual capture".

One should note that the ultimate reason how intellectual honesty could be enforced is the threat of violence. Just because a debate format allowing for the debater to choose their own side they want to debate for will not necessarily mean he will put his soul into the argument, nor will it necessarily mean he will make arguments he genuinely believes in. He can still very well fake it all. He can still lie. 

Only when liars are punished, and disproportionately punished, will intellectual honesty naturally arise. And only when accusations of lying and dishonesty can be challenged with the threat of violence, will the charge of lying and dishonesty be weighty enough to be taken seriously.

“An armed society is a polite society. Manners are good when one may have to back up his acts with his life.” Equivalently, an armed debate chamber is a honest debater chamber.



\item \textbf{Intellectual Capture}

\item \textbf{No Real Clash or Contension - both sides actually agree.} Oftentimes one might find that both sides of the debate actually agreeing with each other. They're merely disagreeing on the details of the policy. The Opposition often agrees with Government's philosophical and moral charges - in that we should help the poorest, the blacks, the coloured, the disabled, the gays, the colonised, the oppressed, or even better, etc - all of that is conceded and very nice and dandy, but it's the Government's policy that the Opposition finds intolerable.\footnote{Note, that the usual charges of White Privilege globally enjoyed by the Whites can easily be refashioned into similar charges against the Chinese in South East Asia society - especially in Singapore and Malaysia - but you won't find these arguments as fervently and passionately doled out. It's called the Chinese Privilege. I prophesize, if and once these arguments are made in the South East Asian debating circuit, it will not take long to become a tangible political crisis in the region. Indeed, such discourse has already found itself into Singaporean elections. It must not be allowed to fester to the same degree of prevalance, acceptability, and dispersion as the White Privilege discourse. It will bring disaster to South East Asia.}
    
    Therefore, the contension delivered compared to the contension expected, appears anti-climatic, petty and childish, and stratgically clever but intellectually perfunctorious if not mendacious.  

    It vaguely reminds Marx's charge against bourgeois parliamentary democracy where the real conflict is not between parties inside parliament but between the bourgeois parliament and the proletariat masses.

    Why does this happen? Because (1) it's strategically clever, if not obligatory, as it renders the other's side entire stack of points of philosophical and moral grounds irrelevant, as they were conceded. Effectively, the Government has wasted its own time preaching to the Opposition who never intended to not convert; (2) the debaters don't choose their sides, so they really have no reason 

    The downstream effect of this is that unless the motion is explicitly philosophical, or in debating parlance a "value motion", debaters often no longer make any attempt to make a principle argument. Only in Britain do people still occassionally make the odd principle argument thanks to their long but nonetheless deterioriating intellectual tradition of honesty, uprightness, and dignity. In authoritarian cultures with no real sovereign individuals, like China, Singapore, and Japan, the principle argument is a stupid move, an allergens to the judge.

    \item \textbf{Principle arguments are a waste of time.}
    \item \textbf{Utilitarianism is the one true moral theory - the measure of all things.}
    \item \textbf{Ego Hyperinflation} 
    \item 
\end{itemize}

These weaknesses in the debating format, combined with the natural ego boosting powers of the very exercise of debating, have begotten a class of intellectuals who are:

\begin{itemize}
    \item intellectually dishonest and extremely
    \item Ungentlemanly rhetoric
    \item Divorced from reality
    \item Extremely disrespectful and dismissive towards traditions or rules established through evolution, since their utility may not be readily established through simple speech making made without the intelligence provided by history
    \item Left leaning
    \item Extremely outgroup focused in terms of welfare, big government since there are no costs to debate land government size
\end{itemize}

\section{We need forms of competitive debate - something far more game theoretically unstable}

\subsection{Senatorial Debate Roman}

\begin{itemize}
    \item each debater carries with themselves a vote
    \item Weighted by:
    \begin{itemize}
        \item Their speaker points?
    \end{itemize}
    \item Each person writes down the side they support initially - this is not to be publicly revealed - if revealed - no problem
    \item The difference in the pros and cons are to be fought by the debaters
    \item So if there are 8 people debating, and all 8 voted for ``pro'', and nobody changed sides, the 8-0 points will awarded to 8 people, so 1 point each.
    \item But if it began as 7 for 1 against, then 6 points are to be up for grabs. If it ended as 3 for and 4 against, then the 6 points will be shared amongst the 4 winning, so 1.5 points each
    \item This is the incentive to switch.
    \item The first speaker gets 2 votes
    \item Points to the winning side - points divided
    \item Points to the person who stuck to the end
    \item Minor points to people to people who stuck to the end
    \item Win
    \item Lose
    \item T
\end{itemize}

The individual pay out matrix of a person:

\begin{table}[h]
\centering
\begin{tabular}{l|cc}
\toprule
 & Stick & Switch \\
\midrule
Win & 3 & 1 \\
Lose & 2 & 0 \\
\bottomrule
\end{tabular}
\end{table}

Stick Switch

Win   | 3    , 1

Lose  |  2   , 0

The winning side confers a certain number of points

We need a structure that incentives to switch

\begin{itemize}
    \item If all 8 debaters started with pro then the game theoretically wise thing to do is just to vote immediately.
\end{itemize}

Or rather, we need a system anchors then to a particular position - their ``interests''

The guiding philosophy of this attempt to redesign a debate format from scratch, is that reality, with its patchwork of real personal interests and individual philosophical commitments, should somehow serve as anchors to reality in this new debate game that we are designing, in the same way that they anchor actual parliamentary debates. The question is then, how?

Obviously, we should bear in mind that this is still a game. We are designing a game, that mimics reality, but is not a copy of reality. And in reality, it is possible, though rare, that people's interest do align perfectly. It is possible for a parliament to vote unanimously, without interfercer or underhand meddling. However, that would not make for a very good debate game. This suggests that while debaters should be allowed to choose their own initial sides, based on their own personal beliefs, we should introduce some kind of entropy to encourage or even force disagreement. In traditional debate formats, this disagreement is completely hardcoded, as teams are allotted proposition and opposition slots with zero agency of their own. Here, I suggest perhaps some kind of soft allotment can be considered here.

In reality, aside from just personal beliefs, individuals are motivated to vote based on their interests - monetary interests, familial obligations, relational commitments. It would not terribly reductive to coalese all of that into a number represented by numbers.

Suppose each debater begins with a bag of capital, which represents his interests. His bag is composed entirely of two coins: 1\$ and 2\$. They can either represent ``proposition'' or ``opposition'' - which is randomly determined at the beginning of the debate. For illustration, we will take 1\$ to represent proposition and 2\$ to mean opposition.

\begin{itemize}
    \item At the start of the debate, the debater makes known his view on the motion: whether he's for the motion, or whether he's against the motion - with his coins. These coins are then put into their respective sides' stashs.
    \begin{itemize}
        \item A debater may choose to offer nothing, in which case he does forgoes the stick-to-then-end bonus points when the ultimate vote comes.
    \end{itemize}
    \item After each speech, each debater has to vote. If he votes for the proposition he must vote with the 1\$, and vice versa. His bag of coins therefore constrains how he positions himself.
    \begin{itemize}
        \item He can put in as many coins as there are people debating.
        \item He can choose to offer nothing.
    \end{itemize}
    \item At the end of the debate, the motion is put to a vote.
    \item And the winning side takes the winning stack of coins and half of the losing side, and divides it amongst the winners. The remaining half of the losing stack is then divided amongst those who voted for the losing side since the beginning.
    \begin{itemize}
        \item Note that given this
    \end{itemize}
    \item if there is a tie, the larger stack wins - and is divided amongst the winners. the other stack is then divided amongst those who stuck with their original proposition.
\end{itemize}

The key is perhaps those who stuck to their guns and won will not get any share of the pot


\subsection{Problems of the British Parliamentary style}

\begin{itemize}
    \item Ideological capture by core adjudicators and judges
    \item Absence of real stakes (infinite money fiat)
    \item Fundamentally leaning towards youthful idealism—leading to leftism drift
    \item Performativity—actually both sides agree 
    \item No room for genuine persuasion—the target of persuasion is the judge not the interlocutor
    \item Intellectual dishonesty—there's no real requirement for one to believe what one is saying; there's no confrontation of one's conscience and convictions against realistic or pragmatic considerations
    \item There's no training in politicking
    \item There is no force. Therefore the ultimate maintenance and begetter of parliamentary decorum and intellectual honesty—violence, or the threat thereof—is absent
    \item Extremely disrespectful and dismissive towards traditions or rules established through evolution, since their utility may not be readily established through simple speech making made without the intelligence provided by history 
    \item As per the analytic tradition of anglo-american philosophy, the British Parliamentary style is utterly incapable in supporting or even tolerating the admission of institutions whose legitimacy is unclear, whose purpose or justification cannot be articulated, or whose rationality is absent. Institutions of these kinds are often called customs, traditions, or even just prejudices dogmas, and myths. These institutions are not well-justified because they weren't designed by man. They evolved into being, through trial and error. Only that which has no history can be defined, only that can be defined can be designed. But customs, traditions, prejudices, dogmas, and myths, are all the product of history. They are not well-justified because they are undefinable. How can you define a woman? The truth is that any answer appealing to sex organs, chromosomes, biology, are also intellectually irrigorous like, though not as functionally ludicrous as those who claim a woman is what you want a woman to be. But, examples - 
    \item Social Contract Theory 
    \item Gender Roles 
    \item Marriage
    \item Religion -
    \item Fancy clothes - legal wigs, and robes 
    \item Property Rights 
    \item Sexual Taboos, Manners
    
    This is also the problem of modern politics in the English speaking world, and is a broader effect of the Enlightenment project.
    \item intellectual fictions 
    \item Social Contract 
    Chesterton’s Fence (G.K. Chesterton):


    
    \item Extremely outgroup focused in terms of welfare, big government since there are no costs to debate government size

    \item 
\end{itemize}



\section{Successful implementation proposal}

Here we present the rules of a round of Roman Senatorial Debate.

\begin{enumerate}
    \item  \textbf{Number of debaters.} The standard number of debaters is 8. Speaking time is 7 minutes.
    \begin{itemize}
        \item The Roman Sentorial Debate format should work from 4 debaters to any number of debaters, even beyond 8, but for the purpose of a tournament, rooms should be assumed to accommodate 8 debaters, so the pace would mirror that of the British Parliamentary style.  
    \end{itemize}
    \item \textbf{Choosing sides.} Before the first speech, each debater must reveal in their hand a number of coins. A sum of 1\$ means they support the motion, and so they're on the proposition, Yay (Y). A sum of 2\$ means they're against the motion, and so they're on the opposition, the Nay side (N). Anything else, including 0\$ indicates they're undecided (U). The sides chosen by the debaters must then be recorded, and so they must be seated appropriately. Rearrange the chairs if necessary so it is clear to everywhere all debaters stand. All sums are then pooled into the jackpot. Then they start the debate.
    \begin{itemize}
        \item The imbalance is intentionally weighted in favor of the motion, making it easier to support than oppose. This is partly to incentivize whoever sponsoring or organizing the debate to feel that their time, energy, and money have been adequately spent in advancing an ideological position that they themselves favour.
        \item The higher cost of opposition is also there to ensure that resisting the motion is a matter of genuine conviction, not casual contrarianism—it makes conscience expensive, and therefore meaningful.
        \item The sum could be scaled up to (prop: 10\$, opp: 20\$), or even (prop: 10,000\$, opp: 20,000\$), depending on the tournament's debater demographic. But the ratio should be kept at 1:2. 
        \item Senators who began 
    \end{itemize}

    \item \textbf{Speaking order.} There are multiple ways to determine the speaking order. Here we present two methods, one simple and one sophisticated. 
        \begin{itemize}
        \item \textbf{Simple method.} The speaking queue is ordered by Y, N, U, in that order, until all sides run out. If one side has already run out, the next side continues on the queue.
        \begin{itemize}
        \item  Say you have 8 debaters, 7 Ys, and 1Ns, then the queue is (Y, N, Y, Y, Y, Y, Y, Y).
        \item  Another example, say you have 8 debaters, 3 Ys, 4 Ns, and 1 U, then the queue is (Y, N, U, Y, N, Y, N, N). 
        \item One more example, say 2 Ys, 5 Ns, and 1 U, then the queue is (Y, N, U, Y, N, N, N, N).
        \item The point of stacking the side where there is a large number of supporting senators towards the "end" of the queue is to (1) give the oppoosing side more "time" for them to make clear their point earlier, and so their speech can have more impact, and (2) to give the more numerous side more time to reconsider their position. 
        \end{itemize}

        
        \item \textbf{Sophisticated method.} The speaker order is not determined, but bidded. Whenever the floor is open, senators may bid for the speaking slot. There are three bids. All unsuccessful bids are pooled into the jackpot. The highest bidder gets the speaking slot. If there are no bidders, then the speaker to the left of the last speaker gets the speaking slot. 
        \begin{itemize}
            \item Obviously, this method of determining speaking orders is more monetarily competitive, and involves more strategizing. It also inflates the jackpot.
        \end{itemize}
    \end{itemize}
    
    \item Each senator makes a 7 minute speech. Any sitting may raise points of information during the 1st and 6th minute of the speech. During any time of their speech, they may add money to the pot.
    \item \textbf{Voting and resolution.} When all speeches have been, the chamber shall undergo a round of voting. This ultimate round of voting involves no monetary commitment. The vote is recorded by name. Each senator must write down on a piece of paper, his name,the side he votes for, and then all the votes are counted and revealed public. Votes with no names are invalid. Senators whose vote are not recorded are considered to have abstained. Senators may vote in favour, in opposition, or abstain or remain undecided. The threshold for winning is 1 + half of the total number of debaters. This means in a chamber of 8 senators, for Proposition or Opposition to win, it must secure 5 votes. The Proposition wins if and only it has secured 5 votes, in which case we say the motion is passed, or adopted. The Opposition wins if and only if it has secured 5 votes as well, in which case we say the motion is defeated, or rejected. In the case where the vote is split, perhaps evenly 4-4, or 4-3-1 in Y-N-U, then we say the motion is unresolved, in which case we do not say the motion has passed or defeated. In such a case we also say the chamber is tied.

    \item \textbf{Points and Jackpot} The general philosophy is that points is what counts towards breaks - senators with the highest points break into the elimination rounds. Jackpot is monetary reward, and fuel. They do not count in any way towards breaks. 
    \begin{itemize}
        \item The points are distributed thus, to every senator:
        \begin{table}[h]
            \centering
            \begin{tabular}{l|ccc}
            \toprule
             & stay 留 & switch 轉軚 & abstain 棄權 \\
            \midrule
            win 贏 & 3 & 1 & 0 \\
            lose 輸 & 2 & 0 & 0 \\
            tie 和 & 0 & 0 & 0 \\
            \bottomrule
            \end{tabular}
        \end{table}
        \item Senators who were undecided at the start of the debate and voted in favour or against the motion are considered to have switched sides, and are rewarded 1 point.
        \item You can see that those who lost the debate but stuck to their guns are rewarded more highly in terms of points than those who switched sides. 
        \item You can see that ties are very punishing to all senators involved. It basically means they have all wasted time and everyone has lost. 
    \end{itemize}

    \item \textbf{Jackpot money distribution.} The jackpot money is split thus:
    \begin{itemize}
        \item $s$ is the total sum in the pot, which always abides by the following: $s = w_{\text{留}} + w_{\text{轉}}$
        \item $v$ is the total number of votes on the winning side, which always abides by the following: $v = v_{\text{留}} + v_{\text{轉}}$
        \item $w_{\text{留}}$ and $w_{\text{轉}}$ are defined as such:
        \begin{enumerate}
            \item $w_{\text{留}} = s \cdot \frac{v_{\text{留}}+1}{v+1}$
            \item $w_{\text{轉}} = s \cdot \frac{v_{\text{轉}}}{v+1}$
        \end{enumerate}
        \item To say this without justification, this creates a quasi-prisoner's dilemma situation in the following box, where those who switch sides and win, win money but lose out relatively on points.
        \item The split of the $w_{\text{留}}$ and $w_{\text{轉}}$ leaves a weird case where everyone switched sides and that side won. In such a case, there'd be nobody to claim the virtual 留贏 share. That share can go to many places: (1) the poorest member, (2) the hosting tournament, (3) divided equally amongst all players\ldots (4) taken by the winning side as well\ldots
        \item The point of the dynamic is: some people care about the money, some people care about passing the motion.

    \end{itemize}

    \item \textbf{The rewards table} Combining the points and jackpot money distribution, the rewards matrix for each senator is:
    \begin{table}[h]
        \centering
        \begin{tabular}{l|ccc}
        \toprule
         & stay 留 & switch 轉軚 & abstain 棄權 \\
        \midrule
        win 贏 & (3,$\frac{s(v_{\text{留}}+1)}{v_{\text{留}}(v+1)}$)  & (1,$\frac{s v_{\text{轉}}}{v_{\text{轉}}(v+1)}$) & (0,0) \\
        lose 輸 & (2,0) & (0,0) & (0,0) \\
        tie 和 & (0,0) & (0,0) & (0,0) \\
        \bottomrule
        \end{tabular}
    \end{table}

    As we can see, the switching side has monetary pay out $\frac{s v_{\text{轉}}}{v_{\text{轉}}(v+1)} = \frac{s}{v+1}$.

    % \item \textbf{Nonstandard number of participants.} 
    % Suppose for some reason, or out of some other consideration, you have 8 senators, but you also want to introduce a number of $n$ participants into a round, without necessarily giving them full senatorial rights. Perhaps you want to increase the complexity or the anchorage of opinions in a round. Or perhaps you want to involve the entire audience in resolving the debate. 

    % These participants may be similar to the backbenchers. 

    \item \textbf{Chairs \& Speakers} Suppose for the sake of decorum, and perhaps for the sake of maintaining accounting validity, you might want to introduce a chair, or a speaker. The chair's role, unlike in British Parliamentary debate, does not involve judging. He is only involved in making sure the outcome properly computed and resolved. 

\end{enumerate}

\begin{figure}[H] % or just [htbp] if not using the float package
    \centering
    \begin{tikzpicture}[node distance=1.5cm]
    
        \node (start) [startstop] {START};
        
        \node (entry) [process, below of=start] {1. Entry \& Coin Declaration};
        \node (chair) [process, below of=entry] {2. Chair Records Sides \& Forms Jackpot};
        \node (order) [process, below of=chair] {3. Speaker Order Determination};
        \node (speak) [process, below of=order] {4. Speaking Phase (7 mins)};
        \node (vote) [process, below of=speak] {5. Final Voting Phase};
        \node (result) [decision, below of=vote, yshift=-0.2cm] {6. Vote Tally};
        \node (points) [process, below of=result, yshift=-1.0cm] {7. Points Assignment};
        \node (jackpot) [process, below of=points] {8. Jackpot Distribution};
        
        \node (end) [startstop, below of=jackpot] {END OF ROUND};
        
        % Arrows
        \draw [arrow] (start) -- (entry);
        \draw [arrow] (entry) -- (chair);
        \draw [arrow] (chair) -- (order);
        \draw [arrow] (order) -- (speak);
        \draw [arrow] (speak) -- (vote);
        \draw [arrow] (vote) -- (result);
        \draw [arrow] (result) -- (points);
        \draw [arrow] (points) -- (jackpot);
        \draw [arrow] (jackpot) -- (end);
    
    \end{tikzpicture}
    \caption{Flowchart of a round of the Roman Senatorial Debate}
    \label{fig:flowchart}
    \end{figure}
            

    



\section{Payout structure by debater count}
Recall that the number of senators can arguably be scaled up to any number. And also recall that the number of votes necessary to win is 1 + half of the total number of debaters.

Here we present the payout structure for different numbers of debaters and voting scenarios.


\begin{table}[H]
    \centering
    \small
    \begin{threeparttable}
    \begin{tabular}{cccccccc}    
    \toprule
    $n_{\text{debaters}}$ & $v_{\text{win}}$ & $v_{\text{留}}$ & $v_{\text{轉}}$ & $w_{\text{留}} = 100 \cdot \frac{v_{\text{留}}+1}{v_{win}+1} $ & $w_{\text{轉}} = 100\cdot\frac{v_{\text{轉}}}{v+1}$ & $\text{w}_{\text{留}/\text{人} }$ & $\text{w}_{\text{轉}/\text{人}}$ \\
    \midrule

% \midrule
6 & 4 & 0 & 4 & 20.0 & 80.0 & n/a\tnote{a} & 20.0 \\
6 & 4 & 1 & 3 & 40.0 & 60.0 & 40.0 & 20.0 \\
6 & 4 & 2 & 2 & 60.0 & 40.0 & 30.0 & 20.0 \\
6 & 4 & 3 & 1 & 80.0 & 20.0 & 26.67 & 20.0 \\
6 & 4 & 4 & 0 & 100.0 & 0.0 & 25.0 & 0.0 \\
6 & 5 & 0 & 5 & 16.67 & 83.33 & n/a\tnote{a} & 16.67 \\
6 & 5 & 1 & 4 & 33.33 & 66.67 & 33.33 & 16.67 \\
6 & 5 & 2 & 3 & 50.0 & 50.0 & 25.0 & 16.67 \\
6 & 5 & 3 & 2 & 66.67 & 33.33 & 22.22 & 16.67 \\
6 & 5 & 4 & 1 & 83.33 & 16.67 & 20.83 & 16.67 \\
6 & 5 & 5 & 0 & 100.0 & 0.0 & 20.0 & 0.0 \\
6 & 6 & 0 & 6 & 14.29 & 85.71 & n/a\tnote{a} & 14.29 \\
6 & 6 & 1 & 5 & 28.57 & 71.43 & 28.57 & 14.29 \\
6 & 6 & 2 & 4 & 42.86 & 57.14 & 21.43 & 14.29 \\
6 & 6 & 3 & 3 & 57.14 & 42.86 & 19.05 & 14.29 \\
6 & 6 & 4 & 2 & 71.43 & 28.57 & 17.86 & 14.29 \\
6 & 6 & 5 & 1 & 85.71 & 14.29 & 17.14 & 14.29 \\
6 & 6 & 6 & 0 & 100.0 & 0.0 & 16.67 & 0.0 \\
\bottomrule
\end{tabular}
\begin{tablenotes}
    \item[a] n/a, because there are literally no participants who won who stayed on the position that they started with. 
    Obviously this is a degenerate and highly unlikely case. But we still need to figure out how to handle the payout which has no one to receive it. It can either (1) be distributed amongst those who won, (2) go to anyone on the losing side who stuck to their guns, or (3) go to the poorest member, or (4) to the tournament organizer. We are not too interested in the details. 
\end{tablenotes}
\end{threeparttable}

\caption{Payout structure for different numbers of debaters and voting scenarios}
\end{table}

\begin{table}[H]
    \centering
    \small
    \begin{threeparttable}
    \begin{tabular}{cccccccc}
    \toprule
    $n_{\text{debaters}}$ & $v_{\text{win}}$ & $v_{\text{留}}$ & $v_{\text{轉}}$ & $w_{\text{留}} = 100 \cdot \frac{v_{\text{留}}+1}{v_{win}+1} $ & $w_{\text{轉}} = 100\cdot\frac{v_{\text{轉}}}{v+1}$ & $\text{w}_{\text{留}/\text{人} }$ & $\text{w}_{\text{轉}/\text{人}}$ \\
    \midrule
7 & 4 & 0 & 4 & 20.0 & 80.0 & n/a\tnote{a} & 20.0 \\
7 & 4 & 1 & 3 & 40.0 & 60.0 & 40.0 & 20.0 \\
7 & 4 & 2 & 2 & 60.0 & 40.0 & 30.0 & 20.0 \\
7 & 4 & 3 & 1 & 80.0 & 20.0 & 26.67 & 20.0 \\
7 & 4 & 4 & 0 & 100.0 & 0.0 & 25.0 & 0.0 \\
7 & 5 & 0 & 5 & 16.67 & 83.33 & n/a\tnote{a} & 16.67 \\
7 & 5 & 1 & 4 & 33.33 & 66.67 & 33.33 & 16.67 \\
7 & 5 & 2 & 3 & 50.0 & 50.0 & 25.0 & 16.67 \\
7 & 5 & 3 & 2 & 66.67 & 33.33 & 22.22 & 16.67 \\
7 & 5 & 4 & 1 & 83.33 & 16.67 & 20.83 & 16.67 \\
7 & 5 & 5 & 0 & 100.0 & 0.0 & 20.0 & 0.0 \\
7 & 6 & 0 & 6 & 14.29 & 85.71 & n/a\tnote{a} & 14.29 \\
7 & 6 & 1 & 5 & 28.57 & 71.43 & 28.57 & 14.29 \\
7 & 6 & 2 & 4 & 42.86 & 57.14 & 21.43 & 14.29 \\
7 & 6 & 3 & 3 & 57.14 & 42.86 & 19.05 & 14.29 \\
7 & 6 & 4 & 2 & 71.43 & 28.57 & 17.86 & 14.29 \\
7 & 6 & 5 & 1 & 85.71 & 14.29 & 17.14 & 14.29 \\
7 & 6 & 6 & 0 & 100.0 & 0.0 & 16.67 & 0.0 \\
7 & 7 & 0 & 7 & 12.5 & 87.5 & n/a\tnote{a} & 12.5 \\
7 & 7 & 1 & 6 & 25.0 & 75.0 & 25.0 & 12.5 \\
7 & 7 & 2 & 5 & 37.5 & 62.5 & 18.75 & 12.5 \\
7 & 7 & 3 & 4 & 50.0 & 50.0 & 16.67 & 12.5 \\
7 & 7 & 4 & 3 & 62.5 & 37.5 & 15.62 & 12.5 \\
7 & 7 & 5 & 2 & 75.0 & 25.0 & 15.0 & 12.5 \\
7 & 7 & 6 & 1 & 87.5 & 12.5 & 14.58 & 12.5 \\
7 & 7 & 7 & 0 & 100.0 & 0.0 & 14.29 & 0.0 \\
\bottomrule
\end{tabular}
\begin{tablenotes}
    \item[a] n/a indicates no 留 voters to receive the payout.
\end{tablenotes}
\caption{Payout structure for different numbers of debaters and voting scenarios}
\end{threeparttable}
\end{table}


\begin{table}[H]
    \centering
    \small
    \begin{threeparttable}
    \begin{tabular}{cccccccc}
    \toprule
    $n_{\text{debaters}}$ & $v_{\text{win}}$ & $v_{\text{留}}$ & $v_{\text{轉}}$ & $w_{\text{留}} = 100 \cdot \frac{v_{\text{留}}+1}{v_{win}+1} $ & $w_{\text{轉}} = 100\cdot\frac{v_{\text{轉}}}{v+1}$ & $\text{w}_{\text{留}/\text{人} }$ & $\text{w}_{\text{轉}/\text{人}}$ \\
    \midrule
8 & 5 & 0 & 5 & 16.67 & 83.33 & n/a\tnote{a} & 16.67 \\
8 & 5 & 1 & 4 & 33.33 & 66.67 & 33.33 & 16.67 \\
8 & 5 & 2 & 3 & 50.0 & 50.0 & 25.0 & 16.67 \\
8 & 5 & 3 & 2 & 66.67 & 33.33 & 22.22 & 16.67 \\
8 & 5 & 4 & 1 & 83.33 & 16.67 & 20.83 & 16.67 \\
8 & 5 & 5 & 0 & 100.0 & 0.0 & 20.0 & 0.0 \\
8 & 6 & 0 & 6 & 14.29 & 85.71 & n/a\tnote{a} & 14.29 \\
8 & 6 & 1 & 5 & 28.57 & 71.43 & 28.57 & 14.29 \\
8 & 6 & 2 & 4 & 42.86 & 57.14 & 21.43 & 14.29 \\
8 & 6 & 3 & 3 & 57.14 & 42.86 & 19.05 & 14.29 \\
8 & 6 & 4 & 2 & 71.43 & 28.57 & 17.86 & 14.29 \\
8 & 6 & 5 & 1 & 85.71 & 14.29 & 17.14 & 14.29 \\
8 & 6 & 6 & 0 & 100.0 & 0.0 & 16.67 & 0.0 \\
8 & 7 & 0 & 7 & 12.5 & 87.5 & n/a\tnote{a} & 12.5 \\
8 & 7 & 1 & 6 & 25.0 & 75.0 & 25.0 & 12.5 \\
8 & 7 & 2 & 5 & 37.5 & 62.5 & 18.75 & 12.5 \\
8 & 7 & 3 & 4 & 50.0 & 50.0 & 16.67 & 12.5 \\
8 & 7 & 4 & 3 & 62.5 & 37.5 & 15.62 & 12.5 \\
8 & 7 & 5 & 2 & 75.0 & 25.0 & 15.0 & 12.5 \\
8 & 7 & 6 & 1 & 87.5 & 12.5 & 14.58 & 12.5 \\
8 & 7 & 7 & 0 & 100.0 & 0.0 & 14.29 & 0.0 \\
8 & 8 & 0 & 8 & 11.11 & 88.89 & n/a\tnote{a} & 11.11 \\
8 & 8 & 1 & 7 & 22.22 & 77.78 & 22.22 & 11.11 \\
8 & 8 & 2 & 6 & 33.33 & 66.67 & 16.67 & 11.11 \\
8 & 8 & 3 & 5 & 44.44 & 55.56 & 14.81 & 11.11 \\
8 & 8 & 4 & 4 & 55.56 & 44.44 & 13.89 & 11.11 \\
8 & 8 & 5 & 3 & 66.67 & 33.33 & 13.33 & 11.11 \\
8 & 8 & 6 & 2 & 77.78 & 22.22 & 12.96 & 11.11 \\
8 & 8 & 7 & 1 & 88.89 & 11.11 & 12.7 & 11.11 \\
8 & 8 & 8 & 0 & 100.0 & 0.0 & 12.5 & 0.0 \\
\bottomrule
\end{tabular}
\begin{tablenotes}
    \item[a] n/a indicates no 留 voters to receive the payout.
\end{tablenotes}
\caption{Payout structure for different numbers of debaters and voting scenarios}
\end{threeparttable}
\end{table}

\begin{table}[H]
    \centering
    \small
    \begin{threeparttable}
    \begin{tabular}{cccccccc}
    \toprule
    $n_{\text{debaters}}$ & $v_{\text{win}}$ & $v_{\text{留}}$ & $v_{\text{轉}}$ & $w_{\text{留}} = 100 \cdot \frac{v_{\text{留}}+1}{v_{win}+1} $ & $w_{\text{轉}} = 100\cdot\frac{v_{\text{轉}}}{v+1}$ & $\text{w}_{\text{留}/\text{人} }$ & $\text{w}_{\text{轉}/\text{人}}$ \\
    \midrule
9 & 5 & 0 & 5 & 16.67 & 83.33 & n/a\tnote{a} & 16.67 \\
9 & 5 & 1 & 4 & 33.33 & 66.67 & 33.33 & 16.67 \\
9 & 5 & 2 & 3 & 50.0 & 50.0 & 25.0 & 16.67 \\
9 & 5 & 3 & 2 & 66.67 & 33.33 & 22.22 & 16.67 \\
9 & 5 & 4 & 1 & 83.33 & 16.67 & 20.83 & 16.67 \\
9 & 5 & 5 & 0 & 100.0 & 0.0 & 20.0 & 0.0 \\
9 & 6 & 0 & 6 & 14.29 & 85.71 & n/a\tnote{a} & 14.29 \\
9 & 6 & 1 & 5 & 28.57 & 71.43 & 28.57 & 14.29 \\
9 & 6 & 2 & 4 & 42.86 & 57.14 & 21.43 & 14.29 \\
9 & 6 & 3 & 3 & 57.14 & 42.86 & 19.05 & 14.29 \\
9 & 6 & 4 & 2 & 71.43 & 28.57 & 17.86 & 14.29 \\
9 & 6 & 5 & 1 & 85.71 & 14.29 & 17.14 & 14.29 \\
9 & 6 & 6 & 0 & 100.0 & 0.0 & 16.67 & 0.0 \\
9 & 7 & 0 & 7 & 12.5 & 87.5 & n/a\tnote{a} & 12.5 \\
9 & 7 & 1 & 6 & 25.0 & 75.0 & 25.0 & 12.5 \\
9 & 7 & 2 & 5 & 37.5 & 62.5 & 18.75 & 12.5 \\
9 & 7 & 3 & 4 & 50.0 & 50.0 & 16.67 & 12.5 \\
9 & 7 & 4 & 3 & 62.5 & 37.5 & 15.62 & 12.5 \\
9 & 7 & 5 & 2 & 75.0 & 25.0 & 15.0 & 12.5 \\
9 & 7 & 6 & 1 & 87.5 & 12.5 & 14.58 & 12.5 \\
9 & 7 & 7 & 0 & 100.0 & 0.0 & 14.29 & 0.0 \\
9 & 8 & 0 & 8 & 11.11 & 88.89 & n/a\tnote{a} & 11.11 \\
9 & 8 & 1 & 7 & 22.22 & 77.78 & 22.22 & 11.11 \\
9 & 8 & 2 & 6 & 33.33 & 66.67 & 16.67 & 11.11 \\
9 & 8 & 3 & 5 & 44.44 & 55.56 & 14.81 & 11.11 \\
9 & 8 & 4 & 4 & 55.56 & 44.44 & 13.89 & 11.11 \\
9 & 8 & 5 & 3 & 66.67 & 33.33 & 13.33 & 11.11 \\
9 & 8 & 6 & 2 & 77.78 & 22.22 & 12.96 & 11.11 \\
9 & 8 & 7 & 1 & 88.89 & 11.11 & 12.7 & 11.11 \\
9 & 8 & 8 & 0 & 100.0 & 0.0 & 12.5 & 0.0 \\
9 & 9 & 0 & 9 & 10.0 & 90.0 & n/a\tnote{a} & 10.0 \\
9 & 9 & 1 & 8 & 20.0 & 80.0 & 20.0 & 10.0 \\
9 & 9 & 2 & 7 & 30.0 & 70.0 & 15.0 & 10.0 \\
9 & 9 & 3 & 6 & 40.0 & 60.0 & 13.33 & 10.0 \\
9 & 9 & 4 & 5 & 50.0 & 50.0 & 12.5 & 10.0 \\
9 & 9 & 5 & 4 & 60.0 & 40.0 & 12.0 & 10.0 \\
9 & 9 & 6 & 3 & 70.0 & 30.0 & 11.67 & 10.0 \\
9 & 9 & 7 & 2 & 80.0 & 20.0 & 11.43 & 10.0 \\
9 & 9 & 8 & 1 & 90.0 & 10.0 & 11.25 & 10.0 \\
9 & 9 & 9 & 0 & 100.0 & 0.0 & 11.11 & 0.0 \\
\bottomrule
\end{tabular}
\begin{tablenotes}
    \item[a] n/a indicates no 留 voters to receive the payout.
\end{tablenotes}
\caption{Payout structure for different numbers of debaters and voting scenarios}
\end{threeparttable}
\end{table}

\begin{table}[H]
    \centering
    \small
    \begin{threeparttable}
    \begin{tabular}{cccccccc}
    \toprule
    $n_{\text{debaters}}$ & $v_{\text{win}}$ & $v_{\text{留}}$ & $v_{\text{轉}}$ & $w_{\text{留}} = 100 \cdot \frac{v_{\text{留}}+1}{v_{win}+1} $ & $w_{\text{轉}} = 100\cdot\frac{v_{\text{轉}}}{v+1}$ & $\text{w}_{\text{留}/\text{人} }$ & $\text{w}_{\text{轉}/\text{人}}$ \\
    \midrule
10 & 6 & 0 & 6 & 14.29 & 85.71 & n/a\tnote{a} & 14.29 \\
10 & 6 & 1 & 5 & 28.57 & 71.43 & 28.57 & 14.29 \\
10 & 6 & 2 & 4 & 42.86 & 57.14 & 21.43 & 14.29 \\
10 & 6 & 3 & 3 & 57.14 & 42.86 & 19.05 & 14.29 \\
10 & 6 & 4 & 2 & 71.43 & 28.57 & 17.86 & 14.29 \\
10 & 6 & 5 & 1 & 85.71 & 14.29 & 17.14 & 14.29 \\
10 & 6 & 6 & 0 & 100.0 & 0.0 & 16.67 & 0.0 \\
10 & 7 & 0 & 7 & 12.5 & 87.5 & n/a\tnote{a} & 12.5 \\
10 & 7 & 1 & 6 & 25.0 & 75.0 & 25.0 & 12.5 \\
10 & 7 & 2 & 5 & 37.5 & 62.5 & 18.75 & 12.5 \\
10 & 7 & 3 & 4 & 50.0 & 50.0 & 16.67 & 12.5 \\
10 & 7 & 4 & 3 & 62.5 & 37.5 & 15.62 & 12.5 \\
10 & 7 & 5 & 2 & 75.0 & 25.0 & 15.0 & 12.5 \\
10 & 7 & 6 & 1 & 87.5 & 12.5 & 14.58 & 12.5 \\
10 & 7 & 7 & 0 & 100.0 & 0.0 & 14.29 & 0.0 \\
10 & 8 & 0 & 8 & 11.11 & 88.89 & n/a\tnote{a} & 11.11 \\
10 & 8 & 1 & 7 & 22.22 & 77.78 & 22.22 & 11.11 \\
10 & 8 & 2 & 6 & 33.33 & 66.67 & 16.67 & 11.11 \\
10 & 8 & 3 & 5 & 44.44 & 55.56 & 14.81 & 11.11 \\
10 & 8 & 4 & 4 & 55.56 & 44.44 & 13.89 & 11.11 \\
10 & 8 & 5 & 3 & 66.67 & 33.33 & 13.33 & 11.11 \\
10 & 8 & 6 & 2 & 77.78 & 22.22 & 12.96 & 11.11 \\
10 & 8 & 7 & 1 & 88.89 & 11.11 & 12.7 & 11.11 \\
10 & 8 & 8 & 0 & 100.0 & 0.0 & 12.5 & 0.0 \\
10 & 9 & 0 & 9 & 10.0 & 90.0 & n/a\tnote{a} & 10.0 \\
10 & 9 & 1 & 8 & 20.0 & 80.0 & 20.0 & 10.0 \\
10 & 9 & 2 & 7 & 30.0 & 70.0 & 15.0 & 10.0 \\
10 & 9 & 3 & 6 & 40.0 & 60.0 & 13.33 & 10.0 \\
10 & 9 & 4 & 5 & 50.0 & 50.0 & 12.5 & 10.0 \\
10 & 9 & 5 & 4 & 60.0 & 40.0 & 12.0 & 10.0 \\
10 & 9 & 6 & 3 & 70.0 & 30.0 & 11.67 & 10.0 \\
10 & 9 & 7 & 2 & 80.0 & 20.0 & 11.43 & 10.0 \\
10 & 9 & 8 & 1 & 90.0 & 10.0 & 11.25 & 10.0 \\
10 & 9 & 9 & 0 & 100.0 & 0.0 & 11.11 & 0.0 \\
10 & 10 & 0 & 10 & 9.09 & 90.91 & n/a\tnote{a} & 9.09 \\
10 & 10 & 1 & 9 & 18.18 & 81.82 & 18.18 & 9.09 \\
10 & 10 & 2 & 8 & 27.27 & 72.73 & 13.64 & 9.09 \\
10 & 10 & 3 & 7 & 36.36 & 63.64 & 12.12 & 9.09 \\
10 & 10 & 4 & 6 & 45.45 & 54.55 & 11.36 & 9.09 \\
10 & 10 & 5 & 5 & 54.55 & 45.45 & 10.91 & 9.09 \\
10 & 10 & 6 & 4 & 63.64 & 36.36 & 10.61 & 9.09 \\
10 & 10 & 7 & 3 & 72.73 & 27.27 & 10.39 & 9.09 \\
10 & 10 & 8 & 2 & 81.82 & 18.18 & 10.23 & 9.09 \\
10 & 10 & 9 & 1 & 90.91 & 9.09 & 10.1 & 9.09 \\
10 & 10 & 10 & 0 & 100.0 & 0.0 & 10.0 & 0.0 \\
\bottomrule
\end{tabular}
\begin{tablenotes}
    \item[a] n/a indicates no 留 voters to receive the payout.
\end{tablenotes}
\caption{Payout structure for different numbers of debaters and voting scenarios}
\end{threeparttable}
\end{table}



\begin{table}[H]
    \centering
    \small
    \begin{threeparttable}
    \begin{tabular}{cccccccc}
    \toprule    \toprule
    $n_{\text{debaters}}$ & $v_{\text{win}}$ & $v_{\text{留}}$ & $v_{\text{轉}}$ & $w_{\text{留}} = 100 \cdot \frac{v_{\text{留}}+1}{v_{win}+1} $ & $w_{\text{轉}} = 100\cdot\frac{v_{\text{轉}}}{v+1}$ & $\text{w}_{\text{留}/\text{人} }$ & $\text{w}_{\text{轉}/\text{人}}$ \\
    \midrule
11 & 6 & 0 & 6 & 14.29 & 85.71 & n/a\tnote{a} & 14.29 \\
11 & 6 & 1 & 5 & 28.57 & 71.43 & 28.57 & 14.29 \\
11 & 6 & 2 & 4 & 42.86 & 57.14 & 21.43 & 14.29 \\
11 & 6 & 3 & 3 & 57.14 & 42.86 & 19.05 & 14.29 \\
11 & 6 & 4 & 2 & 71.43 & 28.57 & 17.86 & 14.29 \\
11 & 6 & 5 & 1 & 85.71 & 14.29 & 17.14 & 14.29 \\
11 & 6 & 6 & 0 & 100.0 & 0.0 & 16.67 & 0.0 \\
11 & 7 & 0 & 7 & 12.5 & 87.5 & n/a\tnote{a} & 12.5 \\
11 & 7 & 1 & 6 & 25.0 & 75.0 & 25.0 & 12.5 \\
11 & 7 & 2 & 5 & 37.5 & 62.5 & 18.75 & 12.5 \\
11 & 7 & 3 & 4 & 50.0 & 50.0 & 16.67 & 12.5 \\
11 & 7 & 4 & 3 & 62.5 & 37.5 & 15.62 & 12.5 \\
11 & 7 & 5 & 2 & 75.0 & 25.0 & 15.0 & 12.5 \\
11 & 7 & 6 & 1 & 87.5 & 12.5 & 14.58 & 12.5 \\
11 & 7 & 7 & 0 & 100.0 & 0.0 & 14.29 & 0.0 \\
11 & 8 & 0 & 8 & 11.11 & 88.89 & n/a\tnote{a} & 11.11 \\
11 & 8 & 1 & 7 & 22.22 & 77.78 & 22.22 & 11.11 \\
11 & 8 & 2 & 6 & 33.33 & 66.67 & 16.67 & 11.11 \\
11 & 8 & 3 & 5 & 44.44 & 55.56 & 14.81 & 11.11 \\
11 & 8 & 4 & 4 & 55.56 & 44.44 & 13.89 & 11.11 \\
11 & 8 & 5 & 3 & 66.67 & 33.33 & 13.33 & 11.11 \\
11 & 8 & 6 & 2 & 77.78 & 22.22 & 12.96 & 11.11 \\
11 & 8 & 7 & 1 & 88.89 & 11.11 & 12.7 & 11.11 \\
11 & 8 & 8 & 0 & 100.0 & 0.0 & 12.5 & 0.0 \\
11 & 9 & 0 & 9 & 10.0 & 90.0 & n/a\tnote{a} & 10.0 \\
11 & 9 & 1 & 8 & 20.0 & 80.0 & 20.0 & 10.0 \\
11 & 9 & 2 & 7 & 30.0 & 70.0 & 15.0 & 10.0 \\
11 & 9 & 3 & 6 & 40.0 & 60.0 & 13.33 & 10.0 \\
11 & 9 & 4 & 5 & 50.0 & 50.0 & 12.5 & 10.0 \\
11 & 9 & 5 & 4 & 60.0 & 40.0 & 12.0 & 10.0 \\
11 & 9 & 6 & 3 & 70.0 & 30.0 & 11.67 & 10.0 \\
11 & 9 & 7 & 2 & 80.0 & 20.0 & 11.43 & 10.0 \\
11 & 9 & 8 & 1 & 90.0 & 10.0 & 11.25 & 10.0 \\
11 & 9 & 9 & 0 & 100.0 & 0.0 & 11.11 & 0.0 \\
11 & 10 & 0 & 10 & 9.09 & 90.91 & n/a\tnote{a} & 9.09 \\
11 & 10 & 1 & 9 & 18.18 & 81.82 & 18.18 & 9.09 \\
11 & 10 & 2 & 8 & 27.27 & 72.73 & 13.64 & 9.09 \\
11 & 10 & 3 & 7 & 36.36 & 63.64 & 12.12 & 9.09 \\
11 & 10 & 4 & 6 & 45.45 & 54.55 & 11.36 & 9.09 \\
11 & 10 & 5 & 5 & 54.55 & 45.45 & 10.91 & 9.09 \\
11 & 10 & 6 & 4 & 63.64 & 36.36 & 10.61 & 9.09 \\
11 & 10 & 7 & 3 & 72.73 & 27.27 & 10.39 & 9.09 \\
11 & 10 & 8 & 2 & 81.82 & 18.18 & 10.23 & 9.09 \\
11 & 10 & 9 & 1 & 90.91 & 9.09 & 10.1 & 9.09 \\
11 & 10 & 10 & 0 & 100.0 & 0.0 & 10.0 & 0.0 \\
11 & 11 & 0 & 11 & 8.33 & 91.67 & n/a\tnote{a} & 8.33 \\
11 & 11 & 1 & 10 & 16.67 & 83.33 & 16.67 & 8.33 \\
11 & 11 & 2 & 9 & 25.0 & 75.0 & 12.5 & 8.33 \\
11 & 11 & 3 & 8 & 33.33 & 66.67 & 11.11 & 8.33 \\
11 & 11 & 4 & 7 & 41.67 & 58.33 & 10.42 & 8.33 \\
11 & 11 & 5 & 6 & 50.0 & 50.0 & 10.0 & 8.33 \\
11 & 11 & 6 & 5 & 58.33 & 41.67 & 9.72 & 8.33 \\
11 & 11 & 7 & 4 & 66.67 & 33.33 & 9.52 & 8.33 \\
11 & 11 & 8 & 3 & 75.0 & 25.0 & 9.38 & 8.33 \\
11 & 11 & 9 & 2 & 83.33 & 16.67 & 9.26 & 8.33 \\
11 & 11 & 10 & 1 & 91.67 & 8.33 & 9.17 & 8.33 \\
11 & 11 & 11 & 0 & 100.0 & 0.0 & 9.09 & 0.0 \\

\bottomrule
\end{tabular}
\begin{tablenotes}
    \item[a] n/a indicates no 留 voters to receive the payout.
\end{tablenotes}
\caption{Payout structure for different numbers of debaters and voting scenarios}
\end{threeparttable}
\end{table}




\begin{table}[H]
    \centering
    \small
    \begin{threeparttable}
    \begin{tabular}{cccccccc}
    \toprule    \toprule
    $n_{\text{debaters}}$ & $v_{\text{win}}$ & $v_{\text{留}}$ & $v_{\text{轉}}$ & $w_{\text{留}} = 100 \cdot \frac{v_{\text{留}}+1}{v_{win}+1} $ & $w_{\text{轉}} = 100\cdot\frac{v_{\text{轉}}}{v+1}$ & $\text{w}_{\text{留}/\text{人} }$ & $\text{w}_{\text{轉}/\text{人}}$ \\
    \midrule
12 & 7 & 0 & 7 & 12.5 & 87.5 & n/a\tnote{a} & 12.5 \\
12 & 7 & 1 & 6 & 25.0 & 75.0 & 25.0 & 12.5 \\
12 & 7 & 2 & 5 & 37.5 & 62.5 & 18.75 & 12.5 \\
12 & 7 & 3 & 4 & 50.0 & 50.0 & 16.67 & 12.5 \\
12 & 7 & 4 & 3 & 62.5 & 37.5 & 15.62 & 12.5 \\
12 & 7 & 5 & 2 & 75.0 & 25.0 & 15.0 & 12.5 \\
12 & 7 & 6 & 1 & 87.5 & 12.5 & 14.58 & 12.5 \\
12 & 7 & 7 & 0 & 100.0 & 0.0 & 14.29 & 0.0 \\
12 & 8 & 0 & 8 & 11.11 & 88.89 & n/a\tnote{a} & 11.11 \\
12 & 8 & 1 & 7 & 22.22 & 77.78 & 22.22 & 11.11 \\
12 & 8 & 2 & 6 & 33.33 & 66.67 & 16.67 & 11.11 \\
12 & 8 & 3 & 5 & 44.44 & 55.56 & 14.81 & 11.11 \\
12 & 8 & 4 & 4 & 55.56 & 44.44 & 13.89 & 11.11 \\
12 & 8 & 5 & 3 & 66.67 & 33.33 & 13.33 & 11.11 \\
12 & 8 & 6 & 2 & 77.78 & 22.22 & 12.96 & 11.11 \\
12 & 8 & 7 & 1 & 88.89 & 11.11 & 12.7 & 11.11 \\
12 & 8 & 8 & 0 & 100.0 & 0.0 & 12.5 & 0.0 \\
12 & 9 & 0 & 9 & 10.0 & 90.0 & n/a\tnote{a} & 10.0 \\
12 & 9 & 1 & 8 & 20.0 & 80.0 & 20.0 & 10.0 \\
12 & 9 & 2 & 7 & 30.0 & 70.0 & 15.0 & 10.0 \\
12 & 9 & 3 & 6 & 40.0 & 60.0 & 13.33 & 10.0 \\
12 & 9 & 4 & 5 & 50.0 & 50.0 & 12.5 & 10.0 \\
12 & 9 & 5 & 4 & 60.0 & 40.0 & 12.0 & 10.0 \\
12 & 9 & 6 & 3 & 70.0 & 30.0 & 11.67 & 10.0 \\
12 & 9 & 7 & 2 & 80.0 & 20.0 & 11.43 & 10.0 \\
12 & 9 & 8 & 1 & 90.0 & 10.0 & 11.25 & 10.0 \\
12 & 9 & 9 & 0 & 100.0 & 0.0 & 11.11 & 0.0 \\
12 & 10 & 0 & 10 & 9.09 & 90.91 & n/a\tnote{a} & 9.09 \\
12 & 10 & 1 & 9 & 18.18 & 81.82 & 18.18 & 9.09 \\
12 & 10 & 2 & 8 & 27.27 & 72.73 & 13.64 & 9.09 \\
12 & 10 & 3 & 7 & 36.36 & 63.64 & 12.12 & 9.09 \\
12 & 10 & 4 & 6 & 45.45 & 54.55 & 11.36 & 9.09 \\
12 & 10 & 5 & 5 & 54.55 & 45.45 & 10.91 & 9.09 \\
12 & 10 & 6 & 4 & 63.64 & 36.36 & 10.61 & 9.09 \\
12 & 10 & 7 & 3 & 72.73 & 27.27 & 10.39 & 9.09 \\
12 & 10 & 8 & 2 & 81.82 & 18.18 & 10.23 & 9.09 \\
12 & 10 & 9 & 1 & 90.91 & 9.09 & 10.1 & 9.09 \\
12 & 10 & 10 & 0 & 100.0 & 0.0 & 10.0 & 0.0 \\
12 & 11 & 0 & 11 & 8.33 & 91.67 & n/a\tnote{a} & 8.33 \\
12 & 11 & 1 & 10 & 16.67 & 83.33 & 16.67 & 8.33 \\
12 & 11 & 2 & 9 & 25.0 & 75.0 & 12.5 & 8.33 \\
12 & 11 & 3 & 8 & 33.33 & 66.67 & 11.11 & 8.33 \\
12 & 11 & 4 & 7 & 41.67 & 58.33 & 10.42 & 8.33 \\
12 & 11 & 5 & 6 & 50.0 & 50.0 & 10.0 & 8.33 \\
12 & 11 & 6 & 5 & 58.33 & 41.67 & 9.72 & 8.33 \\
12 & 11 & 7 & 4 & 66.67 & 33.33 & 9.52 & 8.33 \\
12 & 11 & 8 & 3 & 75.0 & 25.0 & 9.38 & 8.33 \\
12 & 11 & 9 & 2 & 83.33 & 16.67 & 9.26 & 8.33 \\
12 & 11 & 10 & 1 & 91.67 & 8.33 & 9.17 & 8.33 \\
12 & 11 & 11 & 0 & 100.0 & 0.0 & 9.09 & 0.0 \\
12 & 12 & 0 & 12 & 7.69 & 92.31 & n/a\tnote{a} & 7.69 \\
12 & 12 & 1 & 11 & 15.38 & 84.62 & 15.38 & 7.69 \\
12 & 12 & 2 & 10 & 23.08 & 76.92 & 11.54 & 7.69 \\
12 & 12 & 3 & 9 & 30.77 & 69.23 & 10.26 & 7.69 \\
12 & 12 & 4 & 8 & 38.46 & 61.54 & 9.62 & 7.69 \\
12 & 12 & 5 & 7 & 46.15 & 53.85 & 9.23 & 7.69 \\
12 & 12 & 6 & 6 & 53.85 & 46.15 & 8.97 & 7.69 \\
12 & 12 & 7 & 5 & 61.54 & 38.46 & 8.79 & 7.69 \\
12 & 12 & 8 & 4 & 69.23 & 30.77 & 8.65 & 7.69 \\
12 & 12 & 9 & 3 & 76.92 & 23.08 & 8.55 & 7.69 \\
12 & 12 & 10 & 2 & 84.62 & 15.38 & 8.46 & 7.69 \\
12 & 12 & 11 & 1 & 92.31 & 7.69 & 8.39 & 7.69 \\
12 & 12 & 12 & 0 & 100.0 & 0.0 & 8.33 & 0.0 \\

\bottomrule
\end{tabular}
\begin{tablenotes}
    \item[a] n/a indicates no 留 voters to receive the payout.
\end{tablenotes}
\caption{Payout structure for different numbers of debaters and voting scenarios}
\end{threeparttable}
\end{table}





\section{Elimination mechanism}

Unlike the British Parliamentary style tournament where you can set the number of rooms depending on how many outrounds you want, and then the teams are exponentially and deterministically eliminated, the Roman Senatorial Debate does not work like that. 

A chamber might fail to resolve the motion, in which case every senator gets 0 points. A chamber might also resolve the motion unanimously without switching sides, in which case every senator gets 3 points. These irregularities pose difficulties to a straightforward elimination mechanism like that in the British Parliamentary style tournament, as a chamber might return all 8 senators with 3 points, or all senators with 0 points. 











\begin{enumerate}
    \item \textbf{Admission to elimination rounds.} At breaks, teams are admitted into the elimination rounds by virtue of the points they've accumulated. 
    \item \textbf{Tie resolution.} In any case where there is a tie, this either resolved by money, or force. In the case where the option to resolve it by force is enabled, any tied member has the right to resolve it by trial by combat. Women may be allowed to fight, but they are accorded the privilege to find a champion. Men are not accorded the privlege of champerty. Champions must be participating senators. This is in line with the general philosophy of the tournament that debate, politics, and intellectualism makes sense only if it is a game of honour - and honour only emerges if there is an ultimate backstop of violence.
    \item \textbf{Points carryover.} Senators will continue to inherit their points from the preliminary rounds. Senators WILL NOT have their accumulated points reset to 0 in the breakrounds. The point of this is to give intellectual autonomy to the senators to stand their ground. 


    \item \textbf{Number of participants.} The number of senators the tournament admits into the outrounds is the number of rounds times 8 senators. we have: 
        \begin{enumerate}
            \item octos (8 strongest rooms, 4th last round): 64 人 $\rightarrow$
            \item quarters (4 strongest rooms, 3rd last round): 32 人 $\rightarrow$ 
            \item semis (2 strongest rooms, 2nd last round): 16 人 $\rightarrow$ 
            \item grand final (last round): 8 人.
        \end{enumerate}
    
    \item \textbf{Elimination mechanism.} In each of the outrounds, the aggregate half of the rooms with the lowest points are eliminated. So unlike BP where the result of each room individually decides the outcome of the outround, here the aggregate result of all rooms in the outround decides the outcome.
    
    
    \item \textbf{The Market of Speech and Influence.} Since senators carry over their points from the preliminary rounds, those with high scores can speak freely with their immediate fear of elimination heavily discounted. This grants intellectual autonomy to top-performing senators, allowing them to assert their own private views with confidence and demonstrate their full intellectual prowess. This means there is now a privilege to speak, where you can basically air your own views on a topic, and you granted an audience. This, on the whole, means that the tournament becomes a market where the commodity of speech and influence can be acquired by a combination of strategic political manoeuvre and careful money management.
    
    \item \textbf{The Grand Final.} How is the winner of the Grand Final determined? This is tricky and it depends on what matters to you. The recommendation is that the you still run the debate as a normal round, but depending on the tournament's interest, one may arbitrarily define the "winner(s)" to be whoever stuck to his guns and whose side resolved the motion. But if for reasons of ceremony or competition the tournament must produce a "winner", they might resolve to define the winner as the side that the chamber resolved in favour for. 
    
   

\end{enumerate}









\section{Emergent phenomenon}

It is very likely that this set of rules will yield emergent phenomena. It is my sincere hope that the game, along with its emergent phenomena, will beget and engender virtues in its particpants: honour, intellectualism, honesty, duty, and most importantly, the prudent and intelligent use of force. The pen that is dancing is only as mighty as the sword that is at rest.


\subsection{Bribery}

Consider the case where in the first round of votes, the vote distribution is 7 Y and 1 N. This is a highly stable state. The 7 participants have no incentive to switch their vote. Anybody switching would be giving up 3 points for at best 1 point, and the jackpot sum that's already in the bag. 

Then, what's the point of the debate here? There is no point. There would only be a point to debate if there's a destablizing force of some sort. 

If things have been left to their own devices, then the inevitable solution will emerge by itself - bribery, vote-buying.  

Vote-buying, from other senators, and only from other senators, serves as a reasonable destablizing force. 

If the form of vote-buying enabled involves the inflation of the supply of votes, then debate will become a simple matter of money takes all. This is not a good thing. Vote inflation must not be allowed. It destroys the need to debate, the need to persuade, and the need to offer a bribe. It eliminates politicking and favour-trading - which is the whole point of this exercise. But if vote-buying is restricted to the form of bribing other senators, then there will still be a tug of war between conscience and monetary self-interest. 


Since we are dealing with actual money, vote-buying is going to be particularly impactful on less well-off participants. This naturally means there would be a natural incentive for poor participants to join the tournament - for the promise of monetary prises. Just like how the poker game inevitably attracts the poor university student.


One might wonder, does one need to introduce a mechanism to enforce the promises of bribery? The answer is no. In fact, it absolutely should not be introduced. For if you do, then it eliminates the possibility of creating the culture of honour and promisekeeping. Furthermore, dishonest and lying bribe-offerers will likely be punished as the tournament progresses. Reiterated game theory dynamic secures it. If you cheated someone of a promised bribe, what are the chances you will maintain your standing in the tournament? Or in another tournament? 

On the other hand, perhaps there is someone you really want to screw over - either because of personal vendetta or because of the repugnance of their views. In these case, one might be highly motivated to engage in treachery.

The capacity for treachery is something we should cultivate—even if its victims find it distasteful. Participants must internalize this, not only because treachery is an unavoidable feature of life, especially in high politics, but also because those with lofty ideals and ambitions will find it a dagger they cannot do without. If our goal is to foster a culture of trust and honour, as well as the capacity to betray and wound—like the Roman senators of old—then we must give participants the space to exercise both. Most importantly, the point is that a man is good not because he is weak and is unable to inflict pain, but because he is strong and is able to do harm, but choses not to do so. Si vis pacem, para bellum.


It is therefore no overstatement to say that bribery is the mother of honour in this game. 


This begets an interesting question: why are bribes so frowned upon in modern electoral arena? 

The most promising answer is that not all bribes are identical. The follow-up is that favour-trading is the equivalent of bribes that we described and enabled in our debate game - and that is never disallowed. Indeed, favour-trading is what politics is all about. If disallowed, politics disintegrates and ceases to be. 


But that is indeed a rather simplistic answer. Let's pin down a particular form of vote-buying. It seems particularly intuitive why enabling the buying of votes in parliament is not a good idea - it should seem too powerful a mechanism to reorganize interests - it's like staring directly into the sun. 

But how about the case of vote-buying from the general voter? Why can't the ordinary voter sell his vote? 


Bribing parliamentarians enable embezzlement or cronyism - specifically the case of transferring funds to a service provider whose service is purchased by the government. The parliamentary receives a bribe, either directly, or in the form of a kickback.

The key mechanism is that the parliamentarian increased the price the government is willing to pay for the service than the government would have paid if there was no bribe. The difference is then split between the service-provider and the bribe paid to the parliamentarian.

Does this mechanism manifest if ordinary voters are bribed? One might argue it is less likely as the parliamentarian is still

So I think it's safe to say that the case where bribery is enabled for the general voter is more resistant to embezzlement than bribes for parliamentarians. \footnote{Hi}

\subsection{Nonchalant Speech}

\subsection{Militants}


% \subsection{Evolutionary Game Theoretic Dynamics}


\begin{itemize}
    \item \textbf{Honour}
    \item \textbf{Motivations}
\end{itemize}


Charm, multiround game theoretic dynamics, politics 


\subsection{Virtues bred}
virtues bred 







\section{Intellectual Capture}

The debate tournament's potential for intellectual capture is under-appreciated. If appreciated, I shan't be suprised to see interested parties funding their own debate tournaments for ideological purposes.

The problem with the British Parliamentary format is not that it is open to intellectual capture. Its malaise is that (1) in theory, it is entirely captured by the core adjudicators and the body of judge, and (2) in practice, it is entirely captured by youthful idealism, which is naturally prone to levitate from reality, for there is no mechanical temperance by reality. 

All forms of competitive or performative debate are open to intellectual and ideological capture. The Roman Senatorial Debate is no exception - but it is more prone to intellectual capture by ideologies informed by interest, and stakeholderhood. 


How can the Roman Senatorial Debate be used to capture the intellectuals?

- motion setting 
- money 
- 
\section{Recommended Motions}

\begin{enumerate}
    \item This House would introduce the 2nd amendment to the United Kingdom.
    \item This House would fight for King and Country. 
    \item This House believes that the Sovereign should fire the Prime Minister.
    \item This House believes that the Common Law should be imposed on all European countries.
    \item This House believes supernational states are a bad idea. 
    \item This House believes the three ideologies of the 20th centuries, communism, fascism, and liberalism, are all bad ideas.
    \item This House regrets the dissolution of the British Empire. 
    \item This House believes that the United Kingdom should be a republic.
    \item This House would require the Grand Jury to be used in all criminal cases.
    \item This House believes that no taxation without representation and no representation without taxation.
\end{enumerate}


\end{document}


