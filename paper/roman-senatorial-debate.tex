\documentclass[9pt]{article}

\usepackage{amsmath,amssymb}
\usepackage[utf8]{inputenc}
\usepackage{ctex}
\usepackage{epigraph}
\usepackage{amsthm}
\usepackage{multicol}
\usepackage{array}
\usepackage{booktabs}
\usepackage{threeparttable}

% Remove paragraph indentation for cleaner look
\setlength{\parindent}{0pt}
\setlength{\parskip}{1em}

% Reduce line spacing for more compact text
\linespread{0.2}

% Define theorem environments
\newtheorem{theorem}{Theorem}
\newtheorem{lemma}[theorem]{Lemma}
\newtheorem{corollary}[theorem]{Corollary}
\newtheorem{proposition}[theorem]{Proposition}
\newtheorem{definition}[theorem]{Definition}
\newtheorem{example}[theorem]{Example}
\newtheorem{remark}[theorem]{Remark}
\newtheorem{axiom}[theorem]{Axiom}

\renewcommand{\proofname}{Proof}


% Set CJK main font (for Chinese/Japanese/Korean characters)


\setCJKmainfont{BabelStone Han}




% You can also use \newfontfamily for custom non-CJK fonts if needed
% \setCJKmainfont{JyutcitziWithPMingLiURegular}[Path = ./, Extension = .ttf]
% \setCJKmainfont{JyutcitziWithSourceHanSerifTCRegular}[Path = ./, Extension = .ttf]



% \newfontfamily{\jczPMingLiU}{JyutcitziWithPMingLiURegular}[Path = ./fonts/, Extension = .ttf]
% % This has the best rendition for latin characters 
\newfontfamily{\jcz}{JyutcitziWithSourceHanSerifTCRegular}[Path = ./fonts/, Extension = .ttf]
\newcommand{\jcztext}[1]{{\jcz #1}}

\newfontfamily{\batang}{batang}[Path = ./fonts/, Extension = .ttf]
\newCJKfontfamily\koreanfont{Batang}[Path = ./fonts/, Extension = .ttf]
\newfontfamily{\taigi}{GentiumBookPlus-Regular}[Path = ./fonts/, Extension = .ttf]


% Load ruby package for furigana (Ruby text)

\usepackage{ruby}







% IDC - ideographic description characters
% https://en.wikipedia.org/wiki/Chinese_character_description_languages#Ideographic_Description_Sequences

\newcommand{\superimpose}[2]{{%
  \ooalign{%
    \hfil$\m@th\text{#1}\@firstoftwo\text{#2}$\hfil\cr
    \hfil$\m@th\text{#1}\@secondoftwo\text{#2}$\hfil\cr
  }%
}}


% Define the \tb command

\newcommand{\tb}[2]{%
\scalebox{2}[1]{
\ooalign{%
    \hfil\raisebox{0.25em}{\text{\scalebox{0.33}{#1}}}\hfil\cr % Top text, squished and raised
    \hfil\raisebox{-0.25em}{\text{\scalebox{0.33}{#2}}}\hfil\cr % Bottom text, squished and lowered
  }%
  }
}

% The \lr command - can be combined with \tb
\newcommand{\lr}[2]{
  \scalebox{0.5}[1.0]{#1}\scalebox{0.5}[1.0]{#2}\!\!
}

% Define the \ul command for upper left positioning - for characters like 疒
\newcommand{\ul}[2]{%
  \ooalign{%
    \hfil#1\hfil\cr  % Top text (unscaled)
    \hfil\hspace{0.3em}\scalebox{0.8}{#2}\cr % Bottom text (scaled and raised)
    % \hfil\raisebox{0.2em}{\scalebox{0.5}{#2}}\hfil\cr % Bottom text (scaled and raised)
  }%
}

% Define the \tone command for upper right positioning of a diacritic
\newcommand{\tone}[2]{%
  \ooalign{%
    \hfil#1\hfil\cr  % Main text (unscaled)
    \hfil\hspace{0.9em}\raisebox{0.3em}{\scalebox{0.8}{#2}}\hfil\cr % Tone mark (scaled and raised)
  }%
}

% for vertical Chinese boxes
\usepackage{graphicx} % for \rotatebox

\newfontlanguage{Chinese}{CHN}

\setCJKfamilyfont{BabelStoneVert}[RawFeature={vertical;+vert},Script=CJK,Language=Chinese,Vertical=RotatedGlyphs]{BabelStone Han}

\newcommand*\CJKmovesymbol[1]{\raise.35em\hbox{#1}}
\newcommand*\CJKmove{\punctstyle{plain}% do not modify the spacing between punctuations
  \let\CJKsymbol\CJKmovesymbol
  \let\CJKpunctsymbol\CJKsymbol}

% Define a new environment for vertical text
\newcommand{\VertCell}[1]{\rotatebox{-90}{\CJKfamily{BabelStoneVert}\CJKmove #1}}


% *-----------------------------------------------------------------------*
% | Packages and formatting                                               |
% *-----------------------------------------------------------------------*
% *-----------------------------------------------------------------------*
% | Math & Equations     |
% *-----------------------------------------------------------------------*
\usepackage{amsmath} % For advanced math formatting
\usepackage{amssymb} % For mathematical symbols
% \usepackage{tikz} % For drawing logic decision trees


% *-----------------------------------------------------------------------*
% | Table Management                                                      |
% *-----------------------------------------------------------------------*


\usepackage{graphicx}
\usepackage{array}
\usepackage{tabularx}
\usepackage{tabularray}

\usepackage{float}      % Add the float package
\usepackage{longtable}


\usepackage[table,xcdraw]{xcolor}
 \definecolor{stone}{rgb}{0.7,0.9,0.7} % light green
\definecolor{lightblue}{rgb}{0.8,0.9,1} % light blue
\definecolor{liquid}{rgb}{1,0.8,0.9} % light pink
\definecolor{gold}{rgb}{1,0.9,0.7} % light yellow
\definecolor{gas}{rgb}{0.7,0.8,1} % light blue

\newcommand{\elcol}[3]{\if\relax\detokenize{#1}\relax\begin{tabular}{@{}c@{}}#2\\[0.3em]#3\end{tabular}\else\cellcolor{#1}\begin{tabular}{@{}c@{}}#2\\[0.3em]#3\end{tabular}\fi}




% super long table

\usepackage{pdflscape} % For rotated pages

\setlength{\arrayrulewidth}{0.5mm} % Optional: to thicken table lines
\renewcommand{\arraystretch}{1.5} % Optional: to increase row height


\usepackage{makecell} 


\usepackage{threeparttable}


% *-----------------------------------------------------------------------*
% | Chinese and Soochow Numerals                                          |
% *-----------------------------------------------------------------------*
% numerals.tex
% Define Chinese and Soochow numerals for chapter management

\newcommand{\soochowNumeral}[1]{%
  \ifnum#1<10
    \ifcase#1 〇\or 〡\or 〢\or 〣\or 〤\or 〥\or 〦\or 〧\or 〨\or 〩\fi%
  \else
    \ifnum#1<20
      〸\soochowUnits{\numexpr#1-10\relax}%
    \else
      \ifnum#1<30
        〹\soochowUnits{\numexpr#1-20\relax}%
      \else
        \ifnum#1<40
          〺\soochowUnits{\numexpr#1-30\relax}%
        \else
          \ifnum#1<50
            卅\soochowUnits{\numexpr#1-40\relax}%
          \else
            \ifnum#1<60
              〥十\soochowUnits{\numexpr#1-50\relax}%
            \else
              \ifnum#1<70
                〦十\soochowUnits{\numexpr#1-60\relax}%
              \else
                \ifnum#1<80
                  〧十\soochowUnits{\numexpr#1-70\relax}%
                \else
                  \ifnum#1<90
                    〨十\soochowUnits{\numexpr#1-80\relax}%
                  \else
                    \ifnum#1<100
                      〩十\soochowUnits{\numexpr#1-90\relax}%
                    \fi
                  \fi
                \fi
              \fi
            \fi
          \fi
        \fi
      \fi
    \fi
  \fi
}

\newcommand{\soochowUnits}[1]{%
  \ifnum#1=0
  \else
    \ifnum#1<4
      \ifcase#1 \or 一\or 二\or 三\fi%
    \else
      \soochowNumeral{#1}
    \fi
  \fi
}

\newcommand{\chinesenumeral}[1]{%
  \ifnum#1<10
    \ifcase#1 〇\or 一\or 二\or 三\or 四\or 五\or 六\or 七\or 八\or 九\fi%
  \else
    \ifnum#1<20
      十\chinesenumeral{\numexpr#1-10\relax}%
    \else
      \ifnum#1<30
        二十\chinesenumeral{\numexpr#1-20\relax}%
      \else
        \ifnum#1<40
          三十\chinesenumeral{\numexpr#1-30\relax}%
        \else
          \ifnum#1<50
            四十\chinesenumeral{\numexpr#1-40\relax}%
          \else
            \ifnum#1<60
              五十\chinesenumeral{\numexpr#1-50\relax}%
            \else
              \ifnum#1<70
                六十\chinesenumeral{\numexpr#1-60\relax}%
              \else
                \ifnum#1<80
                  七十\chinesenumeral{\numexpr#1-70\relax}%
                \else
                  \ifnum#1<90
                    八十\chinesenumeral{\numexpr#1-80\relax}%
                  \else
                    \ifnum#1<100
                      九十\chinesenumeral{\numexpr#1-90\relax}%
                    \fi
                  \fi
                \fi
              \fi
            \fi
          \fi
        \fi
      \fi
    \fi
  \fi
}


% *-----------------------------------------------------------------------*
% | Table of contents & Chapter management                                |
% *-----------------------------------------------------------------------*

\renewcommand{\figurename}{圗} % So figures would be labeled with 圗 instead of "figure"

% * * * Now for the table of contents
\renewcommand{\contentsname}{目錄} % Traditional Chinese characters for "Contents"
\setcounter{secnumdepth}{0} % no numbering for sections 
% Increase chapter title size in TOC
\usepackage{tocloft} % For customizing table of contents

% Redefine how chapter numbers are displayed in the TOC using Soochow numerals

% comment this out if you don't want to use the soochow numerlas
\renewcommand{\numberline}[1]{\soochowNumeral{#1}\hspace{1em}} % Use Soochow numerals for TOC chapter numbers
\makeatother


% create index - run \makeindex in the document
\usepackage{makeidx}
\makeindex



% % Custom chapter title formatting with Chinese numeral chapter numbers
\usepackage{titlesec}

% Custom chapter title formatting with Chinese numeral chapter numbers
\titleformat{\chapter}[block] % 'block' means the title appears on a new line
  {\Huge\bfseries} % Font size and bold formatting for the title
  {\soochowNumeral{\thechapter}} % Chinese character for chapter number
  {1em} % Space between the number and the title
  {\Huge} % Custom style for the chapter title itself (can modify)



% *-----------------------------------------------------------------------*
% | Document begins                                                       |
% *-----------------------------------------------------------------------*

\title{Introducing the Roman Senatorial Debate}
\author{James Bong}
\date{9th July 2025}

\begin{document}
% Avoid overfull hbox
\sloppy

% Initialize jyutcitzi processing
\jcz{}

\maketitle

\section{Roman Senatorial Style Debate}

Current competitive debate formats like AP, BP, WSDC are structured such that they lead to disingenuous discourse, intellectual capture, and bias and preference for leftist luxury beliefs. I think some of the responsible features are:

\begin{itemize}
    \item They all require judges - which is a bug that makes the entire game, and its outgrowth ecosystem (debate teams, clubs, debate tutorial schools, national training squads and teams) extremely prone to intellectual capture.
    \item Equity nonsense, which basically prevents hate speech. To those who see value in Cicero's deployment of as hominem attacks as effective highlights of praxological inconsistencies, this custom is highly regrettable. The entire custom abstracts the debate away from reality, breaks the connection between the speaker and his speech, notwithstanding the fact that reality is a place populated by instances of fierce burning hatred and irreconcilable differences.
    \item The game does not have a proxy for a real actual stake that often pushes people into politics, like money or power. People debate because there's politics. There's politics because as a solution to conflict, it's cheaper than violence. There is conflict because people have opposing stakes and interests. Debates abstracts all of that away into just intellectualism, without stakes or interests. At best only the speaker's reputation is at stake.
    \item There is no real room or need for persuasion. The target of persuasion is the judge not the interlocutor. This encourages intellectual antagonism without restraint beyond equity nicities. Why bother agreeing anything the other side has said? You're not in the business of persuading the other side.
\end{itemize}

These weaknesses in the debating format, combined with the natural ego boosting powers of the very exercise of debating, have begotten a class of intellectuals who are:

\begin{itemize}
    \item intellectually dishonest and extremely
    \item Ungentlemanly rhetoric
    \item Divorced from reality
    \item Extremely disrespectful and dismissive towards traditions or rules established through evolution, since their utility may not be readily established through simple speech making made without the intelligence provided by history
    \item Left leaning
    \item Extremely outgroup focused in terms of welfare, big government since there are no costs to debate land government size
\end{itemize}

\section{We need forms of competitive debate - something far more game theoretically unstable}

\subsection{Senatorial Debate Roman}

\begin{itemize}
    \item each debater carries with themselves a vote
    \item Weighted by:
    \begin{itemize}
        \item Their speaker points?
    \end{itemize}
    \item Each person writes down the side they support initially - this is not to be publicly revealed - if revealed - no problem
    \item The difference in the pros and cons are to be fought by the debaters
    \item So if there are 8 people debating, and all 8 voted for ``pro'', and nobody changed sides, the 8-0 points will awarded to 8 people, so 1 point each.
    \item But if it began as 7 for 1 against, then 6 points are to be up for grabs. If it ended as 3 for and 4 against, then the 6 points will be shared amongst the 4 winning, so 1.5 points each
    \item This is the incentive to switch.
    \item The first speaker gets 2 votes
    \item Points to the winning side - points divided
    \item Points to the person who stuck to the end
    \item Minor points to people to people who stuck to the end
    \item Win
    \item Lose
    \item T
\end{itemize}

The individual pay out matrix of a person:

\begin{table}[h]
\centering
\begin{tabular}{l|cc}
\toprule
 & Stick & Switch \\
\midrule
Win & 3 & 1 \\
Lose & 2 & 0 \\
\bottomrule
\end{tabular}
\end{table}

Stick Switch

Win   | 3    , 1

Lose  |  2   , 0

The winning side confers a certain number of points

We need a structure that incentives to switch

\begin{itemize}
    \item If all 8 debaters started with pro then the game theoretically wise thing to do is just to vote immediately.
\end{itemize}

Or rather, we need a system anchors then to a particular position - their ``interests''

The guiding philosophy of this attempt to redesign a debate format from scratch, is that reality, with its patchwork of real personal interests and individual philosophical commitments, should somehow serve as anchors to reality in this new debate game that we are designing, in the same way that they anchor actual parliamentary debates. The question is then, how?

Obviously, we should bear in mind that this is still a game. We are designing a game, that mimics reality, but is not a copy of reality. And in reality, it is possible, though rare, that people's interest do align perfectly. It is possible for a parliament to vote unanimously, without interfercer or underhand meddling. However, that would not make for a very good debate game. This suggests that while debaters should be allowed to choose their own initial sides, based on their own personal beliefs, we should introduce some kind of entropy to encourage or even force disagreement. In traditional debate formats, this disagreement is completely hardcoded, as teams are allotted proposition and opposition slots with zero agency of their own. Here, I suggest perhaps some kind of soft allotment can be considered here.

In reality, aside from just personal beliefs, individuals are motivated to vote based on their interests - monetary interests, familial obligations, relational commitments. It would not terribly reductive to coalese all of that into a number represented by numbers.

Suppose each debater begins with a bag of capital, which represents his interests. His bag is composed entirely of two coins: 1\$ and 2\$. They can either represent ``proposition'' or ``opposition'' - which is randomly determined at the beginning of the debate. For illustration, we will take 1\$ to represent proposition and 2\$ to mean opposition.

\begin{itemize}
    \item At the start of the debate, the debater makes known his view on the motion: whether he's for the motion, or whether he's against the motion - with his coins. These coins are then put into their respective sides' stashs.
    \begin{itemize}
        \item A debater may choose to offer nothing, in which case he does forgoes the stick-to-then-end bonus points when the ultimate vote comes.
    \end{itemize}
    \item After each speech, each debater has to vote. If he votes for the proposition he must vote with the 1\$, and vice versa. His bag of coins therefore constrains how he positions himself.
    \begin{itemize}
        \item He can put in as many coins as there are people debating.
        \item He can choose to offer nothing.
    \end{itemize}
    \item At the end of the debate, the motion is put to a vote.
    \item And the winning side takes the winning stack of coins and half of the losing side, and divides it amongst the winners. The remaining half of the losing stack is then divided amongst those who voted for the losing side since the beginning.
    \begin{itemize}
        \item Note that given this
    \end{itemize}
    \item if there is a tie, the larger stack wins - and is divided amongst the winners. the other stack is then divided amongst those who stuck with their original proposition.
\end{itemize}

The key is perhaps those who stuck to their guns and won will not get any share of the pot

\section{Successful implementation proposal}

\begin{enumerate}
    \item The standard number of debaters is 8.
    \item In the beginning of the debate, each debater must reveal in their hand a number of coins. If they add up to 1\$ it means they support the motion. If they add up to 2\$ it means they're against the motion. This is supposed to be biased for the motion for it to pass. The sums are then pooled into the common pot. And they start to debate.
    \item Each debater makes a speech. During any time of their speech, they may add money to the pot.
    \item At the end of all speeches, votes are counted. The number of votes must reach the 1/2 + 1 number of votes threshold to pass.
    \item Points are distributed thus:
\end{enumerate}

\begin{table}[h]
\centering
\begin{tabular}{l|ccc}
\toprule
 & 留 & 轉軚 & 棄權 \\
\midrule
贏 & 3 & 1 & 0 \\
輸 & 2 & 0 & 0 \\
和 & 0 & 0 & 0 \\
\bottomrule
\end{tabular}
\end{table}

\begin{enumerate}
    \item The pot money is split thus:
    \begin{enumerate}
        \item $s$ is the total sum in the pot, which always abides by the following: $s = w_{\text{留}} + w_{\text{轉}}$
        \item $v$ is the total number of votes on the winning side, which always abides by the following: $v = v_{\text{留}} + v_{\text{轉}}$
        \item $w_{\text{留}}$ and $w_{\text{轉}}$ are defined as such:
        \begin{enumerate}
            \item $w_{\text{留}} = s \cdot \frac{v_{\text{留}}+1}{v+1}$
            \item $w_{\text{轉}} = s \cdot \frac{v_{\text{轉}}}{v+1}$
        \end{enumerate}
        \item To say this without justification, this creates a quasi-prisoner's dilemma situation in the following box, where those who switch sides and win, win money but lose out relatively on points.
        \item the split of the $w_{\text{留}}$ and $w_{\text{轉}}$ leave a weird case where everyone switched sides and that side won. In such a case, there'd be nobody to claim the virtual 留贏 share. That share can go to many places: (1) the poorest member, (2) the hosting tournament, (3) divided equally amongst all players\ldots (4) taken by the winning side as well\ldots
    \end{enumerate}
\end{enumerate}

\begin{table}[h]
\centering
\begin{tabular}{l|cc}
\toprule
 & 留 & 轉軚 \\
\midrule
贏 & 3 & 1 \\
輸 & 2 & 0 \\
\bottomrule
\end{tabular}
\end{table}

The point of the dynamic is: some people care about the money, some people care about passing the motion.

\section{Problematic elimination mechanism}

At breaks, teams are admitted into the elimination rounds by virtue of the points they've accumulated.

\begin{enumerate}
    \item Say you have Octofinals, and you have 8 rooms; 64 people.
    \item Out-rounds: teams are eliminated by points, and then wealth. in the octofinals, the 32 people have to be eliminated
    \item Octos → quarters → semis → finals : 64 → 32 → 16 → 8
    \item the finals are done in the same way: do they want to pass the motion?
\end{enumerate}

At breaks, teams are admitted into the elimination rounds by virtue of the points they've accumulated. If there is a tie, resolve by money. If still tie, coin flip. Or trial by combat. (champions could be allowed or not)

In eliminations suppose you started the thought experiment with 64 people (8*8 rooms, BP logic).

8 - 8 - 8 - 8 - 8 - 8 - 8 - 8

The winning side is the one that got 5 votes. If the room was tied, the entire room is eliminated. So best/most complex/most pregnant/logistically challenging scenario is if all rooms has a resolution, i.e. all returned 5 votes for or against the motion.

so you have 5*8=40 people left

5 - 5 - 5 - 5 - 5 - 5 - 5 - 5

Then we group them again into rooms of 8, so you have 5 rooms

8 - 8 - 8 - 8 - 8

which then again return 5s

5 - 5 - 5 - 5 - 5

So you have 5*5=25 = 24+1 = 8*2 + 9, so you have three rooms

8 - 8 - 9

Again each room needs 5 votes

5 - 5 - 5

so you have 15 people.

With 15 now, you can continue the culling, or you can play with some other variations - introducing a lower house, or a dual voting.

You can do another culling so,

8 - 7

5 - 4

which gives you a 9 man grand final senate

It's not straight forward to work out backwards how the rooms should be distributed if you want to have a 8 person grand final senate.

Anyway, it's more interesting to work with a 15 people senate.

From the 15 people, take out the 8 best (by lot, rank by points, by support, by 黃袍加身 - trial by combat, assassination, I don't care), 8 will serve in the senate. 7 will be in the house.

There are only 4 possibilities as to what their rights are, by and large: a multiplication of whether they have the right to speak, and whether they have the right to vote.

Giving them the right to vote is slightly more straightforward as it merely complicates the game theoretic structure. Giving them the right to speak poses logistic difficulties - but it increases entertainment value.

Maybe POIs is the only form of speech allowable.

I think I will stick to bribery. I think it has multiple advantages, especially when compared to enabling people to buy extra votes, thereby inflating the supply of votes. Most importantly, it introduces questions of how to enforce bribes. obviously I'm not going to supply the mechanism - so it will have to be some kind of an honour and reiterated game theory dynamic. perhaps this will engender a culture of honour.

This begets an interesting question: why isn't bribery of the lowest level of voters - the general election voter, not allowed? I think there are good arguments to be made against being able to bribe parliamentarians, but do those arguments carry over to the general voter?

Bribing parliamentarians enable embezzlement or cronyism - specifically the case of transferring funds to a service provider whose service is purchased by the government. The parliamentary receives a bribe, either directly, or in the form of a kickback.

The key mechanism is that the parliamentarian increased the price the government is willing to pay for the service than the government would have paid if there was no bribe. The difference is then split between the service-provider and the bribe paid to the parliamentarian.

Does this mechanism manifest if ordinary voters are bribed? One might argue it is less likely as the parliamentarian is still

So I think it's safe to say that the case where bribery is enabled for the general voter is more resistant to embezzlement than bribes for parliamentarians. \footnote{Hi}

\section{Payout structure by debater count}

\begin{table}[h]
    \centering
    \small
    \begin{threeparttable}
    \begin{tabular}{cccccccc}    
    \toprule
    $n_{\text{debaters}}$ & $v_{\text{win}}$ & $v_{\text{留}}$ & $v_{\text{轉}}$ & $w_{\text{留}} = 100 \cdot \frac{v_{\text{留}}+1}{v_{win}+1} $ & $w_{\text{轉}} = 100\cdot\frac{v_{\text{轉}}}{v+1}$ & $\text{w}_{\text{留}/\text{人} }$ & $\text{w}_{\text{轉}/\text{人}}$ \\
    \midrule

% \midrule
6 & 4 & 0 & 4 & 20.0 & 80.0 & n/a\tnote{a} & 20.0 \\
6 & 4 & 1 & 3 & 40.0 & 60.0 & 40.0 & 20.0 \\
6 & 4 & 2 & 2 & 60.0 & 40.0 & 30.0 & 20.0 \\
6 & 4 & 3 & 1 & 80.0 & 20.0 & 26.67 & 20.0 \\
6 & 4 & 4 & 0 & 100.0 & 0.0 & 25.0 & 0.0 \\
6 & 5 & 0 & 5 & 16.67 & 83.33 & 0.0 & 16.67 \\
6 & 5 & 1 & 4 & 33.33 & 66.67 & 33.33 & 16.67 \\
6 & 5 & 2 & 3 & 50.0 & 50.0 & 25.0 & 16.67 \\
6 & 5 & 3 & 2 & 66.67 & 33.33 & 22.22 & 16.67 \\
6 & 5 & 4 & 1 & 83.33 & 16.67 & 20.83 & 16.67 \\
6 & 5 & 5 & 0 & 100.0 & 0.0 & 20.0 & 0.0 \\
6 & 6 & 0 & 6 & 14.29 & 85.71 & 0.0 & 14.29 \\
6 & 6 & 1 & 5 & 28.57 & 71.43 & 28.57 & 14.29 \\
6 & 6 & 2 & 4 & 42.86 & 57.14 & 21.43 & 14.29 \\
6 & 6 & 3 & 3 & 57.14 & 42.86 & 19.05 & 14.29 \\
6 & 6 & 4 & 2 & 71.43 & 28.57 & 17.86 & 14.29 \\
6 & 6 & 5 & 1 & 85.71 & 14.29 & 17.14 & 14.29 \\
6 & 6 & 6 & 0 & 100.0 & 0.0 & 16.67 & 0.0 \\
\bottomrule
\end{tabular}
\begin{tablenotes}
    \item[a] n/a means no 留-side participants to receive rewards.
\end{tablenotes}
\end{threeparttable}

\caption{Payout structure for different numbers of debaters and voting scenarios}
\end{table}

\begin{table}[h]
    \centering
    \small
    \begin{threeparttable}
    \begin{tabular}{cccccccc}
    \toprule
    $n_{\text{debaters}}$ & $v_{\text{win}}$ & $v_{\text{留}}$ & $v_{\text{轉}}$ & $w_{\text{留}} = 100 \cdot \frac{v_{\text{留}}+1}{v_{win}+1} $ & $w_{\text{轉}} = 100\cdot\frac{v_{\text{轉}}}{v+1}$ & $\text{w}_{\text{留}/\text{人} }$ & $\text{w}_{\text{轉}/\text{人}}$ \\
    \midrule
7 & 4 & 0 & 4 & 20.0 & 80.0 & n/a & 20.0 \\
7 & 4 & 1 & 3 & 40.0 & 60.0 & 40.0 & 20.0 \\
7 & 4 & 2 & 2 & 60.0 & 40.0 & 30.0 & 20.0 \\
7 & 4 & 3 & 1 & 80.0 & 20.0 & 26.67 & 20.0 \\
7 & 4 & 4 & 0 & 100.0 & 0.0 & 25.0 & 0.0 \\
7 & 5 & 0 & 5 & 16.67 & 83.33 & 0.0 & 16.67 \\
7 & 5 & 1 & 4 & 33.33 & 66.67 & 33.33 & 16.67 \\
7 & 5 & 2 & 3 & 50.0 & 50.0 & 25.0 & 16.67 \\
7 & 5 & 3 & 2 & 66.67 & 33.33 & 22.22 & 16.67 \\
7 & 5 & 4 & 1 & 83.33 & 16.67 & 20.83 & 16.67 \\
7 & 5 & 5 & 0 & 100.0 & 0.0 & 20.0 & 0.0 \\
7 & 6 & 0 & 6 & 14.29 & 85.71 & 0.0 & 14.29 \\
7 & 6 & 1 & 5 & 28.57 & 71.43 & 28.57 & 14.29 \\
7 & 6 & 2 & 4 & 42.86 & 57.14 & 21.43 & 14.29 \\
7 & 6 & 3 & 3 & 57.14 & 42.86 & 19.05 & 14.29 \\
7 & 6 & 4 & 2 & 71.43 & 28.57 & 17.86 & 14.29 \\
7 & 6 & 5 & 1 & 85.71 & 14.29 & 17.14 & 14.29 \\
7 & 6 & 6 & 0 & 100.0 & 0.0 & 16.67 & 0.0 \\
7 & 7 & 0 & 7 & 12.5 & 87.5 & 0.0 & 12.5 \\
7 & 7 & 1 & 6 & 25.0 & 75.0 & 25.0 & 12.5 \\
7 & 7 & 2 & 5 & 37.5 & 62.5 & 18.75 & 12.5 \\
7 & 7 & 3 & 4 & 50.0 & 50.0 & 16.67 & 12.5 \\
7 & 7 & 4 & 3 & 62.5 & 37.5 & 15.62 & 12.5 \\
7 & 7 & 5 & 2 & 75.0 & 25.0 & 15.0 & 12.5 \\
7 & 7 & 6 & 1 & 87.5 & 12.5 & 14.58 & 12.5 \\
7 & 7 & 7 & 0 & 100.0 & 0.0 & 14.29 & 0.0 \\
\bottomrule
\end{tabular}
\begin{tablenotes}
    \item[a] n/a indicates no 留 voters to receive the payout.
\end{tablenotes}
\caption{Payout structure for different numbers of debaters and voting scenarios}
\end{threeparttable}
\end{table}


\begin{table}[h]
    \centering
    \small
    \begin{threeparttable}
    \begin{tabular}{cccccccc}
    \toprule
    $n_{\text{debaters}}$ & $v_{\text{win}}$ & $v_{\text{留}}$ & $v_{\text{轉}}$ & $w_{\text{留}} = 100 \cdot \frac{v_{\text{留}}+1}{v_{win}+1} $ & $w_{\text{轉}} = 100\cdot\frac{v_{\text{轉}}}{v+1}$ & $\text{w}_{\text{留}/\text{人} }$ & $\text{w}_{\text{轉}/\text{人}}$ \\
    \midrule
8 & 5 & 0 & 5 & 16.67 & 83.33 & 0.0 & 16.67 \\
8 & 5 & 1 & 4 & 33.33 & 66.67 & 33.33 & 16.67 \\
8 & 5 & 2 & 3 & 50.0 & 50.0 & 25.0 & 16.67 \\
8 & 5 & 3 & 2 & 66.67 & 33.33 & 22.22 & 16.67 \\
8 & 5 & 4 & 1 & 83.33 & 16.67 & 20.83 & 16.67 \\
8 & 5 & 5 & 0 & 100.0 & 0.0 & 20.0 & 0.0 \\
8 & 6 & 0 & 6 & 14.29 & 85.71 & 0.0 & 14.29 \\
8 & 6 & 1 & 5 & 28.57 & 71.43 & 28.57 & 14.29 \\
8 & 6 & 2 & 4 & 42.86 & 57.14 & 21.43 & 14.29 \\
8 & 6 & 3 & 3 & 57.14 & 42.86 & 19.05 & 14.29 \\
8 & 6 & 4 & 2 & 71.43 & 28.57 & 17.86 & 14.29 \\
8 & 6 & 5 & 1 & 85.71 & 14.29 & 17.14 & 14.29 \\
8 & 6 & 6 & 0 & 100.0 & 0.0 & 16.67 & 0.0 \\
8 & 7 & 0 & 7 & 12.5 & 87.5 & 0.0 & 12.5 \\
8 & 7 & 1 & 6 & 25.0 & 75.0 & 25.0 & 12.5 \\
8 & 7 & 2 & 5 & 37.5 & 62.5 & 18.75 & 12.5 \\
8 & 7 & 3 & 4 & 50.0 & 50.0 & 16.67 & 12.5 \\
8 & 7 & 4 & 3 & 62.5 & 37.5 & 15.62 & 12.5 \\
8 & 7 & 5 & 2 & 75.0 & 25.0 & 15.0 & 12.5 \\
8 & 7 & 6 & 1 & 87.5 & 12.5 & 14.58 & 12.5 \\
8 & 7 & 7 & 0 & 100.0 & 0.0 & 14.29 & 0.0 \\
8 & 8 & 0 & 8 & 11.11 & 88.89 & 0.0 & 11.11 \\
8 & 8 & 1 & 7 & 22.22 & 77.78 & 22.22 & 11.11 \\
8 & 8 & 2 & 6 & 33.33 & 66.67 & 16.67 & 11.11 \\
8 & 8 & 3 & 5 & 44.44 & 55.56 & 14.81 & 11.11 \\
8 & 8 & 4 & 4 & 55.56 & 44.44 & 13.89 & 11.11 \\
8 & 8 & 5 & 3 & 66.67 & 33.33 & 13.33 & 11.11 \\
8 & 8 & 6 & 2 & 77.78 & 22.22 & 12.96 & 11.11 \\
8 & 8 & 7 & 1 & 88.89 & 11.11 & 12.7 & 11.11 \\
8 & 8 & 8 & 0 & 100.0 & 0.0 & 12.5 & 0.0 \\
\bottomrule
\end{tabular}
\begin{tablenotes}
    \item[a] n/a indicates no 留 voters to receive the payout.
\end{tablenotes}
\caption{Payout structure for different numbers of debaters and voting scenarios}
\end{threeparttable}
\end{table}

\begin{table}[h]
    \centering
    \small
    \begin{threeparttable}
    \begin{tabular}{cccccccc}
    \toprule
    $n_{\text{debaters}}$ & $v_{\text{win}}$ & $v_{\text{留}}$ & $v_{\text{轉}}$ & $w_{\text{留}} = 100 \cdot \frac{v_{\text{留}}+1}{v_{win}+1} $ & $w_{\text{轉}} = 100\cdot\frac{v_{\text{轉}}}{v+1}$ & $\text{w}_{\text{留}/\text{人} }$ & $\text{w}_{\text{轉}/\text{人}}$ \\
    \midrule
9 & 5 & 0 & 5 & 16.67 & 83.33 & 0.0 & 16.67 \\
9 & 5 & 1 & 4 & 33.33 & 66.67 & 33.33 & 16.67 \\
9 & 5 & 2 & 3 & 50.0 & 50.0 & 25.0 & 16.67 \\
9 & 5 & 3 & 2 & 66.67 & 33.33 & 22.22 & 16.67 \\
9 & 5 & 4 & 1 & 83.33 & 16.67 & 20.83 & 16.67 \\
9 & 5 & 5 & 0 & 100.0 & 0.0 & 20.0 & 0.0 \\
9 & 6 & 0 & 6 & 14.29 & 85.71 & 0.0 & 14.29 \\
9 & 6 & 1 & 5 & 28.57 & 71.43 & 28.57 & 14.29 \\
9 & 6 & 2 & 4 & 42.86 & 57.14 & 21.43 & 14.29 \\
9 & 6 & 3 & 3 & 57.14 & 42.86 & 19.05 & 14.29 \\
9 & 6 & 4 & 2 & 71.43 & 28.57 & 17.86 & 14.29 \\
9 & 6 & 5 & 1 & 85.71 & 14.29 & 17.14 & 14.29 \\
9 & 6 & 6 & 0 & 100.0 & 0.0 & 16.67 & 0.0 \\
9 & 7 & 0 & 7 & 12.5 & 87.5 & 0.0 & 12.5 \\
9 & 7 & 1 & 6 & 25.0 & 75.0 & 25.0 & 12.5 \\
9 & 7 & 2 & 5 & 37.5 & 62.5 & 18.75 & 12.5 \\
9 & 7 & 3 & 4 & 50.0 & 50.0 & 16.67 & 12.5 \\
9 & 7 & 4 & 3 & 62.5 & 37.5 & 15.62 & 12.5 \\
9 & 7 & 5 & 2 & 75.0 & 25.0 & 15.0 & 12.5 \\
9 & 7 & 6 & 1 & 87.5 & 12.5 & 14.58 & 12.5 \\
9 & 7 & 7 & 0 & 100.0 & 0.0 & 14.29 & 0.0 \\
9 & 8 & 0 & 8 & 11.11 & 88.89 & 0.0 & 11.11 \\
9 & 8 & 1 & 7 & 22.22 & 77.78 & 22.22 & 11.11 \\
9 & 8 & 2 & 6 & 33.33 & 66.67 & 16.67 & 11.11 \\
9 & 8 & 3 & 5 & 44.44 & 55.56 & 14.81 & 11.11 \\
9 & 8 & 4 & 4 & 55.56 & 44.44 & 13.89 & 11.11 \\
9 & 8 & 5 & 3 & 66.67 & 33.33 & 13.33 & 11.11 \\
9 & 8 & 6 & 2 & 77.78 & 22.22 & 12.96 & 11.11 \\
9 & 8 & 7 & 1 & 88.89 & 11.11 & 12.7 & 11.11 \\
9 & 8 & 8 & 0 & 100.0 & 0.0 & 12.5 & 0.0 \\
9 & 9 & 0 & 9 & 10.0 & 90.0 & 0.0 & 10.0 \\
9 & 9 & 1 & 8 & 20.0 & 80.0 & 20.0 & 10.0 \\
9 & 9 & 2 & 7 & 30.0 & 70.0 & 15.0 & 10.0 \\
9 & 9 & 3 & 6 & 40.0 & 60.0 & 13.33 & 10.0 \\
9 & 9 & 4 & 5 & 50.0 & 50.0 & 12.5 & 10.0 \\
9 & 9 & 5 & 4 & 60.0 & 40.0 & 12.0 & 10.0 \\
9 & 9 & 6 & 3 & 70.0 & 30.0 & 11.67 & 10.0 \\
9 & 9 & 7 & 2 & 80.0 & 20.0 & 11.43 & 10.0 \\
9 & 9 & 8 & 1 & 90.0 & 10.0 & 11.25 & 10.0 \\
9 & 9 & 9 & 0 & 100.0 & 0.0 & 11.11 & 0.0 \\
\bottomrule
\end{tabular}
\begin{tablenotes}
    \item[a] n/a indicates no 留 voters to receive the payout.
\end{tablenotes}
\caption{Payout structure for different numbers of debaters and voting scenarios}
\end{threeparttable}
\end{table}

\begin{table}[h]
    \centering
    \small
    \begin{threeparttable}
    \begin{tabular}{cccccccc}
    \toprule
    $n_{\text{debaters}}$ & $v_{\text{win}}$ & $v_{\text{留}}$ & $v_{\text{轉}}$ & $w_{\text{留}} = 100 \cdot \frac{v_{\text{留}}+1}{v_{win}+1} $ & $w_{\text{轉}} = 100\cdot\frac{v_{\text{轉}}}{v+1}$ & $\text{w}_{\text{留}/\text{人} }$ & $\text{w}_{\text{轉}/\text{人}}$ \\
    \midrule
10 & 6 & 0 & 6 & 14.29 & 85.71 & 0.0 & 14.29 \\
10 & 6 & 1 & 5 & 28.57 & 71.43 & 28.57 & 14.29 \\
10 & 6 & 2 & 4 & 42.86 & 57.14 & 21.43 & 14.29 \\
10 & 6 & 3 & 3 & 57.14 & 42.86 & 19.05 & 14.29 \\
10 & 6 & 4 & 2 & 71.43 & 28.57 & 17.86 & 14.29 \\
10 & 6 & 5 & 1 & 85.71 & 14.29 & 17.14 & 14.29 \\
10 & 6 & 6 & 0 & 100.0 & 0.0 & 16.67 & 0.0 \\
10 & 7 & 0 & 7 & 12.5 & 87.5 & 0.0 & 12.5 \\
10 & 7 & 1 & 6 & 25.0 & 75.0 & 25.0 & 12.5 \\
10 & 7 & 2 & 5 & 37.5 & 62.5 & 18.75 & 12.5 \\
10 & 7 & 3 & 4 & 50.0 & 50.0 & 16.67 & 12.5 \\
10 & 7 & 4 & 3 & 62.5 & 37.5 & 15.62 & 12.5 \\
10 & 7 & 5 & 2 & 75.0 & 25.0 & 15.0 & 12.5 \\
10 & 7 & 6 & 1 & 87.5 & 12.5 & 14.58 & 12.5 \\
10 & 7 & 7 & 0 & 100.0 & 0.0 & 14.29 & 0.0 \\
10 & 8 & 0 & 8 & 11.11 & 88.89 & 0.0 & 11.11 \\
10 & 8 & 1 & 7 & 22.22 & 77.78 & 22.22 & 11.11 \\
10 & 8 & 2 & 6 & 33.33 & 66.67 & 16.67 & 11.11 \\
10 & 8 & 3 & 5 & 44.44 & 55.56 & 14.81 & 11.11 \\
10 & 8 & 4 & 4 & 55.56 & 44.44 & 13.89 & 11.11 \\
10 & 8 & 5 & 3 & 66.67 & 33.33 & 13.33 & 11.11 \\
10 & 8 & 6 & 2 & 77.78 & 22.22 & 12.96 & 11.11 \\
10 & 8 & 7 & 1 & 88.89 & 11.11 & 12.7 & 11.11 \\
10 & 8 & 8 & 0 & 100.0 & 0.0 & 12.5 & 0.0 \\
10 & 9 & 0 & 9 & 10.0 & 90.0 & 0.0 & 10.0 \\
10 & 9 & 1 & 8 & 20.0 & 80.0 & 20.0 & 10.0 \\
10 & 9 & 2 & 7 & 30.0 & 70.0 & 15.0 & 10.0 \\
10 & 9 & 3 & 6 & 40.0 & 60.0 & 13.33 & 10.0 \\
10 & 9 & 4 & 5 & 50.0 & 50.0 & 12.5 & 10.0 \\
10 & 9 & 5 & 4 & 60.0 & 40.0 & 12.0 & 10.0 \\
10 & 9 & 6 & 3 & 70.0 & 30.0 & 11.67 & 10.0 \\
10 & 9 & 7 & 2 & 80.0 & 20.0 & 11.43 & 10.0 \\
10 & 9 & 8 & 1 & 90.0 & 10.0 & 11.25 & 10.0 \\
10 & 9 & 9 & 0 & 100.0 & 0.0 & 11.11 & 0.0 \\
10 & 10 & 0 & 10 & 9.09 & 90.91 & 0.0 & 9.09 \\
10 & 10 & 1 & 9 & 18.18 & 81.82 & 18.18 & 9.09 \\
10 & 10 & 2 & 8 & 27.27 & 72.73 & 13.64 & 9.09 \\
10 & 10 & 3 & 7 & 36.36 & 63.64 & 12.12 & 9.09 \\
10 & 10 & 4 & 6 & 45.45 & 54.55 & 11.36 & 9.09 \\
10 & 10 & 5 & 5 & 54.55 & 45.45 & 10.91 & 9.09 \\
10 & 10 & 6 & 4 & 63.64 & 36.36 & 10.61 & 9.09 \\
10 & 10 & 7 & 3 & 72.73 & 27.27 & 10.39 & 9.09 \\
10 & 10 & 8 & 2 & 81.82 & 18.18 & 10.23 & 9.09 \\
10 & 10 & 9 & 1 & 90.91 & 9.09 & 10.1 & 9.09 \\
10 & 10 & 10 & 0 & 100.0 & 0.0 & 10.0 & 0.0 \\
\bottomrule
\end{tabular}
\begin{tablenotes}
    \item[a] n/a indicates no 留 voters to receive the payout.
\end{tablenotes}
\caption{Payout structure for different numbers of debaters and voting scenarios}
\end{threeparttable}
\end{table}



\begin{table}[h]
    \centering
    \small
    \begin{threeparttable}
    \begin{tabular}{cccccccc}
    \toprule    \toprule
    $n_{\text{debaters}}$ & $v_{\text{win}}$ & $v_{\text{留}}$ & $v_{\text{轉}}$ & $w_{\text{留}} = 100 \cdot \frac{v_{\text{留}}+1}{v_{win}+1} $ & $w_{\text{轉}} = 100\cdot\frac{v_{\text{轉}}}{v+1}$ & $\text{w}_{\text{留}/\text{人} }$ & $\text{w}_{\text{轉}/\text{人}}$ \\
    \midrule
11 & 6 & 0 & 6 & 14.29 & 85.71 & 0.0 & 14.29 \\
11 & 6 & 1 & 5 & 28.57 & 71.43 & 28.57 & 14.29 \\
11 & 6 & 2 & 4 & 42.86 & 57.14 & 21.43 & 14.29 \\
11 & 6 & 3 & 3 & 57.14 & 42.86 & 19.05 & 14.29 \\
11 & 6 & 4 & 2 & 71.43 & 28.57 & 17.86 & 14.29 \\
11 & 6 & 5 & 1 & 85.71 & 14.29 & 17.14 & 14.29 \\
11 & 6 & 6 & 0 & 100.0 & 0.0 & 16.67 & 0.0 \\
11 & 7 & 0 & 7 & 12.5 & 87.5 & 0.0 & 12.5 \\
11 & 7 & 1 & 6 & 25.0 & 75.0 & 25.0 & 12.5 \\
11 & 7 & 2 & 5 & 37.5 & 62.5 & 18.75 & 12.5 \\
11 & 7 & 3 & 4 & 50.0 & 50.0 & 16.67 & 12.5 \\
11 & 7 & 4 & 3 & 62.5 & 37.5 & 15.62 & 12.5 \\
11 & 7 & 5 & 2 & 75.0 & 25.0 & 15.0 & 12.5 \\
11 & 7 & 6 & 1 & 87.5 & 12.5 & 14.58 & 12.5 \\
11 & 7 & 7 & 0 & 100.0 & 0.0 & 14.29 & 0.0 \\
11 & 8 & 0 & 8 & 11.11 & 88.89 & 0.0 & 11.11 \\
11 & 8 & 1 & 7 & 22.22 & 77.78 & 22.22 & 11.11 \\
11 & 8 & 2 & 6 & 33.33 & 66.67 & 16.67 & 11.11 \\
11 & 8 & 3 & 5 & 44.44 & 55.56 & 14.81 & 11.11 \\
11 & 8 & 4 & 4 & 55.56 & 44.44 & 13.89 & 11.11 \\
11 & 8 & 5 & 3 & 66.67 & 33.33 & 13.33 & 11.11 \\
11 & 8 & 6 & 2 & 77.78 & 22.22 & 12.96 & 11.11 \\
11 & 8 & 7 & 1 & 88.89 & 11.11 & 12.7 & 11.11 \\
11 & 8 & 8 & 0 & 100.0 & 0.0 & 12.5 & 0.0 \\
11 & 9 & 0 & 9 & 10.0 & 90.0 & 0.0 & 10.0 \\
11 & 9 & 1 & 8 & 20.0 & 80.0 & 20.0 & 10.0 \\
11 & 9 & 2 & 7 & 30.0 & 70.0 & 15.0 & 10.0 \\
11 & 9 & 3 & 6 & 40.0 & 60.0 & 13.33 & 10.0 \\
11 & 9 & 4 & 5 & 50.0 & 50.0 & 12.5 & 10.0 \\
11 & 9 & 5 & 4 & 60.0 & 40.0 & 12.0 & 10.0 \\
11 & 9 & 6 & 3 & 70.0 & 30.0 & 11.67 & 10.0 \\
11 & 9 & 7 & 2 & 80.0 & 20.0 & 11.43 & 10.0 \\
11 & 9 & 8 & 1 & 90.0 & 10.0 & 11.25 & 10.0 \\
11 & 9 & 9 & 0 & 100.0 & 0.0 & 11.11 & 0.0 \\
11 & 10 & 0 & 10 & 9.09 & 90.91 & 0.0 & 9.09 \\
11 & 10 & 1 & 9 & 18.18 & 81.82 & 18.18 & 9.09 \\
11 & 10 & 2 & 8 & 27.27 & 72.73 & 13.64 & 9.09 \\
11 & 10 & 3 & 7 & 36.36 & 63.64 & 12.12 & 9.09 \\
11 & 10 & 4 & 6 & 45.45 & 54.55 & 11.36 & 9.09 \\
11 & 10 & 5 & 5 & 54.55 & 45.45 & 10.91 & 9.09 \\
11 & 10 & 6 & 4 & 63.64 & 36.36 & 10.61 & 9.09 \\
11 & 10 & 7 & 3 & 72.73 & 27.27 & 10.39 & 9.09 \\
11 & 10 & 8 & 2 & 81.82 & 18.18 & 10.23 & 9.09 \\
11 & 10 & 9 & 1 & 90.91 & 9.09 & 10.1 & 9.09 \\
11 & 10 & 10 & 0 & 100.0 & 0.0 & 10.0 & 0.0 \\
11 & 11 & 0 & 11 & 8.33 & 91.67 & 0.0 & 8.33 \\
11 & 11 & 1 & 10 & 16.67 & 83.33 & 16.67 & 8.33 \\
11 & 11 & 2 & 9 & 25.0 & 75.0 & 12.5 & 8.33 \\
11 & 11 & 3 & 8 & 33.33 & 66.67 & 11.11 & 8.33 \\
11 & 11 & 4 & 7 & 41.67 & 58.33 & 10.42 & 8.33 \\
11 & 11 & 5 & 6 & 50.0 & 50.0 & 10.0 & 8.33 \\
11 & 11 & 6 & 5 & 58.33 & 41.67 & 9.72 & 8.33 \\
11 & 11 & 7 & 4 & 66.67 & 33.33 & 9.52 & 8.33 \\
11 & 11 & 8 & 3 & 75.0 & 25.0 & 9.38 & 8.33 \\
11 & 11 & 9 & 2 & 83.33 & 16.67 & 9.26 & 8.33 \\
11 & 11 & 10 & 1 & 91.67 & 8.33 & 9.17 & 8.33 \\
11 & 11 & 11 & 0 & 100.0 & 0.0 & 9.09 & 0.0 \\

\bottomrule
\end{tabular}
\begin{tablenotes}
    \item[a] n/a indicates no 留 voters to receive the payout.
\end{tablenotes}
\caption{Payout structure for different numbers of debaters and voting scenarios}
\end{threeparttable}
\end{table}




\begin{table}[h]
    \centering
    \small
    \begin{threeparttable}
    \begin{tabular}{cccccccc}
    \toprule    \toprule
    $n_{\text{debaters}}$ & $v_{\text{win}}$ & $v_{\text{留}}$ & $v_{\text{轉}}$ & $w_{\text{留}} = 100 \cdot \frac{v_{\text{留}}+1}{v_{win}+1} $ & $w_{\text{轉}} = 100\cdot\frac{v_{\text{轉}}}{v+1}$ & $\text{w}_{\text{留}/\text{人} }$ & $\text{w}_{\text{轉}/\text{人}}$ \\
    \midrule
12 & 7 & 0 & 7 & 12.5 & 87.5 & 0.0 & 12.5 \\
12 & 7 & 1 & 6 & 25.0 & 75.0 & 25.0 & 12.5 \\
12 & 7 & 2 & 5 & 37.5 & 62.5 & 18.75 & 12.5 \\
12 & 7 & 3 & 4 & 50.0 & 50.0 & 16.67 & 12.5 \\
12 & 7 & 4 & 3 & 62.5 & 37.5 & 15.62 & 12.5 \\
12 & 7 & 5 & 2 & 75.0 & 25.0 & 15.0 & 12.5 \\
12 & 7 & 6 & 1 & 87.5 & 12.5 & 14.58 & 12.5 \\
12 & 7 & 7 & 0 & 100.0 & 0.0 & 14.29 & 0.0 \\
12 & 8 & 0 & 8 & 11.11 & 88.89 & 0.0 & 11.11 \\
12 & 8 & 1 & 7 & 22.22 & 77.78 & 22.22 & 11.11 \\
12 & 8 & 2 & 6 & 33.33 & 66.67 & 16.67 & 11.11 \\
12 & 8 & 3 & 5 & 44.44 & 55.56 & 14.81 & 11.11 \\
12 & 8 & 4 & 4 & 55.56 & 44.44 & 13.89 & 11.11 \\
12 & 8 & 5 & 3 & 66.67 & 33.33 & 13.33 & 11.11 \\
12 & 8 & 6 & 2 & 77.78 & 22.22 & 12.96 & 11.11 \\
12 & 8 & 7 & 1 & 88.89 & 11.11 & 12.7 & 11.11 \\
12 & 8 & 8 & 0 & 100.0 & 0.0 & 12.5 & 0.0 \\
12 & 9 & 0 & 9 & 10.0 & 90.0 & 0.0 & 10.0 \\
12 & 9 & 1 & 8 & 20.0 & 80.0 & 20.0 & 10.0 \\
12 & 9 & 2 & 7 & 30.0 & 70.0 & 15.0 & 10.0 \\
12 & 9 & 3 & 6 & 40.0 & 60.0 & 13.33 & 10.0 \\
12 & 9 & 4 & 5 & 50.0 & 50.0 & 12.5 & 10.0 \\
12 & 9 & 5 & 4 & 60.0 & 40.0 & 12.0 & 10.0 \\
12 & 9 & 6 & 3 & 70.0 & 30.0 & 11.67 & 10.0 \\
12 & 9 & 7 & 2 & 80.0 & 20.0 & 11.43 & 10.0 \\
12 & 9 & 8 & 1 & 90.0 & 10.0 & 11.25 & 10.0 \\
12 & 9 & 9 & 0 & 100.0 & 0.0 & 11.11 & 0.0 \\
12 & 10 & 0 & 10 & 9.09 & 90.91 & 0.0 & 9.09 \\
12 & 10 & 1 & 9 & 18.18 & 81.82 & 18.18 & 9.09 \\
12 & 10 & 2 & 8 & 27.27 & 72.73 & 13.64 & 9.09 \\
12 & 10 & 3 & 7 & 36.36 & 63.64 & 12.12 & 9.09 \\
12 & 10 & 4 & 6 & 45.45 & 54.55 & 11.36 & 9.09 \\
12 & 10 & 5 & 5 & 54.55 & 45.45 & 10.91 & 9.09 \\
12 & 10 & 6 & 4 & 63.64 & 36.36 & 10.61 & 9.09 \\
12 & 10 & 7 & 3 & 72.73 & 27.27 & 10.39 & 9.09 \\
12 & 10 & 8 & 2 & 81.82 & 18.18 & 10.23 & 9.09 \\
12 & 10 & 9 & 1 & 90.91 & 9.09 & 10.1 & 9.09 \\
12 & 10 & 10 & 0 & 100.0 & 0.0 & 10.0 & 0.0 \\
12 & 11 & 0 & 11 & 8.33 & 91.67 & 0.0 & 8.33 \\
12 & 11 & 1 & 10 & 16.67 & 83.33 & 16.67 & 8.33 \\
12 & 11 & 2 & 9 & 25.0 & 75.0 & 12.5 & 8.33 \\
12 & 11 & 3 & 8 & 33.33 & 66.67 & 11.11 & 8.33 \\
12 & 11 & 4 & 7 & 41.67 & 58.33 & 10.42 & 8.33 \\
12 & 11 & 5 & 6 & 50.0 & 50.0 & 10.0 & 8.33 \\
12 & 11 & 6 & 5 & 58.33 & 41.67 & 9.72 & 8.33 \\
12 & 11 & 7 & 4 & 66.67 & 33.33 & 9.52 & 8.33 \\
12 & 11 & 8 & 3 & 75.0 & 25.0 & 9.38 & 8.33 \\
12 & 11 & 9 & 2 & 83.33 & 16.67 & 9.26 & 8.33 \\
12 & 11 & 10 & 1 & 91.67 & 8.33 & 9.17 & 8.33 \\
12 & 11 & 11 & 0 & 100.0 & 0.0 & 9.09 & 0.0 \\
12 & 12 & 0 & 12 & 7.69 & 92.31 & 0.0 & 7.69 \\
12 & 12 & 1 & 11 & 15.38 & 84.62 & 15.38 & 7.69 \\
12 & 12 & 2 & 10 & 23.08 & 76.92 & 11.54 & 7.69 \\
12 & 12 & 3 & 9 & 30.77 & 69.23 & 10.26 & 7.69 \\
12 & 12 & 4 & 8 & 38.46 & 61.54 & 9.62 & 7.69 \\
12 & 12 & 5 & 7 & 46.15 & 53.85 & 9.23 & 7.69 \\
12 & 12 & 6 & 6 & 53.85 & 46.15 & 8.97 & 7.69 \\
12 & 12 & 7 & 5 & 61.54 & 38.46 & 8.79 & 7.69 \\
12 & 12 & 8 & 4 & 69.23 & 30.77 & 8.65 & 7.69 \\
12 & 12 & 9 & 3 & 76.92 & 23.08 & 8.55 & 7.69 \\
12 & 12 & 10 & 2 & 84.62 & 15.38 & 8.46 & 7.69 \\
12 & 12 & 11 & 1 & 92.31 & 7.69 & 8.39 & 7.69 \\
12 & 12 & 12 & 0 & 100.0 & 0.0 & 8.33 & 0.0 \\

\bottomrule
\end{tabular}
\begin{tablenotes}
    \item[a] n/a indicates no 留 voters to receive the payout.
\end{tablenotes}
\caption{Payout structure for different numbers of debaters and voting scenarios}
\end{threeparttable}
\end{table}



\end{document}


