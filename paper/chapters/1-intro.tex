\section{Introduction}

I grew up and was trained in the British Parliamentary style in my intellectual youth. It might perhaps be rather unbecoming, embarassing, to still talk about one's "debating days". The person smells as if he has plateaued in the debating arena, and he's hankering to his glorious past at the expense of his interlocuter's patience. But the British Parliamentary style was incredibly influential to me, on my mode of internal dialectic, my rhetoric style, and substantive philosophical commitments. It was particularly impactful because prior to the British Parliamentary style, none of the various formats of debate I was acquainted with offered a comparable degree of intellectual challenge, game complexity, or even just speech time for one to really make a pie \footnote{oeuvre} out of the motion. I was also convinced that the British Parliamentary style itself could serve as truthseeking, soothsaying machine. I was convinced that was a successful gameified implementation of the socratic dialectic method. The fact it was culturally contiguous with its namesake and inspiration the British Parliament, and that it shares nontrivial similarities in both operation and evolutionary tendencies (as I observed them when I was a competitive debater), made this position very attractive. The historical and cultural weight of the style, and the respect accorded by the venues in which debates of this style were often held, automatically lends legitimacy to one's speech, and therefore, a sense of responsibility - the British Parliamentary style was the style in which Oxford debated the motion "This House Would fight for King and Country", a debate that Hitler listened to on the radio and a debate that he frequently cited in his military decisions concerning Britain. The demands and standards of dynamism, quickwittedness, and argumentative flare, sanctioned by the rambunctious institution known as the point of information, was entirely alien to the repetoire of Sinitic intellectual discourse methodologies. I was even convinced that a backward, intellectually inferior, vocabularily and logically impoverished language, through the British Parliamentary style, could be beaten like glass - and acquire the latent and subterrenean instruments of easy good thinking that the English speaker has taken for granted - what I call "English Rationality". In my eyes, all other debate styles such as All Parliamentary, and World Schools', are mere sloppy imitations of the superior British Parliamentary style, watered down for those with weak intellectual palates.

I no longer agree with this. Or rather, I think although the British Parliamentary style can still deliver some of these promises to young and impressionable minds - particularly the intellectually honest, even if they are immature - these benefits do not scale or carry over into the more advanced stages of the style. Once a person or a society has progressed beyond the novical stages of the British Parliamentary experience, the style rapidly degrades into a chamber that breeds and amplifies intellectual pathologies.


A person, and by extension its debating society, quickly approaches the limits of the British Parliamentary style if they are not intellectually honest or responsible - if they don't check the stuff coming out of their mouths against God - and most people today don't do this, ever in their lives. They never ask "what if I am wrong?" - and they never ever imagine a voice or a being or a mind - God - being there being there to give the ultimate answer. They never test their propositions twice in their head. This laziness is of course common amongst all humans. Intellectual fastidiousness is a rare and precious mutation. And the discpline to commit to that fastidiousness without accidental descension to insanity is even rarer. But the point is, this intellectual laziness, along with other mechanisms and forces, form the prime cause of group mind drift. Since all the minds that make up a tournament have minimal will or intellectual steadfastness of their own, a tournament of minds becomes a brownian motion with drift, as the implicit ideological vectors embedded in the choice of motions, equity policies, adjudication programming, among other things, supervene their latent ideological baggage on the intellectual body of debating individuals. \footnote{「風行而草偃,披之無令而民從也。」
"When the wind moves, the grass bends; the people follow him without being commanded." ——《荀子‧君道》 (Xunzi – On Kingship).
}

The British Parliamentary style does not punish intellectual lazy individuals. In fact, no debating style does. No debating style has a mechanism to impose a duty to be intellectually honest and fastidious to oneself. Nor is there encouragement to do so. And so the downstream problems of that intellectual dishonest multiply. Why is there no such imposition? I think the primary reason is that no debating format has ever come into terms with these problems. The British Parliametnary style is guilty of this failure, only because it is the most advanced style, and has the most developed debating community, and is therefore the first to come into grips with this problem.

There are many other conditions that can beget degenerate intellectual group behaviour, and therefore push against the limits of the British Parliamentary style - but regardless of what they are, once a debating community reaches the limits of the British Parliamentary style, it starts to emit harmful intellectual radiation, and it gradually irradiates its debaters, its debating community, and even the politics of the state. In its most mature form, and in the most celebrated events where idols are made, the British Parliamentary style has become a farce.

It has become necessary for one to enumerate and declare the problems and ills that the British Parliamentary style manifests before us, the mechanisms and design flaws responsible for them, and the higher order downstream impacts on the intellectual developments of participants and wider society.

\begin{itemize}
    \item \textbf{The Dictatorship of the Adjudicator Class.} The adjudicator 
    
    - which is a bug that makes the entire game, and its outgrowth ecosystem (debate teams, clubs, debate tutorial schools, national training squads and teams) extremely prone to intellectual capture.

    The British Parliamentary debate organizing authorities and the culture that they themselves and their participants have upheld voluntarily imposes expectations, norms, and customs, on the adjudicator. These norms are toothless in force, as violating them solicits no punishment - noncompliance solicits at best only embarassment, humiliation, or bin-room adjudication duties. 

    But the problem is not at all noncompliance with these norms. The vast and overwhelming number adjudicators willingly and happily adhere to the programming instilled upon them from senior and esteemed adjudicators, and noncompliance is almost unfailingly the result of intellectual misfiring, incompetence, or inexperience - not malice or intellectual unyieldiness.

    This is to say, that the internal policing is done very well. The adjudication programming and its enforcement mechanism via culture imposition and prestige distribution is very effective. There are extremeley rare cases, none actually in my experience, where the adjudicator has ruled entirely based on his own personal belief system entirely untempered by the expectations of the adjudicator class or the debating class. The debater can be reasonably confident what metric the adjudicator will be using to judge the debate, what the conditions of victory and defeat are, and how the adjudicator will apply the metric of adjudication. The outcome is reasonably predictable, and therefore, the game is reasonably fair. There is a game. 

    The fact that such a convergence of expectations could be forged is a miracle. It is almost exactly the same as in Common Law courts where Judge, Prosecution, and Defence all operate on the same standards and expectations - and it's fair game. Even with the introduction of a jury can one still operate with a stable set of expectations. 
    \footnote{Does Sharia courts have the same convergence of expectations?}
    
    So what's the problem? The problem is that it is very effective in imposing the wrong things. The problem is that the programming imposed is bad.


    There is no real judgement. There's only computation. 
    This is why British Parliamentary debate, just like Common Law courts in Britain, Hong Kong, Singapore, and elsewhere, are all 做大戲 - they're just shows. 

    The only exception is America, where court battles are still genuine battles. 


    \item \textbf{Equity} - which basically prevents hate speech. To those who see value in Cicero's deployment of as hominem attacks as effective highlights of praxological inconsistencies, this custom is highly regrettable. The entire custom abstracts the debate away from reality, breaks the connection between the speaker and his speech, notwithstanding the fact that reality is a place populated by instances of fierce burning hatred and irreconcilable differences.
    \item \textbf{Stakelessness - the nonexistent bumps against the decrepit} - The game does not have a proxy for a real actual stake that often pushes people into politics, like money or power. People debate because there's politics. There's politics because as a solution to conflict, it's cheaper than violence. There is conflict because people have opposing stakes and interests. Debates abstracts all of that away into just intellectualism, without stakes or interests. At best only the speaker's reputation is at stake.
    \item \textbf{Persuasion} - There is no real room or need for persuasion. The target of persuasion is the judge not the interlocutor. This encourages intellectual antagonism without restraint beyond equity nicities. Why bother agreeing anything the other side has said? You're not in the business of persuading the other side.
    \item \textbf{Intellectual dishonesty, and mindless intellectual outgrowth.} The British Parliamentary style, like all debate formats, impose on the debater a side. The debater does not choose his or her own side. The reason for this is quite straightforwardly obvious - it's pedagogical. "It's about seeing an issue from different sides". That's fair enough, and very probably very important for young and impressionable minds. The confiscation of the debater's choice of a side, forces the debater to confront and come into terms with the arguments of a side he may not necessarily agree with, and in doing so, compels his neuroplasticity to do its magic. Indeed, very probably very important for young individuals. If they are to ever be good at carrying out the dialectic, with others, and more importantly, within themselves, then they must be able to argue from this view, that view, and possibly the view from nowhere. The debater is trained to be the advocate, the devil's advocate, His Majesty's Government, His Majesty's Most Loyal Opposition, all at the same time.
    
    What this means fundamentally, is that the British Parliamentary style does not care what you really think. Nobody cares what you think really. And indeed, as a debater does more and more rounds, he himself cares less and less what he really thinks as well. Because the debate tournament is an reiterated game, and a competitive game, with prizes of prestige, recognition, and sometimes possibly money attached. The argument as to how the debater is encouraged to search for the argument that can most plausibly win, becomes very obvious. Nobody cares if you believe in what you say.
    
    Why is this an issue? This is an issue,


    1.  mass intellectual drift, no real input - intellectual poverty - mindless evolutionary outgrowth.
    2. it does not fundamentally put one's belief to the test. One can conjure up arguments, but one has no stake or loyalty to those arguments - when those arguments fall, one's ego is not bruised, if one has learnt to disinvest one's ego from the argument made. Arguments made are not arguments one believes in, and arguments made are made for the sake of victory, not for the sake of announcing sincere belief. You are more less likely to find genuine epistemic discovery in a debate than you are to find the moon made of cheese.

    
    
    \item \textbf{Intellectual Capture}
    \item \textbf{No Real Clash or Contension - both sides actually agree.} Oftentimes one might find that both sides of the debate actually agreeing with each other. They're merely disagreeing on the details of the policy. The Opposition often agrees with Government's philosophical and moral charges - in that we should help the poorest, the blacks, the coloured, the disabled, the gays, the colonised, the oppressed, or even better, etc - all of that is conceded and very nice and dandy, but it's the Government's policy that the Opposition finds intolerable.\footnote{Note, that the usual charges of White Privilege globally enjoyed by the Whites can easily be refashioned into similar charges against the Chinese in South East Asia society - especially in Singapore and Malaysia - but you won't find these arguments as fervently and passionately doled out. It's called the Chinese Privilege. I prophesize, if and once these arguments are made in the South East Asian debating circuit, it will not take long to become a tangible political crisis in the region. Indeed, such discourse has already found itself into Singaporean elections. It must not be allowed to fester to the same degree of prevalance, acceptability, and dispersion as the White Privilege discourse. It will bring disaster to South East Asia.}
    
    Therefore, the contension delivered compared to the contension expected, appears anti-climatic, petty and childish, and stratgically clever but intellectually perfunctorious if not mendacious.  

    It vaguely reminds Marx's charge against bourgeois parliamentary democracy where the real conflict is not between parties inside parliament but between the bourgeois parliament and the proletariat masses.

    Why does this happen? Because (1) it's strategically clever, if not obligatory, as it renders the other's side entire stack of points of philosophical and moral grounds irrelevant, as they were conceded. Effectively, the Government has wasted its own time preaching to the Opposition who never intended to not convert; (2) the debaters don't choose their sides, so they really have no reason 

    The downstream effect of this is that unless the motion is explicitly philosophical, or in debating parlance a "value motion", debaters often no longer make any attempt to make a principle argument, 

    \item \textbf{Principle arguments are a waste of time.}
    \item \textbf{Utilitarianism is the one true moral theory - the measure of all things.}
    \item \textbf{Ego Hyperinflation} 
    \item 
\end{itemize}

These weaknesses in the debating format, combined with the natural ego boosting powers of the very exercise of debating, have begotten a class of intellectuals who are:

\begin{itemize}
    \item intellectually dishonest and extremely
    \item Ungentlemanly rhetoric
    \item Divorced from reality
    \item Extremely disrespectful and dismissive towards traditions or rules established through evolution, since their utility may not be readily established through simple speech making made without the intelligence provided by history
    \item Left leaning
    \item Extremely outgroup focused in terms of welfare, big government since there are no costs to debate land government size
\end{itemize}

\section{We need forms of competitive debate - something far more game theoretically unstable}

\subsection{Senatorial Debate Roman}

\begin{itemize}
    \item each debater carries with themselves a vote
    \item Weighted by:
    \begin{itemize}
        \item Their speaker points?
    \end{itemize}
    \item Each person writes down the side they support initially - this is not to be publicly revealed - if revealed - no problem
    \item The difference in the pros and cons are to be fought by the debaters
    \item So if there are 8 people debating, and all 8 voted for ``pro'', and nobody changed sides, the 8-0 points will awarded to 8 people, so 1 point each.
    \item But if it began as 7 for 1 against, then 6 points are to be up for grabs. If it ended as 3 for and 4 against, then the 6 points will be shared amongst the 4 winning, so 1.5 points each
    \item This is the incentive to switch.
    \item The first speaker gets 2 votes
    \item Points to the winning side - points divided
    \item Points to the person who stuck to the end
    \item Minor points to people to people who stuck to the end
    \item Win
    \item Lose
    \item T
\end{itemize}

The individual pay out matrix of a person:

\begin{table}[h]
\centering
\begin{tabular}{l|cc}
\toprule
 & Stick & Switch \\
\midrule
Win & 3 & 1 \\
Lose & 2 & 0 \\
\bottomrule
\end{tabular}
\end{table}

Stick Switch

Win   | 3    , 1

Lose  |  2   , 0

The winning side confers a certain number of points

We need a structure that incentives to switch

\begin{itemize}
    \item If all 8 debaters started with pro then the game theoretically wise thing to do is just to vote immediately.
\end{itemize}

Or rather, we need a system anchors then to a particular position - their ``interests''

The guiding philosophy of this attempt to redesign a debate format from scratch, is that reality, with its patchwork of real personal interests and individual philosophical commitments, should somehow serve as anchors to reality in this new debate game that we are designing, in the same way that they anchor actual parliamentary debates. The question is then, how?

Obviously, we should bear in mind that this is still a game. We are designing a game, that mimics reality, but is not a copy of reality. And in reality, it is possible, though rare, that people's interest do align perfectly. It is possible for a parliament to vote unanimously, without interfercer or underhand meddling. However, that would not make for a very good debate game. This suggests that while debaters should be allowed to choose their own initial sides, based on their own personal beliefs, we should introduce some kind of entropy to encourage or even force disagreement. In traditional debate formats, this disagreement is completely hardcoded, as teams are allotted proposition and opposition slots with zero agency of their own. Here, I suggest perhaps some kind of soft allotment can be considered here.

In reality, aside from just personal beliefs, individuals are motivated to vote based on their interests - monetary interests, familial obligations, relational commitments. It would not terribly reductive to coalese all of that into a number represented by numbers.

Suppose each debater begins with a bag of capital, which represents his interests. His bag is composed entirely of two coins: 1\$ and 2\$. They can either represent ``proposition'' or ``opposition'' - which is randomly determined at the beginning of the debate. For illustration, we will take 1\$ to represent proposition and 2\$ to mean opposition.

\begin{itemize}
    \item At the start of the debate, the debater makes known his view on the motion: whether he's for the motion, or whether he's against the motion - with his coins. These coins are then put into their respective sides' stashs.
    \begin{itemize}
        \item A debater may choose to offer nothing, in which case he does forgoes the stick-to-then-end bonus points when the ultimate vote comes.
    \end{itemize}
    \item After each speech, each debater has to vote. If he votes for the proposition he must vote with the 1\$, and vice versa. His bag of coins therefore constrains how he positions himself.
    \begin{itemize}
        \item He can put in as many coins as there are people debating.
        \item He can choose to offer nothing.
    \end{itemize}
    \item At the end of the debate, the motion is put to a vote.
    \item And the winning side takes the winning stack of coins and half of the losing side, and divides it amongst the winners. The remaining half of the losing stack is then divided amongst those who voted for the losing side since the beginning.
    \begin{itemize}
        \item Note that given this
    \end{itemize}
    \item if there is a tie, the larger stack wins - and is divided amongst the winners. the other stack is then divided amongst those who stuck with their original proposition.
\end{itemize}

The key is perhaps those who stuck to their guns and won will not get any share of the pot


\subsection{Problems of the British Parliamentary style}

\begin{itemize}
    \item Ideological capture by core adjudicators and judges
    \item Absence of real stakes (infinite money fiat)
    \item Fundamentally leaning towards youthful idealism—leading to leftism drift
    \item Performativity—actually both sides agree 
    \item No room for genuine persuasion—the target of persuasion is the judge not the interlocutor
    \item Intellectual dishonesty—there's no real requirement for one to believe what one is saying; there's no confrontation of one's conscience and convictions against realistic or pragmatic considerations
    \item There's no training in politicking
    \item There is no force. Therefore the ultimate maintenance and begetter of parliamentary decorum and intellectual honesty—violence, or the threat thereof—is absent
    \item Extremely disrespectful and dismissive towards traditions or rules established through evolution, since their utility may not be readily established through simple speech making made without the intelligence provided by history 
    \item As per the analytic tradition of anglo-american philosophy, the British Parliamentary style is utterly incapable in supporting or even tolerating the admission of institutions whose legitimacy is unclear, whose purpose or justification cannot be articulated, or whose rationality is absent. Institutions of these kinds are often called customs, traditions, or even just prejudices dogmas, and myths. These institutions are not well-justified because they weren't designed by man. They evolved into being, through trial and error. Only that which has no history can be defined, only that can be defined can be designed. But customs, traditions, prejudices, dogmas, and myths, are all the product of history. They are not well-justified because they are undefinable. How can you define a woman? The truth is that any answer appealing to sex organs, chromosomes, biology, are also intellectually irrigorous like, though not as functionally ludicrous as those who claim a woman is what you want a woman to be. But, examples - 
    \item Social Contract Theory 
    \item Gender Roles 
    \item Marriage
    \item Religion -
    \item Fancy clothes - legal wigs, and robes 
    \item Property Rights 
    \item Sexual Taboos, Manners
    
    This is also the problem of modern politics in the English speaking world, and is a broader effect of the Enlightenment project.
    \item intellectual fictions 
    \item Social Contract 
    Chesterton’s Fence (G.K. Chesterton):


    
    \item Extremely outgroup focused in terms of welfare, big government since there are no costs to debate government size

    \item 
\end{itemize}


