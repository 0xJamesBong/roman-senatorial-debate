\section{Introduction}

% Current competitive debate formats like AP, BP, WSDC are structured such that they lead to disingenuous discourse, intellectual capture, and bias and preference for leftist luxury beliefs. I think some of the responsible features are:

% \begin{itemize}
%     \item They all require judges - which is a bug that makes the entire game, and its outgrowth ecosystem (debate teams, clubs, debate tutorial schools, national training squads and teams) extremely prone to intellectual capture.
%     \item Equity nonsense, which basically prevents hate speech. To those who see value in Cicero's deployment of as hominem attacks as effective highlights of praxological inconsistencies, this custom is highly regrettable. The entire custom abstracts the debate away from reality, breaks the connection between the speaker and his speech, notwithstanding the fact that reality is a place populated by instances of fierce burning hatred and irreconcilable differences.
%     \item The game does not have a proxy for a real actual stake that often pushes people into politics, like money or power. People debate because there's politics. There's politics because as a solution to conflict, it's cheaper than violence. There is conflict because people have opposing stakes and interests. Debates abstracts all of that away into just intellectualism, without stakes or interests. At best only the speaker's reputation is at stake.
%     \item There is no real room or need for persuasion. The target of persuasion is the judge not the interlocutor. This encourages intellectual antagonism without restraint beyond equity nicities. Why bother agreeing anything the other side has said? You're not in the business of persuading the other side.
% \end{itemize}

% These weaknesses in the debating format, combined with the natural ego boosting powers of the very exercise of debating, have begotten a class of intellectuals who are:

% \begin{itemize}
%     \item intellectually dishonest and extremely
%     \item Ungentlemanly rhetoric
%     \item Divorced from reality
%     \item Extremely disrespectful and dismissive towards traditions or rules established through evolution, since their utility may not be readily established through simple speech making made without the intelligence provided by history
%     \item Left leaning
%     \item Extremely outgroup focused in terms of welfare, big government since there are no costs to debate land government size
% \end{itemize}

% \section{We need forms of competitive debate - something far more game theoretically unstable}

% \subsection{Senatorial Debate Roman}

% \begin{itemize}
%     \item each debater carries with themselves a vote
%     \item Weighted by:
%     \begin{itemize}
%         \item Their speaker points?
%     \end{itemize}
%     \item Each person writes down the side they support initially - this is not to be publicly revealed - if revealed - no problem
%     \item The difference in the pros and cons are to be fought by the debaters
%     \item So if there are 8 people debating, and all 8 voted for ``pro'', and nobody changed sides, the 8-0 points will awarded to 8 people, so 1 point each.
%     \item But if it began as 7 for 1 against, then 6 points are to be up for grabs. If it ended as 3 for and 4 against, then the 6 points will be shared amongst the 4 winning, so 1.5 points each
%     \item This is the incentive to switch.
%     \item The first speaker gets 2 votes
%     \item Points to the winning side - points divided
%     \item Points to the person who stuck to the end
%     \item Minor points to people to people who stuck to the end
%     \item Win
%     \item Lose
%     \item T
% \end{itemize}

% The individual pay out matrix of a person:

% \begin{table}[h]
% \centering
% \begin{tabular}{l|cc}
% \toprule
%  & Stick & Switch \\
% \midrule
% Win & 3 & 1 \\
% Lose & 2 & 0 \\
% \bottomrule
% \end{tabular}
% \end{table}

% Stick Switch

% Win   | 3    , 1

% Lose  |  2   , 0

% The winning side confers a certain number of points

% We need a structure that incentives to switch

% \begin{itemize}
%     \item If all 8 debaters started with pro then the game theoretically wise thing to do is just to vote immediately.
% \end{itemize}

% Or rather, we need a system anchors then to a particular position - their ``interests''

% The guiding philosophy of this attempt to redesign a debate format from scratch, is that reality, with its patchwork of real personal interests and individual philosophical commitments, should somehow serve as anchors to reality in this new debate game that we are designing, in the same way that they anchor actual parliamentary debates. The question is then, how?

% Obviously, we should bear in mind that this is still a game. We are designing a game, that mimics reality, but is not a copy of reality. And in reality, it is possible, though rare, that people's interest do align perfectly. It is possible for a parliament to vote unanimously, without interfercer or underhand meddling. However, that would not make for a very good debate game. This suggests that while debaters should be allowed to choose their own initial sides, based on their own personal beliefs, we should introduce some kind of entropy to encourage or even force disagreement. In traditional debate formats, this disagreement is completely hardcoded, as teams are allotted proposition and opposition slots with zero agency of their own. Here, I suggest perhaps some kind of soft allotment can be considered here.

% In reality, aside from just personal beliefs, individuals are motivated to vote based on their interests - monetary interests, familial obligations, relational commitments. It would not terribly reductive to coalese all of that into a number represented by numbers.

% Suppose each debater begins with a bag of capital, which represents his interests. His bag is composed entirely of two coins: 1\$ and 2\$. They can either represent ``proposition'' or ``opposition'' - which is randomly determined at the beginning of the debate. For illustration, we will take 1\$ to represent proposition and 2\$ to mean opposition.

% \begin{itemize}
%     \item At the start of the debate, the debater makes known his view on the motion: whether he's for the motion, or whether he's against the motion - with his coins. These coins are then put into their respective sides' stashs.
%     \begin{itemize}
%         \item A debater may choose to offer nothing, in which case he does forgoes the stick-to-then-end bonus points when the ultimate vote comes.
%     \end{itemize}
%     \item After each speech, each debater has to vote. If he votes for the proposition he must vote with the 1\$, and vice versa. His bag of coins therefore constrains how he positions himself.
%     \begin{itemize}
%         \item He can put in as many coins as there are people debating.
%         \item He can choose to offer nothing.
%     \end{itemize}
%     \item At the end of the debate, the motion is put to a vote.
%     \item And the winning side takes the winning stack of coins and half of the losing side, and divides it amongst the winners. The remaining half of the losing stack is then divided amongst those who voted for the losing side since the beginning.
%     \begin{itemize}
%         \item Note that given this
%     \end{itemize}
%     \item if there is a tie, the larger stack wins - and is divided amongst the winners. the other stack is then divided amongst those who stuck with their original proposition.
% \end{itemize}

% The key is perhaps those who stuck to their guns and won will not get any share of the pot


\subsection{Problems of the British Parliamentary style}

\begin{itemize}
    \item Ideological capture by core adjudicators and judges
    \item Absence of real stakes (infinite money fiat)
    \item Fundamentally leaning towards youthful idealism—leading to leftism drift
    \item Performativity—actually both sides agree 
    \item No room for genuine persuasion—the target of persuasion is the judge not the interlocutor
    \item Intellectual dishonesty—there's no real requirement for one to believe what one is saying; there's no confrontation of one's conscience and convictions against realistic or pragmatic considerations
    \item There's no training in politicking
    \item There is no force. Therefore the ultimate maintenance and begetter of parliamentary decorum and intellectual honesty—violence, or the threat thereof—is absent
    \item Extremely disrespectful and dismissive towards traditions or rules established through evolution, since their utility may not be readily established through simple speech making made without the intelligence provided by history 
    
    \item Extremely outgroup focused in terms of welfare, big government since there are no costs to debate government size
    
\end{itemize}


