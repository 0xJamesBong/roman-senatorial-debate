


\section{Problematic elimination mechanism}

Unlike the British Parliamentary style


At breaks, teams are admitted into the elimination rounds by virtue of the points they've accumulated.

\begin{enumerate}
    \item Say you have Octofinals, and you have 8 rooms; 64 people.
    \item Out-rounds: teams are eliminated by points, and then wealth. in the octofinals, the 32 people have to be eliminated
    \item Octos → quarters → semis → finals : 64 → 32 → 16 → 8
    \item the finals are done in the same way: do they want to pass the motion?
\end{enumerate}

At breaks, teams are admitted into the elimination rounds by virtue of the points they've accumulated. If there is a tie, resolve by money. If still tie, coin flip. Or trial by combat. (champions could be allowed or not)

In eliminations suppose you started the thought experiment with 64 people (8*8 rooms, BP logic).

8 - 8 - 8 - 8 - 8 - 8 - 8 - 8

The winning side is the one that got 5 votes. If the room was tied, the entire room is eliminated. So best/most complex/most pregnant/logistically challenging scenario is if all rooms has a resolution, i.e. all returned 5 votes for or against the motion.

so you have 5*8=40 people left

5 - 5 - 5 - 5 - 5 - 5 - 5 - 5

Then we group them again into rooms of 8, so you have 5 rooms

8 - 8 - 8 - 8 - 8

which then again return 5s

5 - 5 - 5 - 5 - 5

So you have 5*5=25 = 24+1 = 8*2 + 9, so you have three rooms

8 - 8 - 9

Again each room needs 5 votes

5 - 5 - 5

so you have 15 people.

With 15 now, you can continue the culling, or you can play with some other variations - introducing a lower house, or a dual voting.

You can do another culling so,

8 - 7

5 - 4

which gives you a 9 man grand final senate

It's not straight forward to work out backwards how the rooms should be distributed if you want to have a 8 person grand final senate.

Anyway, it's more interesting to work with a 15 people senate.

From the 15 people, take out the 8 best (by lot, rank by points, by support, by 黃袍加身 - trial by combat, assassination, I don't care), 8 will serve in the senate. 7 will be in the house.

There are only 4 possibilities as to what their rights are, by and large: a multiplication of whether they have the right to speak, and whether they have the right to vote.

Giving them the right to vote is slightly more straightforward as it merely complicates the game theoretic structure. Giving them the right to speak poses logistic difficulties - but it increases entertainment value.

Maybe POIs is the only form of speech allowable.

I think I will stick to bribery. I think it has multiple advantages, especially when compared to enabling people to buy extra votes, thereby inflating the supply of votes. Most importantly, it introduces questions of how to enforce bribes. obviously I'm not going to supply the mechanism - so it will have to be some kind of an honour and reiterated game theory dynamic. perhaps this will engender a culture of honour.

This begets an interesting question: why isn't bribery of the lowest level of voters - the general election voter, not allowed? I think there are good arguments to be made against being able to bribe parliamentarians, but do those arguments carry over to the general voter?

Bribing parliamentarians enable embezzlement or cronyism - specifically the case of transferring funds to a service provider whose service is purchased by the government. The parliamentary receives a bribe, either directly, or in the form of a kickback.

The key mechanism is that the parliamentarian increased the price the government is willing to pay for the service than the government would have paid if there was no bribe. The difference is then split between the service-provider and the bribe paid to the parliamentarian.

Does this mechanism manifest if ordinary voters are bribed? One might argue it is less likely as the parliamentarian is still

So I think it's safe to say that the case where bribery is enabled for the general voter is more resistant to embezzlement than bribes for parliamentarians. \footnote{Hi}