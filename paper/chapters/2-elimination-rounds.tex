


\section{Elimination mechanism}

Unlike the British Parliamentary style tournament where you can set the number of rooms depending on how many outrounds you want, and then the teams are exponentially and deterministically eliminated, the Roman Senatorial Debate does not work like that. 

A chamber might fail to resolve the motion, in which case every senator gets 0 points. A chamber might also resolve the motion unanimously without switching sides, in which case every senator gets 3 points. These irregularities pose difficulties to a straightforward elimination mechanism like that in the British Parliamentary style tournament, as a chamber might return all 8 senators with 3 points, or all senators with 0 points. 











\begin{enumerate}
    \item \textbf{Admission to elimination rounds.} At breaks, teams are admitted into the elimination rounds by virtue of the points they've accumulated. 
    \item \textbf{Tie resolution.} In any case where there is a tie, this either resolved by money, or force. In the case where the option to resolve it by force is enabled, any tied member has the right to resolve it by trial by combat. Women may be allowed to fight, but they are accorded the privilege to find a champion. Men are not accorded the privlege of champerty. Champions must be participating senators. This is in line with the general philosophy of the tournament that debate, politics, and intellectualism makes sense only if it is a game of honour - and honour only emerges if there is an ultimate backstop of violence.
    \item \textbf{Points carryover.} Senators will continue to inherit their points from the preliminary rounds. Senators WILL NOT have their accumulated points reset to 0 in the breakrounds. The point of this is to give intellectual autonomy to the senators to stand their ground. 


    \item \textbf{Number of participants.} The number of senators the tournament admits into the outrounds is the number of rounds times 8 senators. we have: 
        \begin{enumerate}
            \item octos (8 strongest rooms, 4th last round): 64 人 $\rightarrow$
            \item quarters (4 strongest rooms, 3rd last round): 32 人 $\rightarrow$ 
            \item semis (2 strongest rooms, 2nd last round): 16 人 $\rightarrow$ 
            \item grand final (last round): 8 人.
        \end{enumerate}
    
    \item \textbf{Elimination mechanism.} In each of the outrounds, the aggregate half of the rooms with the lowest points are eliminated. So unlike BP where the result of each room individually decides the outcome of the outround, here the aggregate result of all rooms in the outround decides the outcome.
    
    
    \item \textbf{The Market of Speech and Influence.} Since senators carry over their points from the preliminary rounds, those with high scores can speak freely with their immediate fear of elimination heavily discounted. This grants intellectual autonomy to top-performing senators, allowing them to assert their own private views with confidence and demonstrate their full intellectual prowess. This means there is now a privilege to speak, where you can basically air your own views on a topic, and you granted an audience. This, on the whole, means that the tournament becomes a market where the commodity of speech and influence can be acquired by a combination of strategic political manoeuvre and careful money management.
    
    \item \textbf{The Grand Final.} How is the winner of the Grand Final determined? This is tricky and it depends on what matters to you. The recommendation is that the you still run the debate as a normal round, but depending on the tournament's interest, one may arbitrarily define the "winner(s)" to be whoever stuck to his guns and whose side resolved the motion. 







    
\end{enumerate}






It's not straight forward to work out backwards how the rooms should be distributed if you want to have a 8 person grand final senate.

Anyway, it's more interesting to work with a 15 people senate.

From the 15 people, take out the 8 best (by lot, rank by points, by support, by 黃袍加身 - trial by combat, assassination, I don't care), 8 will serve in the senate. 7 will be in the house.

There are only 4 possibilities as to what their rights are, by and large: a multiplication of whether they have the right to speak, and whether they have the right to vote.

Giving them the right to vote is slightly more straightforward as it merely complicates the game theoretic structure. Giving them the right to speak poses logistic difficulties - but it increases entertainment value.

Maybe POIs is the only form of speech allowable.

I think I will stick to bribery. I think it has multiple advantages, especially when compared to enabling people to buy extra votes, thereby inflating the supply of votes. Most importantly, it introduces questions of how to enforce bribes. obviously I'm not going to supply the mechanism - so it will have to be some kind of an honour and reiterated game theory dynamic. perhaps this will engender a culture of honour.

This begets an interesting question: why isn't bribery of the lowest level of voters - the general election voter, not allowed? I think there are good arguments to be made against being able to bribe parliamentarians, but do those arguments carry over to the general voter?

Bribing parliamentarians enable embezzlement or cronyism - specifically the case of transferring funds to a service provider whose service is purchased by the government. The parliamentary receives a bribe, either directly, or in the form of a kickback.

The key mechanism is that the parliamentarian increased the price the government is willing to pay for the service than the government would have paid if there was no bribe. The difference is then split between the service-provider and the bribe paid to the parliamentarian.

Does this mechanism manifest if ordinary voters are bribed? One might argue it is less likely as the parliamentarian is still

So I think it's safe to say that the case where bribery is enabled for the general voter is more resistant to embezzlement than bribes for parliamentarians. \footnote{Hi}