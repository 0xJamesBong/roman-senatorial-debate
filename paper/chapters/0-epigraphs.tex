% \section{Epigraphs}

\begin{table}[H]
\centering
\small
\begin{tabular}{p{0.45\textwidth}p{0.45\textwidth}}
\epigraph{Speech was bestowed upon man to disguise his thoughts.}{---Tiberius (as quoted by Tacitus, Annals VI.6)} &
\epigraph{In the Senate, men do not speak the truth. They speak to win.}{---(Paraphrase of sentiments found throughout Sallust and Cicero)} \\
\epigraph{The Senate is a council of kings.}{---Ennius (quoted by Cicero)} &
\epigraph{Let arms yield to the toga, the laurel of the triumph to the tongue of the orator.}{---Cicero, Pro Milone} \\
\epigraph{Senators are not ashamed to praise whom they must, nor to attack whom they dare.}{---Tacitus, Annals I.74} &
\epigraph{The more corrupt the state, the more numerous the laws.}{---Tacitus (Ab excessu divi Augusti)} \\
\end{tabular}
\end{table}

\begin{table}[H]
\centering
\small
\begin{tabular}{p{0.45\textwidth}p{0.45\textwidth}}
\epigraph{The parliament passes some acts or decree which may have the most devastating consequences, yet nobody bears the responsibility for it. Nobody can be called to account. For surely one cannot say that a Cabinet discharges its responsibility when it retires after having brought about a catastrophe. Or can we say that the responsibility is fully discharged when a new coalition is formed or parliament dissolved? Can the principle of responsibility mean anything else than the responsibility of a definite person?}{---Adolf Hitler, Mein Kampf} &
\epigraph{Parliament can do anything but make a man a woman and a woman a man.}{---Sir Edward Coke (attributed)} \\
\epigraph{The British Constitution has always been puzzling and paradoxical. It is an unwritten constitution, and yet it is written all over the place.}{---E.M. Forster} &
\epigraph{The British Parliament is like a curious old clock: its hands move, but no one quite knows why.}{---Aneurin Bevan (paraphrased)} \\
\end{tabular}
\end{table}

\begin{table}[H]
\centering
\small
\begin{tabular}{p{0.45\textwidth}p{0.45\textwidth}}
\epigraph{The House of Commons is the longest running farce in the West End.}{---Clement Attlee (allegedly said in frustration during debates)} &
\epigraph{Parliament is a talking shop.}{---Common saying, derisive or admiring depending on tone} \\
\epigraph{It is the duty of the opposition to oppose.}{---Stanley Baldwin} &
\epigraph{In the Parliament of a free nation, all men are entitled to speak, and all others are entitled not to listen.}{---Attributed to Lord Melbourne} \\
\end{tabular}
\end{table}

\begin{table}[H]
\centering
\small
\begin{tabular}{p{0.45\textwidth}p{0.45\textwidth}}
\epigraph{Congress is so strange. A man gets up to speak and says nothing. Nobody listens—and then everybody disagrees.}{---Boris Marshalov (often misattributed to Will Rogers)} &
\epigraph{In America, anybody can be president. That's one of the risks you take.}{---Adlai Stevenson} \\
\end{tabular}
\end{table}

\begin{table}[H]
\centering
\small
\begin{tabular}{p{0.45\textwidth}p{0.45\textwidth}}
\epigraph{With few exceptions, democracy has not brought good government to new developing countries. What it has done is to disrupt authority and to make it difficult for governments to take decisive action.}{---Lee Kuan Yew, The Wit and Wisdom of Lee Kuan Yew} &
\epigraph{I do not believe that democracy necessarily leads to development. I believe what a country needs to develop is discipline more than democracy.}{---Lee Kuan Yew, Interview with Fareed Zakaria, Foreign Affairs, 1994} \\
\end{tabular}
\end{table}

\begin{table}[H]
\centering
\small
\begin{tabular}{p{0.45\textwidth}p{0.45\textwidth}}
\epigraph{I'm not intellectually convinced that one-man-one-vote is the best. We practice it because that's what the British bequeathed us and we haven't really found a need to challenge that. But we have tweaked the system to prevent it from producing a third-world parliament.}{---Lee Kuan Yew, The Man and His Ideas (interview collection)} &
\epigraph{If all you have is one man, one vote, then the numerically largest group is going to dominate the rest, and in a multi-racial society, that's the recipe for disaster.}{---Lee Kuan Yew, Quoted in Lee Kuan Yew: Hard Truths to Keep Singapore Going (2011)} \\
\end{tabular}
\end{table}

