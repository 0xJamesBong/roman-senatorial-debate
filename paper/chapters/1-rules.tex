\section{Successful implementation proposal}

Here we present the rules of a round of Roman Senatorial Debate.

\begin{enumerate}
    \item  \textbf{Number of debaters.} The standard number of debaters is 8. Speaking time is 7 minutes.
    \begin{itemize}
        \item The Roman Sentorial Debate format should work from 4 debaters to any number of debaters, even beyond 8, but for the purpose of a tournament, rooms should be assumed to accommodate 8 debaters, so the pace would mirror that of the British Parliamentary style.  
    \end{itemize}
    \item \textbf{Choosing sides.} Before the first speech, each debater must reveal in their hand a number of coins. A sum of 1\$ means they support the motion, and so they're on the proposition, Yay (Y). A sum of 2\$ means they're against the motion, and so they're on the opposition, the Nay side (N). Anything else, including 0\$ indicates they're undecided (U). The sides chosen by the debaters must then be recorded, and so they must be seated appropriately. Rearrange the chairs if necessary so it is clear to everywhere all debaters stand. All sums are then pooled into the jackpot. Then they start the debate.
    \begin{itemize}
        \item The imbalance is intentionally weighted in favor of the motion, making it easier to support than oppose. This is partly to incentivize whoever sponsoring or organizing the debate to feel that their time, energy, and money have been adequately spent in advancing an ideological position that they themselves favour.
        \item The higher cost of opposition is also there to ensure that resisting the motion is a matter of genuine conviction, not casual contrarianism—it makes conscience expensive, and therefore meaningful.
        \item The sum could be scaled up to (prop: 10\$, opp: 20\$), or even (prop: 10,000\$, opp: 20,000\$), depending on the tournament's debater demographic. But the ratio should be kept at 1:2. 
        \item Senators who began 
    \end{itemize}

    \item \textbf{Speaking order.} There are multiple ways to determine the speaking order. Here we present two methods, one simple and one sophisticated. 
        \begin{itemize}
        \item \textbf{Simple method.} The speaking queue is ordered by Y, N, U, in that order, until all sides run out. If one side has already run out, the next side continues on the queue.
        \begin{itemize}
        \item  Say you have 8 debaters, 7 Ys, and 1Ns, then the queue is (Y, N, Y, Y, Y, Y, Y, Y).
        \item  Another example, say you have 8 debaters, 3 Ys, 4 Ns, and 1 U, then the queue is (Y, N, U, Y, N, Y, N, N). 
        \item One more example, say 2 Ys, 5 Ns, and 1 U, then the queue is (Y, N, U, Y, N, N, N, N).
        \item The point of stacking the side where there is a large number of supporting senators towards the "end" of the queue is to (1) give the oppoosing side more "time" for them to make clear their point earlier, and so their speech can have more impact, and (2) to give the more numerous side more time to reconsider their position. 
        \end{itemize}

        
        \item \textbf{Sophisticated method.} The speaker order is not determined, but bidded. Whenever the floor is open, senators may bid for the speaking slot. There are three bids. All unsuccessful bids are pooled into the jackpot. The highest bidder gets the speaking slot. If there are no bidders, then the speaker to the left of the last speaker gets the speaking slot. 
        \begin{itemize}
            \item Obviously, this method of determining speaking orders is more monetarily competitive, and involves more strategizing. It also inflates the jackpot.
        \end{itemize}
    \end{itemize}
    
    \item Each senator makes a 7 minute speech. Any sitting may raise points of information during the 1st and 6th minute of the speech. During any time of their speech, they may add money to the pot.
    \item \textbf{Voting and resolution.} When all speeches have been, the chamber shall undergo a round of voting. This ultimate round of voting involves no monetary commitment. The vote is recorded by name. Each senator must write down on a piece of paper, his name,the side he votes for, and then all the votes are counted and revealed public. Votes with no names are invalid. Senators whose vote are not recorded are considered to have abstained. Senators may vote in favour, in opposition, or abstain or remain undecided. The threshold for winning is 1 + half of the total number of debaters. This means in a chamber of 8 senators, for Proposition or Opposition to win, it must secure 5 votes. The Proposition wins if and only it has secured 5 votes, in which case we say the motion is passed, or adopted. The Opposition wins if and only if it has secured 5 votes as well, in which case we say the motion is defeated, or rejected. In the case where the vote is split, perhaps evenly 4-4, or 4-3-1 in Y-N-U, then we say the motion is unresolved, in which case we do not say the motion has passed or defeated. In such a case we also say the chamber is tied.

    \item \textbf{Points and Jackpot} The general philosophy is that points is what counts towards breaks - senators with the highest points break into the elimination rounds. Jackpot is monetary reward, and fuel. They do not count in any way towards breaks. 
    \begin{itemize}
        \item The points are distributed thus, to every senator:
        \begin{table}[h]
            \centering
            \begin{tabular}{l|ccc}
            \toprule
             & stay 留 & switch 轉軚 & abstain 棄權 \\
            \midrule
            win 贏 & 3 & 1 & 0 \\
            lose 輸 & 2 & 0 & 0 \\
            tie 和 & 0 & 0 & 0 \\
            \bottomrule
            \end{tabular}
        \end{table}
        \item Senators who were undecided at the start of the debate and voted in favour or against the motion are considered to have switched sides, and are rewarded 1 point.
        \item You can see that those who lost the debate but stuck to their guns are rewarded more highly in terms of points than those who switched sides. 
        \item You can see that ties are very punishing to all senators involved. It basically means they have all wasted time and everyone has lost. 
    \end{itemize}

    \item The jackpot money is split thus:
    \begin{itemize}
        \item $s$ is the total sum in the pot, which always abides by the following: $s = w_{\text{留}} + w_{\text{轉}}$
        \item $v$ is the total number of votes on the winning side, which always abides by the following: $v = v_{\text{留}} + v_{\text{轉}}$
        \item $w_{\text{留}}$ and $w_{\text{轉}}$ are defined as such:
        \begin{enumerate}
            \item $w_{\text{留}} = s \cdot \frac{v_{\text{留}}+1}{v+1}$
            \item $w_{\text{轉}} = s \cdot \frac{v_{\text{轉}}}{v+1}$
        \end{enumerate}
        \item To say this without justification, this creates a quasi-prisoner's dilemma situation in the following box, where those who switch sides and win, win money but lose out relatively on points.
        \item The split of the $w_{\text{留}}$ and $w_{\text{轉}}$ leaves a weird case where everyone switched sides and that side won. In such a case, there'd be nobody to claim the virtual 留贏 share. That share can go to many places: (1) the poorest member, (2) the hosting tournament, (3) divided equally amongst all players\ldots (4) taken by the winning side as well\ldots
        \item The point of the dynamic is: some people care about the money, some people care about passing the motion.

    \end{itemize}

    % \item \textbf{Nonstandard number of participants.} 
    % Suppose for some reason, or out of some other consideration, you have 8 senators, but you also want to introduce a number of $n$ participants into a round, without necessarily giving them full senatorial rights. Perhaps you want to increase the complexity or the anchorage of opinions in a round. Or perhaps you want to involve the entire audience in resolving the debate. 

    % These participants may be similar to the backbenchers. 

    \item \textbf{Chairs \& Speakers} Suppose for the sake of decorum, and perhaps for the sake of maintaining accounting validity, you might want to introduce a chair, or a speaker. The chair's role, unlike in British Parliamentary debate, does not involve judging. He is only involved in making sure the outcome properly computed and resolved. 

\end{enumerate}



\section{Payout structure by debater count}
Recall that the number of senators can arguably be scaled up to any number. And also recall that the number of votes necessary to win is 1 + half of the total number of debaters.

Here we present the payout structure for different numbers of debaters and voting scenarios.


\begin{table}[H]
    \centering
    \small
    \begin{threeparttable}
    \begin{tabular}{cccccccc}    
    \toprule
    $n_{\text{debaters}}$ & $v_{\text{win}}$ & $v_{\text{留}}$ & $v_{\text{轉}}$ & $w_{\text{留}} = 100 \cdot \frac{v_{\text{留}}+1}{v_{win}+1} $ & $w_{\text{轉}} = 100\cdot\frac{v_{\text{轉}}}{v+1}$ & $\text{w}_{\text{留}/\text{人} }$ & $\text{w}_{\text{轉}/\text{人}}$ \\
    \midrule

% \midrule
6 & 4 & 0 & 4 & 20.0 & 80.0 & n/a\tnote{a} & 20.0 \\
6 & 4 & 1 & 3 & 40.0 & 60.0 & 40.0 & 20.0 \\
6 & 4 & 2 & 2 & 60.0 & 40.0 & 30.0 & 20.0 \\
6 & 4 & 3 & 1 & 80.0 & 20.0 & 26.67 & 20.0 \\
6 & 4 & 4 & 0 & 100.0 & 0.0 & 25.0 & 0.0 \\
6 & 5 & 0 & 5 & 16.67 & 83.33 & n/a\tnote{a} & 16.67 \\
6 & 5 & 1 & 4 & 33.33 & 66.67 & 33.33 & 16.67 \\
6 & 5 & 2 & 3 & 50.0 & 50.0 & 25.0 & 16.67 \\
6 & 5 & 3 & 2 & 66.67 & 33.33 & 22.22 & 16.67 \\
6 & 5 & 4 & 1 & 83.33 & 16.67 & 20.83 & 16.67 \\
6 & 5 & 5 & 0 & 100.0 & 0.0 & 20.0 & 0.0 \\
6 & 6 & 0 & 6 & 14.29 & 85.71 & n/a\tnote{a} & 14.29 \\
6 & 6 & 1 & 5 & 28.57 & 71.43 & 28.57 & 14.29 \\
6 & 6 & 2 & 4 & 42.86 & 57.14 & 21.43 & 14.29 \\
6 & 6 & 3 & 3 & 57.14 & 42.86 & 19.05 & 14.29 \\
6 & 6 & 4 & 2 & 71.43 & 28.57 & 17.86 & 14.29 \\
6 & 6 & 5 & 1 & 85.71 & 14.29 & 17.14 & 14.29 \\
6 & 6 & 6 & 0 & 100.0 & 0.0 & 16.67 & 0.0 \\
\bottomrule
\end{tabular}
\begin{tablenotes}
    \item[a] n/a, because there are literally no participants who won who stayed on the position that they started with. 
    Obviously this is a degenerate and highly unlikely case. But we still need to figure out how to handle the payout which has no one to receive it. It can either (1) be distributed amongst those who won, (2) go to anyone on the losing side who stuck to their guns, or (3) go to the poorest member, or (4) to the tournament organizer. We are not too interested in the details. 
\end{tablenotes}
\end{threeparttable}

\caption{Payout structure for different numbers of debaters and voting scenarios}
\end{table}

\begin{table}[H]
    \centering
    \small
    \begin{threeparttable}
    \begin{tabular}{cccccccc}
    \toprule
    $n_{\text{debaters}}$ & $v_{\text{win}}$ & $v_{\text{留}}$ & $v_{\text{轉}}$ & $w_{\text{留}} = 100 \cdot \frac{v_{\text{留}}+1}{v_{win}+1} $ & $w_{\text{轉}} = 100\cdot\frac{v_{\text{轉}}}{v+1}$ & $\text{w}_{\text{留}/\text{人} }$ & $\text{w}_{\text{轉}/\text{人}}$ \\
    \midrule
7 & 4 & 0 & 4 & 20.0 & 80.0 & n/a\tnote{a} & 20.0 \\
7 & 4 & 1 & 3 & 40.0 & 60.0 & 40.0 & 20.0 \\
7 & 4 & 2 & 2 & 60.0 & 40.0 & 30.0 & 20.0 \\
7 & 4 & 3 & 1 & 80.0 & 20.0 & 26.67 & 20.0 \\
7 & 4 & 4 & 0 & 100.0 & 0.0 & 25.0 & 0.0 \\
7 & 5 & 0 & 5 & 16.67 & 83.33 & n/a\tnote{a} & 16.67 \\
7 & 5 & 1 & 4 & 33.33 & 66.67 & 33.33 & 16.67 \\
7 & 5 & 2 & 3 & 50.0 & 50.0 & 25.0 & 16.67 \\
7 & 5 & 3 & 2 & 66.67 & 33.33 & 22.22 & 16.67 \\
7 & 5 & 4 & 1 & 83.33 & 16.67 & 20.83 & 16.67 \\
7 & 5 & 5 & 0 & 100.0 & 0.0 & 20.0 & 0.0 \\
7 & 6 & 0 & 6 & 14.29 & 85.71 & n/a\tnote{a} & 14.29 \\
7 & 6 & 1 & 5 & 28.57 & 71.43 & 28.57 & 14.29 \\
7 & 6 & 2 & 4 & 42.86 & 57.14 & 21.43 & 14.29 \\
7 & 6 & 3 & 3 & 57.14 & 42.86 & 19.05 & 14.29 \\
7 & 6 & 4 & 2 & 71.43 & 28.57 & 17.86 & 14.29 \\
7 & 6 & 5 & 1 & 85.71 & 14.29 & 17.14 & 14.29 \\
7 & 6 & 6 & 0 & 100.0 & 0.0 & 16.67 & 0.0 \\
7 & 7 & 0 & 7 & 12.5 & 87.5 & n/a\tnote{a} & 12.5 \\
7 & 7 & 1 & 6 & 25.0 & 75.0 & 25.0 & 12.5 \\
7 & 7 & 2 & 5 & 37.5 & 62.5 & 18.75 & 12.5 \\
7 & 7 & 3 & 4 & 50.0 & 50.0 & 16.67 & 12.5 \\
7 & 7 & 4 & 3 & 62.5 & 37.5 & 15.62 & 12.5 \\
7 & 7 & 5 & 2 & 75.0 & 25.0 & 15.0 & 12.5 \\
7 & 7 & 6 & 1 & 87.5 & 12.5 & 14.58 & 12.5 \\
7 & 7 & 7 & 0 & 100.0 & 0.0 & 14.29 & 0.0 \\
\bottomrule
\end{tabular}
\begin{tablenotes}
    \item[a] n/a indicates no 留 voters to receive the payout.
\end{tablenotes}
\caption{Payout structure for different numbers of debaters and voting scenarios}
\end{threeparttable}
\end{table}


\begin{table}[H]
    \centering
    \small
    \begin{threeparttable}
    \begin{tabular}{cccccccc}
    \toprule
    $n_{\text{debaters}}$ & $v_{\text{win}}$ & $v_{\text{留}}$ & $v_{\text{轉}}$ & $w_{\text{留}} = 100 \cdot \frac{v_{\text{留}}+1}{v_{win}+1} $ & $w_{\text{轉}} = 100\cdot\frac{v_{\text{轉}}}{v+1}$ & $\text{w}_{\text{留}/\text{人} }$ & $\text{w}_{\text{轉}/\text{人}}$ \\
    \midrule
8 & 5 & 0 & 5 & 16.67 & 83.33 & n/a\tnote{a} & 16.67 \\
8 & 5 & 1 & 4 & 33.33 & 66.67 & 33.33 & 16.67 \\
8 & 5 & 2 & 3 & 50.0 & 50.0 & 25.0 & 16.67 \\
8 & 5 & 3 & 2 & 66.67 & 33.33 & 22.22 & 16.67 \\
8 & 5 & 4 & 1 & 83.33 & 16.67 & 20.83 & 16.67 \\
8 & 5 & 5 & 0 & 100.0 & 0.0 & 20.0 & 0.0 \\
8 & 6 & 0 & 6 & 14.29 & 85.71 & n/a\tnote{a} & 14.29 \\
8 & 6 & 1 & 5 & 28.57 & 71.43 & 28.57 & 14.29 \\
8 & 6 & 2 & 4 & 42.86 & 57.14 & 21.43 & 14.29 \\
8 & 6 & 3 & 3 & 57.14 & 42.86 & 19.05 & 14.29 \\
8 & 6 & 4 & 2 & 71.43 & 28.57 & 17.86 & 14.29 \\
8 & 6 & 5 & 1 & 85.71 & 14.29 & 17.14 & 14.29 \\
8 & 6 & 6 & 0 & 100.0 & 0.0 & 16.67 & 0.0 \\
8 & 7 & 0 & 7 & 12.5 & 87.5 & n/a\tnote{a} & 12.5 \\
8 & 7 & 1 & 6 & 25.0 & 75.0 & 25.0 & 12.5 \\
8 & 7 & 2 & 5 & 37.5 & 62.5 & 18.75 & 12.5 \\
8 & 7 & 3 & 4 & 50.0 & 50.0 & 16.67 & 12.5 \\
8 & 7 & 4 & 3 & 62.5 & 37.5 & 15.62 & 12.5 \\
8 & 7 & 5 & 2 & 75.0 & 25.0 & 15.0 & 12.5 \\
8 & 7 & 6 & 1 & 87.5 & 12.5 & 14.58 & 12.5 \\
8 & 7 & 7 & 0 & 100.0 & 0.0 & 14.29 & 0.0 \\
8 & 8 & 0 & 8 & 11.11 & 88.89 & n/a\tnote{a} & 11.11 \\
8 & 8 & 1 & 7 & 22.22 & 77.78 & 22.22 & 11.11 \\
8 & 8 & 2 & 6 & 33.33 & 66.67 & 16.67 & 11.11 \\
8 & 8 & 3 & 5 & 44.44 & 55.56 & 14.81 & 11.11 \\
8 & 8 & 4 & 4 & 55.56 & 44.44 & 13.89 & 11.11 \\
8 & 8 & 5 & 3 & 66.67 & 33.33 & 13.33 & 11.11 \\
8 & 8 & 6 & 2 & 77.78 & 22.22 & 12.96 & 11.11 \\
8 & 8 & 7 & 1 & 88.89 & 11.11 & 12.7 & 11.11 \\
8 & 8 & 8 & 0 & 100.0 & 0.0 & 12.5 & 0.0 \\
\bottomrule
\end{tabular}
\begin{tablenotes}
    \item[a] n/a indicates no 留 voters to receive the payout.
\end{tablenotes}
\caption{Payout structure for different numbers of debaters and voting scenarios}
\end{threeparttable}
\end{table}

\begin{table}[H]
    \centering
    \small
    \begin{threeparttable}
    \begin{tabular}{cccccccc}
    \toprule
    $n_{\text{debaters}}$ & $v_{\text{win}}$ & $v_{\text{留}}$ & $v_{\text{轉}}$ & $w_{\text{留}} = 100 \cdot \frac{v_{\text{留}}+1}{v_{win}+1} $ & $w_{\text{轉}} = 100\cdot\frac{v_{\text{轉}}}{v+1}$ & $\text{w}_{\text{留}/\text{人} }$ & $\text{w}_{\text{轉}/\text{人}}$ \\
    \midrule
9 & 5 & 0 & 5 & 16.67 & 83.33 & n/a\tnote{a} & 16.67 \\
9 & 5 & 1 & 4 & 33.33 & 66.67 & 33.33 & 16.67 \\
9 & 5 & 2 & 3 & 50.0 & 50.0 & 25.0 & 16.67 \\
9 & 5 & 3 & 2 & 66.67 & 33.33 & 22.22 & 16.67 \\
9 & 5 & 4 & 1 & 83.33 & 16.67 & 20.83 & 16.67 \\
9 & 5 & 5 & 0 & 100.0 & 0.0 & 20.0 & 0.0 \\
9 & 6 & 0 & 6 & 14.29 & 85.71 & n/a\tnote{a} & 14.29 \\
9 & 6 & 1 & 5 & 28.57 & 71.43 & 28.57 & 14.29 \\
9 & 6 & 2 & 4 & 42.86 & 57.14 & 21.43 & 14.29 \\
9 & 6 & 3 & 3 & 57.14 & 42.86 & 19.05 & 14.29 \\
9 & 6 & 4 & 2 & 71.43 & 28.57 & 17.86 & 14.29 \\
9 & 6 & 5 & 1 & 85.71 & 14.29 & 17.14 & 14.29 \\
9 & 6 & 6 & 0 & 100.0 & 0.0 & 16.67 & 0.0 \\
9 & 7 & 0 & 7 & 12.5 & 87.5 & n/a\tnote{a} & 12.5 \\
9 & 7 & 1 & 6 & 25.0 & 75.0 & 25.0 & 12.5 \\
9 & 7 & 2 & 5 & 37.5 & 62.5 & 18.75 & 12.5 \\
9 & 7 & 3 & 4 & 50.0 & 50.0 & 16.67 & 12.5 \\
9 & 7 & 4 & 3 & 62.5 & 37.5 & 15.62 & 12.5 \\
9 & 7 & 5 & 2 & 75.0 & 25.0 & 15.0 & 12.5 \\
9 & 7 & 6 & 1 & 87.5 & 12.5 & 14.58 & 12.5 \\
9 & 7 & 7 & 0 & 100.0 & 0.0 & 14.29 & 0.0 \\
9 & 8 & 0 & 8 & 11.11 & 88.89 & n/a\tnote{a} & 11.11 \\
9 & 8 & 1 & 7 & 22.22 & 77.78 & 22.22 & 11.11 \\
9 & 8 & 2 & 6 & 33.33 & 66.67 & 16.67 & 11.11 \\
9 & 8 & 3 & 5 & 44.44 & 55.56 & 14.81 & 11.11 \\
9 & 8 & 4 & 4 & 55.56 & 44.44 & 13.89 & 11.11 \\
9 & 8 & 5 & 3 & 66.67 & 33.33 & 13.33 & 11.11 \\
9 & 8 & 6 & 2 & 77.78 & 22.22 & 12.96 & 11.11 \\
9 & 8 & 7 & 1 & 88.89 & 11.11 & 12.7 & 11.11 \\
9 & 8 & 8 & 0 & 100.0 & 0.0 & 12.5 & 0.0 \\
9 & 9 & 0 & 9 & 10.0 & 90.0 & n/a\tnote{a} & 10.0 \\
9 & 9 & 1 & 8 & 20.0 & 80.0 & 20.0 & 10.0 \\
9 & 9 & 2 & 7 & 30.0 & 70.0 & 15.0 & 10.0 \\
9 & 9 & 3 & 6 & 40.0 & 60.0 & 13.33 & 10.0 \\
9 & 9 & 4 & 5 & 50.0 & 50.0 & 12.5 & 10.0 \\
9 & 9 & 5 & 4 & 60.0 & 40.0 & 12.0 & 10.0 \\
9 & 9 & 6 & 3 & 70.0 & 30.0 & 11.67 & 10.0 \\
9 & 9 & 7 & 2 & 80.0 & 20.0 & 11.43 & 10.0 \\
9 & 9 & 8 & 1 & 90.0 & 10.0 & 11.25 & 10.0 \\
9 & 9 & 9 & 0 & 100.0 & 0.0 & 11.11 & 0.0 \\
\bottomrule
\end{tabular}
\begin{tablenotes}
    \item[a] n/a indicates no 留 voters to receive the payout.
\end{tablenotes}
\caption{Payout structure for different numbers of debaters and voting scenarios}
\end{threeparttable}
\end{table}

\begin{table}[H]
    \centering
    \small
    \begin{threeparttable}
    \begin{tabular}{cccccccc}
    \toprule
    $n_{\text{debaters}}$ & $v_{\text{win}}$ & $v_{\text{留}}$ & $v_{\text{轉}}$ & $w_{\text{留}} = 100 \cdot \frac{v_{\text{留}}+1}{v_{win}+1} $ & $w_{\text{轉}} = 100\cdot\frac{v_{\text{轉}}}{v+1}$ & $\text{w}_{\text{留}/\text{人} }$ & $\text{w}_{\text{轉}/\text{人}}$ \\
    \midrule
10 & 6 & 0 & 6 & 14.29 & 85.71 & n/a\tnote{a} & 14.29 \\
10 & 6 & 1 & 5 & 28.57 & 71.43 & 28.57 & 14.29 \\
10 & 6 & 2 & 4 & 42.86 & 57.14 & 21.43 & 14.29 \\
10 & 6 & 3 & 3 & 57.14 & 42.86 & 19.05 & 14.29 \\
10 & 6 & 4 & 2 & 71.43 & 28.57 & 17.86 & 14.29 \\
10 & 6 & 5 & 1 & 85.71 & 14.29 & 17.14 & 14.29 \\
10 & 6 & 6 & 0 & 100.0 & 0.0 & 16.67 & 0.0 \\
10 & 7 & 0 & 7 & 12.5 & 87.5 & n/a\tnote{a} & 12.5 \\
10 & 7 & 1 & 6 & 25.0 & 75.0 & 25.0 & 12.5 \\
10 & 7 & 2 & 5 & 37.5 & 62.5 & 18.75 & 12.5 \\
10 & 7 & 3 & 4 & 50.0 & 50.0 & 16.67 & 12.5 \\
10 & 7 & 4 & 3 & 62.5 & 37.5 & 15.62 & 12.5 \\
10 & 7 & 5 & 2 & 75.0 & 25.0 & 15.0 & 12.5 \\
10 & 7 & 6 & 1 & 87.5 & 12.5 & 14.58 & 12.5 \\
10 & 7 & 7 & 0 & 100.0 & 0.0 & 14.29 & 0.0 \\
10 & 8 & 0 & 8 & 11.11 & 88.89 & n/a\tnote{a} & 11.11 \\
10 & 8 & 1 & 7 & 22.22 & 77.78 & 22.22 & 11.11 \\
10 & 8 & 2 & 6 & 33.33 & 66.67 & 16.67 & 11.11 \\
10 & 8 & 3 & 5 & 44.44 & 55.56 & 14.81 & 11.11 \\
10 & 8 & 4 & 4 & 55.56 & 44.44 & 13.89 & 11.11 \\
10 & 8 & 5 & 3 & 66.67 & 33.33 & 13.33 & 11.11 \\
10 & 8 & 6 & 2 & 77.78 & 22.22 & 12.96 & 11.11 \\
10 & 8 & 7 & 1 & 88.89 & 11.11 & 12.7 & 11.11 \\
10 & 8 & 8 & 0 & 100.0 & 0.0 & 12.5 & 0.0 \\
10 & 9 & 0 & 9 & 10.0 & 90.0 & n/a\tnote{a} & 10.0 \\
10 & 9 & 1 & 8 & 20.0 & 80.0 & 20.0 & 10.0 \\
10 & 9 & 2 & 7 & 30.0 & 70.0 & 15.0 & 10.0 \\
10 & 9 & 3 & 6 & 40.0 & 60.0 & 13.33 & 10.0 \\
10 & 9 & 4 & 5 & 50.0 & 50.0 & 12.5 & 10.0 \\
10 & 9 & 5 & 4 & 60.0 & 40.0 & 12.0 & 10.0 \\
10 & 9 & 6 & 3 & 70.0 & 30.0 & 11.67 & 10.0 \\
10 & 9 & 7 & 2 & 80.0 & 20.0 & 11.43 & 10.0 \\
10 & 9 & 8 & 1 & 90.0 & 10.0 & 11.25 & 10.0 \\
10 & 9 & 9 & 0 & 100.0 & 0.0 & 11.11 & 0.0 \\
10 & 10 & 0 & 10 & 9.09 & 90.91 & n/a\tnote{a} & 9.09 \\
10 & 10 & 1 & 9 & 18.18 & 81.82 & 18.18 & 9.09 \\
10 & 10 & 2 & 8 & 27.27 & 72.73 & 13.64 & 9.09 \\
10 & 10 & 3 & 7 & 36.36 & 63.64 & 12.12 & 9.09 \\
10 & 10 & 4 & 6 & 45.45 & 54.55 & 11.36 & 9.09 \\
10 & 10 & 5 & 5 & 54.55 & 45.45 & 10.91 & 9.09 \\
10 & 10 & 6 & 4 & 63.64 & 36.36 & 10.61 & 9.09 \\
10 & 10 & 7 & 3 & 72.73 & 27.27 & 10.39 & 9.09 \\
10 & 10 & 8 & 2 & 81.82 & 18.18 & 10.23 & 9.09 \\
10 & 10 & 9 & 1 & 90.91 & 9.09 & 10.1 & 9.09 \\
10 & 10 & 10 & 0 & 100.0 & 0.0 & 10.0 & 0.0 \\
\bottomrule
\end{tabular}
\begin{tablenotes}
    \item[a] n/a indicates no 留 voters to receive the payout.
\end{tablenotes}
\caption{Payout structure for different numbers of debaters and voting scenarios}
\end{threeparttable}
\end{table}



\begin{table}[H]
    \centering
    \small
    \begin{threeparttable}
    \begin{tabular}{cccccccc}
    \toprule    \toprule
    $n_{\text{debaters}}$ & $v_{\text{win}}$ & $v_{\text{留}}$ & $v_{\text{轉}}$ & $w_{\text{留}} = 100 \cdot \frac{v_{\text{留}}+1}{v_{win}+1} $ & $w_{\text{轉}} = 100\cdot\frac{v_{\text{轉}}}{v+1}$ & $\text{w}_{\text{留}/\text{人} }$ & $\text{w}_{\text{轉}/\text{人}}$ \\
    \midrule
11 & 6 & 0 & 6 & 14.29 & 85.71 & n/a\tnote{a} & 14.29 \\
11 & 6 & 1 & 5 & 28.57 & 71.43 & 28.57 & 14.29 \\
11 & 6 & 2 & 4 & 42.86 & 57.14 & 21.43 & 14.29 \\
11 & 6 & 3 & 3 & 57.14 & 42.86 & 19.05 & 14.29 \\
11 & 6 & 4 & 2 & 71.43 & 28.57 & 17.86 & 14.29 \\
11 & 6 & 5 & 1 & 85.71 & 14.29 & 17.14 & 14.29 \\
11 & 6 & 6 & 0 & 100.0 & 0.0 & 16.67 & 0.0 \\
11 & 7 & 0 & 7 & 12.5 & 87.5 & n/a\tnote{a} & 12.5 \\
11 & 7 & 1 & 6 & 25.0 & 75.0 & 25.0 & 12.5 \\
11 & 7 & 2 & 5 & 37.5 & 62.5 & 18.75 & 12.5 \\
11 & 7 & 3 & 4 & 50.0 & 50.0 & 16.67 & 12.5 \\
11 & 7 & 4 & 3 & 62.5 & 37.5 & 15.62 & 12.5 \\
11 & 7 & 5 & 2 & 75.0 & 25.0 & 15.0 & 12.5 \\
11 & 7 & 6 & 1 & 87.5 & 12.5 & 14.58 & 12.5 \\
11 & 7 & 7 & 0 & 100.0 & 0.0 & 14.29 & 0.0 \\
11 & 8 & 0 & 8 & 11.11 & 88.89 & n/a\tnote{a} & 11.11 \\
11 & 8 & 1 & 7 & 22.22 & 77.78 & 22.22 & 11.11 \\
11 & 8 & 2 & 6 & 33.33 & 66.67 & 16.67 & 11.11 \\
11 & 8 & 3 & 5 & 44.44 & 55.56 & 14.81 & 11.11 \\
11 & 8 & 4 & 4 & 55.56 & 44.44 & 13.89 & 11.11 \\
11 & 8 & 5 & 3 & 66.67 & 33.33 & 13.33 & 11.11 \\
11 & 8 & 6 & 2 & 77.78 & 22.22 & 12.96 & 11.11 \\
11 & 8 & 7 & 1 & 88.89 & 11.11 & 12.7 & 11.11 \\
11 & 8 & 8 & 0 & 100.0 & 0.0 & 12.5 & 0.0 \\
11 & 9 & 0 & 9 & 10.0 & 90.0 & n/a\tnote{a} & 10.0 \\
11 & 9 & 1 & 8 & 20.0 & 80.0 & 20.0 & 10.0 \\
11 & 9 & 2 & 7 & 30.0 & 70.0 & 15.0 & 10.0 \\
11 & 9 & 3 & 6 & 40.0 & 60.0 & 13.33 & 10.0 \\
11 & 9 & 4 & 5 & 50.0 & 50.0 & 12.5 & 10.0 \\
11 & 9 & 5 & 4 & 60.0 & 40.0 & 12.0 & 10.0 \\
11 & 9 & 6 & 3 & 70.0 & 30.0 & 11.67 & 10.0 \\
11 & 9 & 7 & 2 & 80.0 & 20.0 & 11.43 & 10.0 \\
11 & 9 & 8 & 1 & 90.0 & 10.0 & 11.25 & 10.0 \\
11 & 9 & 9 & 0 & 100.0 & 0.0 & 11.11 & 0.0 \\
11 & 10 & 0 & 10 & 9.09 & 90.91 & n/a\tnote{a} & 9.09 \\
11 & 10 & 1 & 9 & 18.18 & 81.82 & 18.18 & 9.09 \\
11 & 10 & 2 & 8 & 27.27 & 72.73 & 13.64 & 9.09 \\
11 & 10 & 3 & 7 & 36.36 & 63.64 & 12.12 & 9.09 \\
11 & 10 & 4 & 6 & 45.45 & 54.55 & 11.36 & 9.09 \\
11 & 10 & 5 & 5 & 54.55 & 45.45 & 10.91 & 9.09 \\
11 & 10 & 6 & 4 & 63.64 & 36.36 & 10.61 & 9.09 \\
11 & 10 & 7 & 3 & 72.73 & 27.27 & 10.39 & 9.09 \\
11 & 10 & 8 & 2 & 81.82 & 18.18 & 10.23 & 9.09 \\
11 & 10 & 9 & 1 & 90.91 & 9.09 & 10.1 & 9.09 \\
11 & 10 & 10 & 0 & 100.0 & 0.0 & 10.0 & 0.0 \\
11 & 11 & 0 & 11 & 8.33 & 91.67 & n/a\tnote{a} & 8.33 \\
11 & 11 & 1 & 10 & 16.67 & 83.33 & 16.67 & 8.33 \\
11 & 11 & 2 & 9 & 25.0 & 75.0 & 12.5 & 8.33 \\
11 & 11 & 3 & 8 & 33.33 & 66.67 & 11.11 & 8.33 \\
11 & 11 & 4 & 7 & 41.67 & 58.33 & 10.42 & 8.33 \\
11 & 11 & 5 & 6 & 50.0 & 50.0 & 10.0 & 8.33 \\
11 & 11 & 6 & 5 & 58.33 & 41.67 & 9.72 & 8.33 \\
11 & 11 & 7 & 4 & 66.67 & 33.33 & 9.52 & 8.33 \\
11 & 11 & 8 & 3 & 75.0 & 25.0 & 9.38 & 8.33 \\
11 & 11 & 9 & 2 & 83.33 & 16.67 & 9.26 & 8.33 \\
11 & 11 & 10 & 1 & 91.67 & 8.33 & 9.17 & 8.33 \\
11 & 11 & 11 & 0 & 100.0 & 0.0 & 9.09 & 0.0 \\

\bottomrule
\end{tabular}
\begin{tablenotes}
    \item[a] n/a indicates no 留 voters to receive the payout.
\end{tablenotes}
\caption{Payout structure for different numbers of debaters and voting scenarios}
\end{threeparttable}
\end{table}




\begin{table}[H]
    \centering
    \small
    \begin{threeparttable}
    \begin{tabular}{cccccccc}
    \toprule    \toprule
    $n_{\text{debaters}}$ & $v_{\text{win}}$ & $v_{\text{留}}$ & $v_{\text{轉}}$ & $w_{\text{留}} = 100 \cdot \frac{v_{\text{留}}+1}{v_{win}+1} $ & $w_{\text{轉}} = 100\cdot\frac{v_{\text{轉}}}{v+1}$ & $\text{w}_{\text{留}/\text{人} }$ & $\text{w}_{\text{轉}/\text{人}}$ \\
    \midrule
12 & 7 & 0 & 7 & 12.5 & 87.5 & n/a\tnote{a} & 12.5 \\
12 & 7 & 1 & 6 & 25.0 & 75.0 & 25.0 & 12.5 \\
12 & 7 & 2 & 5 & 37.5 & 62.5 & 18.75 & 12.5 \\
12 & 7 & 3 & 4 & 50.0 & 50.0 & 16.67 & 12.5 \\
12 & 7 & 4 & 3 & 62.5 & 37.5 & 15.62 & 12.5 \\
12 & 7 & 5 & 2 & 75.0 & 25.0 & 15.0 & 12.5 \\
12 & 7 & 6 & 1 & 87.5 & 12.5 & 14.58 & 12.5 \\
12 & 7 & 7 & 0 & 100.0 & 0.0 & 14.29 & 0.0 \\
12 & 8 & 0 & 8 & 11.11 & 88.89 & n/a\tnote{a} & 11.11 \\
12 & 8 & 1 & 7 & 22.22 & 77.78 & 22.22 & 11.11 \\
12 & 8 & 2 & 6 & 33.33 & 66.67 & 16.67 & 11.11 \\
12 & 8 & 3 & 5 & 44.44 & 55.56 & 14.81 & 11.11 \\
12 & 8 & 4 & 4 & 55.56 & 44.44 & 13.89 & 11.11 \\
12 & 8 & 5 & 3 & 66.67 & 33.33 & 13.33 & 11.11 \\
12 & 8 & 6 & 2 & 77.78 & 22.22 & 12.96 & 11.11 \\
12 & 8 & 7 & 1 & 88.89 & 11.11 & 12.7 & 11.11 \\
12 & 8 & 8 & 0 & 100.0 & 0.0 & 12.5 & 0.0 \\
12 & 9 & 0 & 9 & 10.0 & 90.0 & n/a\tnote{a} & 10.0 \\
12 & 9 & 1 & 8 & 20.0 & 80.0 & 20.0 & 10.0 \\
12 & 9 & 2 & 7 & 30.0 & 70.0 & 15.0 & 10.0 \\
12 & 9 & 3 & 6 & 40.0 & 60.0 & 13.33 & 10.0 \\
12 & 9 & 4 & 5 & 50.0 & 50.0 & 12.5 & 10.0 \\
12 & 9 & 5 & 4 & 60.0 & 40.0 & 12.0 & 10.0 \\
12 & 9 & 6 & 3 & 70.0 & 30.0 & 11.67 & 10.0 \\
12 & 9 & 7 & 2 & 80.0 & 20.0 & 11.43 & 10.0 \\
12 & 9 & 8 & 1 & 90.0 & 10.0 & 11.25 & 10.0 \\
12 & 9 & 9 & 0 & 100.0 & 0.0 & 11.11 & 0.0 \\
12 & 10 & 0 & 10 & 9.09 & 90.91 & n/a\tnote{a} & 9.09 \\
12 & 10 & 1 & 9 & 18.18 & 81.82 & 18.18 & 9.09 \\
12 & 10 & 2 & 8 & 27.27 & 72.73 & 13.64 & 9.09 \\
12 & 10 & 3 & 7 & 36.36 & 63.64 & 12.12 & 9.09 \\
12 & 10 & 4 & 6 & 45.45 & 54.55 & 11.36 & 9.09 \\
12 & 10 & 5 & 5 & 54.55 & 45.45 & 10.91 & 9.09 \\
12 & 10 & 6 & 4 & 63.64 & 36.36 & 10.61 & 9.09 \\
12 & 10 & 7 & 3 & 72.73 & 27.27 & 10.39 & 9.09 \\
12 & 10 & 8 & 2 & 81.82 & 18.18 & 10.23 & 9.09 \\
12 & 10 & 9 & 1 & 90.91 & 9.09 & 10.1 & 9.09 \\
12 & 10 & 10 & 0 & 100.0 & 0.0 & 10.0 & 0.0 \\
12 & 11 & 0 & 11 & 8.33 & 91.67 & n/a\tnote{a} & 8.33 \\
12 & 11 & 1 & 10 & 16.67 & 83.33 & 16.67 & 8.33 \\
12 & 11 & 2 & 9 & 25.0 & 75.0 & 12.5 & 8.33 \\
12 & 11 & 3 & 8 & 33.33 & 66.67 & 11.11 & 8.33 \\
12 & 11 & 4 & 7 & 41.67 & 58.33 & 10.42 & 8.33 \\
12 & 11 & 5 & 6 & 50.0 & 50.0 & 10.0 & 8.33 \\
12 & 11 & 6 & 5 & 58.33 & 41.67 & 9.72 & 8.33 \\
12 & 11 & 7 & 4 & 66.67 & 33.33 & 9.52 & 8.33 \\
12 & 11 & 8 & 3 & 75.0 & 25.0 & 9.38 & 8.33 \\
12 & 11 & 9 & 2 & 83.33 & 16.67 & 9.26 & 8.33 \\
12 & 11 & 10 & 1 & 91.67 & 8.33 & 9.17 & 8.33 \\
12 & 11 & 11 & 0 & 100.0 & 0.0 & 9.09 & 0.0 \\
12 & 12 & 0 & 12 & 7.69 & 92.31 & n/a\tnote{a} & 7.69 \\
12 & 12 & 1 & 11 & 15.38 & 84.62 & 15.38 & 7.69 \\
12 & 12 & 2 & 10 & 23.08 & 76.92 & 11.54 & 7.69 \\
12 & 12 & 3 & 9 & 30.77 & 69.23 & 10.26 & 7.69 \\
12 & 12 & 4 & 8 & 38.46 & 61.54 & 9.62 & 7.69 \\
12 & 12 & 5 & 7 & 46.15 & 53.85 & 9.23 & 7.69 \\
12 & 12 & 6 & 6 & 53.85 & 46.15 & 8.97 & 7.69 \\
12 & 12 & 7 & 5 & 61.54 & 38.46 & 8.79 & 7.69 \\
12 & 12 & 8 & 4 & 69.23 & 30.77 & 8.65 & 7.69 \\
12 & 12 & 9 & 3 & 76.92 & 23.08 & 8.55 & 7.69 \\
12 & 12 & 10 & 2 & 84.62 & 15.38 & 8.46 & 7.69 \\
12 & 12 & 11 & 1 & 92.31 & 7.69 & 8.39 & 7.69 \\
12 & 12 & 12 & 0 & 100.0 & 0.0 & 8.33 & 0.0 \\

\bottomrule
\end{tabular}
\begin{tablenotes}
    \item[a] n/a indicates no 留 voters to receive the payout.
\end{tablenotes}
\caption{Payout structure for different numbers of debaters and voting scenarios}
\end{threeparttable}
\end{table}

