


\section{Emergent phenomenon}

It is very likely that this set of rules will yield emergent phenomena. It is my sincere hope that the game, along with its emergent phenomena, will beget and engender virtues in its particpants: honour, intellectualism, honesty, duty, and most importantly, the prudent and intelligent use of force. The pen that is dancing is only as mighty as the sword that is at rest.


\subsection{Bribery}

Consider the case where in the first round of votes, the vote distribution is 7 Y and 1 N. This is a highly stable state. The 7 participants have no incentive to switch their vote. Anybody switching would be giving up 3 points for at best 1 point, and the jackpot sum that's already in the bag. 

Then, what's the point of the debate here? There is no point. There would only be a point to debate if there's a destablizing force of some sort. 

If things have been left to their own devices, then the inevitable solution will emerge by itself - bribery, vote-buying.  

Vote-buying, from other senators, and only from other senators, serves as a reasonable destablizing force. 

If the form of vote-buying enabled involves the inflation of the supply of votes, then debate will become a simple matter of money takes all. This is not a good thing. Vote inflation must not be allowed. It destroys the need to debate, the need to persuade, and the need to offer a bribe. It eliminates politicking and favour-trading - which is the whole point of this exercise. But if vote-buying is restricted to the form of bribing other senators, then there will still be a tug of war between conscience and monetary self-interest. 


Since we are dealing with actual money, vote-buying is going to be particularly impactful on less well-off participants. This naturally means there would be a natural incentive for poor participants to join the tournament - for the promise of monetary prises. Just like how the poker game inevitably attracts the poor university student.


One might wonder, does one need to introduce a mechanism to enforce the promises of bribery? The answer is no. In fact, it absolutely should not be introduced. For if you do, then it eliminates the possibility of creating the culture of honour and promisekeeping. Furthermore, dishonest and lying bribe-offerers will likely be punished as the tournament progresses. Reiterated game theory dynamic secures it. If you cheated someone of a promised bribe, what are the chances you will maintain your standing in the tournament? Or in another tournament? 

On the other hand, perhaps there is someone you really want to screw over - either because of personal vendetta or because of the repugnance of their views. In these case, one might be highly motivated to engage in treachery.

The capacity for treachery is something we should cultivate—even if its victims find it distasteful. Participants must internalize this, not only because treachery is an unavoidable feature of life, especially in high politics, but also because those with lofty ideals and ambitions will find it a dagger they cannot do without. If our goal is to foster a culture of trust and honour, as well as the capacity to betray and wound—like the Roman senators of old—then we must give participants the space to exercise both. Most importantly, the point is that a man is good not because he is weak and is unable to inflict pain, but because he is strong and is able to do harm, but choses not to do so. Si vis pacem, para bellum.


It is therefore no overstatement to say that bribery is the mother of honour in this game. 


This begets an interesting question: why are bribes so frowned upon in modern electoral arena? 

The most promising answer is that not all bribes are identical. The follow-up is that favour-trading is the equivalent of bribes that we described and enabled in our debate game - and that is never disallowed. Indeed, favour-trading is what politics is all about. If disallowed, politics disintegrates and ceases to be. 


So why do we not allow the ordinary voter to sell his vote? Why do we not allow the rich to buy votes from the poor? It seems a very direct way to redistribute wealth, and certainly seems far better than the current system of wealth redistribution through government policy in areas of interest, or the electoral campaign phenomenon where the rich donates huge sums to candidates to either fatten their pockets or to fund their advertisement efforts. Neither the state nor the people are enriched in these transactions. 


It seems particularly intuitive why enabling the buying of votes in parliament is not a good idea - it should seem too powerful a mechanism to reorganize interests - it's like staring directly into the sun. To emerge from intuition land, and get into something more rigourous, allowing the bribing of parliamentarians enable embezzlement and cronyism. Embezzlement take the following the form: the government, empowered by the bribed parliamentarian of transferring funds to a service provider whose service is purchased by the government. The parliamentarian then receives a bribe, either directly in the form of cash, or in the form of a kickback. The government is robbed. 

The key mechanism is that the parliamentarian increased the price the government is willing to pay for the service than the government would have paid if there was no bribe. The difference is then split between the service-provider and the bribe paid to the parliamentarian. The service-provider is enriched, the parliamentarian is enriched, and the government is impoverished. 

Calling this embezzlement is probably mildly misleading in the sense it seems to suggest it occurs far less often than one would expect. In reality, this is just lobbying - and it undoubtedly happens every day.


Does this mechanism still manifest in some other form if it is the ordinary voter who's bribed, and not the parliamentarian? The voter merely elects the parliamentarian, but the parliamentarian doesn't get any kickbacks - the "kickback" is already spent on buying the votes from the voter, and for the vote-buyer to recuperate his expenditure, the parliamentarian must somehow spend government budget funds more extravagantly than he would have done otherwise on some service from the service-provider. Is the parliamentarian more, or less incentivized to do this, than he would have been if he was not bribed/lobbied? 



So I think it's safe to say that the case where bribery is enabled for the general voter is more resistant to embezzlement than bribes for parliamentarians. \footnote{Hi}

\subsection{Nonchalant Speech}

\subsection{Militants}


% \subsection{Evolutionary Game Theoretic Dynamics}


\begin{itemize}
    \item \textbf{Honour}
    \item \textbf{Motivations}
\end{itemize}


Charm, multiround game theoretic dynamics, politics 


\subsection{Virtues bred}
virtues bred 






